% !TeX root = ../libro.tex
% !TeX encoding = utf8

\setchapterpreamble[c][0.75\linewidth]{% 
	\sffamily
  % Al inicio de cada capítulo puede incluirse un breve resumen. Esto es opcional.
	\dictum[David Hilbert]{¿Quién de nosotros no se alegraría de levantar el velo tras el que se oculta el futuro; de echar una mirada a los próximos avances de nuestra ciencia y a los secretos de su desarrollo durante los siglos futuros?}

  \par\bigskip
}
\chapter{Una cortina hacia el futuro}\label{ch:Hilbert-1}

Con estas ambiciosas palabras comenzaba David Hilbert su conferencia desde el estrado de la universidad de La Sorbona, París, dando
así comienzo al Congreso Internacional de Matemáticos de 1900.
Dicha conferencia se distinguiría notablemente del resto de Congresos en los que ya había participado, debido a su gran influencia 
en la matemática de la época y a la ambición de la misma, que ha conseguido dejarnos retales hasta la actualidad. 


\section{Los objetivos}

Hilbert tuvo como principal reto en su discurso el crear un punto de inflexión en la matemática de la época, tratando de trazar 
una dirección y un sentido hacia el avance y el futuro de la matemática. Dicho objetivo lo abordó mediante la propuesta de una lista de 
23 problemas, ya que el propio Hilbert afirmaba que las matemáticas avanzaban proponiendo problemas, y que éstos en sí mismos 
son un signo de que la disciplina sigue viva.

Dicho programa ha tenido tal sonoridad y repercusión en la matemática del siglo XX debido tanto a la diversidad de campos que abarca, 
como a la propia influencia del conferenciante. Tal ha sido dicha importancia, que varios de estos problemas están aún sin resolver, 
en vistas de ideas y planteamientos matemáticos que aún no han sido organizados de manera correcta, o inclusive de nuevos campos y 
descubrimientos ocultos tras el velo de la ignorancia.


\section{La lista}

La recopilación de dichos 23 problemas abarca no sólo el campo que nos concierne, sino también otros tangenciales que eran de especial 
interés para Hilbert, como lo era la física. En este campo presentó especial interés gracias en gran medida a su buen amigo Hermann Minkowski,
y gracias a él estuvo activo incluso muchos años después de la muerte de su compañero, participando en los principales avances de la física del momento.

Dentro de los problemas de las matemáticas que presenta, podemos ver varios campos de especial interés en la matemática de la época, tocando geometrías no euclídeas
(todo un campo revolucionario de la época), un comienzo de atisbo en el análisis funcional (fundado en gran parte por el propio David Hilbert),
o incluso problemas que afrontan la crisis fundacional de la matemática, un lastre del que la matemática parecía no poder librarse, pero en el que Hilbert 
tenía mucha fe y seguridad.

Como comentábamos anteriormente, Hilbert era uno de los grandes matemáticos de la época, y debido a ésto, en sus estudios trabajaba con la 
matemática latente de la época. Por ello, en la lista de problemas podemos observar que gran número de ellos (alrededor de tres cuartas partes del
programa) están constituidos por problemas de campos que el propio Hilbert trataba, y que por tanto, debían ser temas procedentes del motor de avance
y del posible futuro que pudiera tomar la matemática debido al programa. 

  
\subsection{Los 23 problemas}

En la siguiente \autoref{tb:programa} se presenta una lista de los 23 problemas anunciados por Hilbert en el Congreso Internacional de Matemáticos en París,
aunque dicha lista no fuera publicada hasta un tiempo después. Como se puede ver, muchos de ellos carecen de una explicación clara y concisa del objetivo del problema,
y por éste motivo se ha considerado que muchos de ellos eran para Hilbert una dirección que tomar en el avance de la matemática, más que un problema como tal.

\begin{longtable}{|c|p{7cm}|c|} \toprule
  \centering
  % \begin{tabular}{|c|p{7cm}|c|} 
    Problema & Enunciado & Estado \\ \midrule \midrule
    1 & La hipótesis del continuo & Solución parcial \\ \midrule 
    2 & La compatibilidad de los axiomas de la aritmética & Resuelto \\ \midrule 
    3 & La igualdad de los volúmenes de dos 
        tetraedros de la misma base y la misma altura & Resuelto \\ \midrule 
    4 & Construcción de todas las métricas cuyas rectas sean geodésicas & Demasiado genérico \\ \midrule
    5 & El concepto de Lie de grupo continuo de transformaciones sin 
        la hipótesis de la diferenciabilidad de las funciones que 
        definen el grupo & Resuelto \\ \midrule
    6 & Tratamiento matemático de los axiomas de la física & Solución parcial \\ \midrule
    7 & Irracionalidad y trascendencia de ciertos números & Resuelto \\ \midrule
    8 & Problemas de números primos & Sin resolver \\ \midrule
    9 & Demostración de la ley de reciprocidad más 
        general en cualquier campo de números & Parcialmente resuelto \\ \midrule
    10 & Determinación de la resolubilidadde la ecuación diofántica & Resuelto \\ \midrule
    11 & Formas cuadráticas con coeficientes numéricos 
        algebraicos cualesquiera & Parcialmente resuelto \\ \midrule
    12 & Extensión del teorema de Kronecker sobre campos abelianos 
        a cualquier dominio de racionalidad algebraico & Sin resolver \\ \midrule
    13 & Imposibilidad de la solución general de 7º
        grado por medio de funciones de sólo dos argumentos & Resuelto \\ \midrule
    14 & Demostración de la finitud de ciertos sistemas completos de funciones & Resuelto \\ \midrule
    15 & Fundamentación rigurosa del cálculo enumerativo de Schubert & Solución parcial\\ \midrule
    16 & Problema de la topología de curvas y superficies algebraicas & Sin resolver \\ \midrule
    17 & Expresión de formas definidas por cuadrados & Resuelto \\ \midrule
    18 & Construcción del espacio a partir de poliedros congruentes & Resuelto \\ \midrule
    19 & ¿Son siempre necesariamente analíticas las soluciones de 
        problemas regulares en el cálculo de variaciones? & Resuelto \\ \midrule
    20 & El problema general de los valores de contorno & Resuelto \\ \midrule
    21 & Demostración de la existencia de ecuaciones diferenciales
        lineales que tienen prescrito un grupo monodrómico & Resuelto \\ \midrule
    22 & Uniformización de relaciones analíticas 
        por medio de funciones automorfas & Resuelto \\ \midrule
    23 & Desarrollo adicional de los métodos del cálculo de variaciones & Sin resolver \\ \bottomrule
  % \end{tabular}
  \caption{Lista de los 23 problemas del Programa}
  \label{tb:programa}
\end{longtable}

En la tabla podemos ver que muchos de los temas que trató no son un enunciado a un problema como tal, sino más bien a un campo de investigación, el cual en su 
discurso recogía varios problemas a resolver o simplemente unas directrices que seguir.



\section{El escenario previo}

No todos los problemas fueron motivo de estudio de Hilbert. Algunos, como los problemas que tienen que ver con la función zeta de Riemann o el teorema de uniformización,
ya eran problemas resonantes de la época que no se podían ignorar como problemas relevantes para el próximo siglo. En éste sentido Hilbert quiso hacer un compendio, 
no sólo de problemas de su propio campo e inquietudes, sino también de las grandes obviedades de la época. Por éste motivo uno también puede intuir otros posibles problemas
que hubieran podido ser motivo de estudio, y que sin embargo Hilbert ignoró\footnote{Un ejemplo resonante es el problema de dar una definición formal de integral de una función.}.

Pero entre los problemas que ataviaban a la sociedad matemática de la época, uno de los temas más recurrentes era la crisis fundacional. Nos encotramos ante un marco histórico 
en el que resultados recientes hacen temblar los cimientos de la matemática\footnote{Cabe destacar como ejemplos resonantes a la teoría de conjuntos de Cantor y a la paradoja 
de Russel, entre otros.}, y por consiguiente todo campo natural sedimentado en ella. Es por ello por lo que Hilbert reclama parte del protagonismo en la escena y propone dos problemas
de fundamentación axiomática: la axiomatización de la aritmética y de la física\footnote{Hilbert tenía gran confianza y certeza en que la axiomatización de la aritmética y la demostración de 
completitud de la misma era cuestión de tiempo, y por tanto también propuso la axiomatización de la física, que era el campo en el que también estaba enfocado. Véase \cite{kreisel1976have}.}.
Éste ve como necesario e imprescindible la resolución de dicho problema en el que, a pesar de tener gran certeza en la veracidad de su consistencia, era necesario una demostración rigurosa
y completa, que evitara la futura incertidumbre de posibles contradicciones no resolubles, y que quitaran de todo sentido a las matemáticas\footnote{En este sentido se puede ver más profundamente 
en palabras de Morris Kline, en \cite{kline2000matematicas}.}.

Es por éste problema (que trataremos más adelante) junto con muchos otros, que fueron motivación para Hilbert para incluirlos en su programa, y no sólo ganar más resonancia con ellos sino 
también tratar de motivar en su solución para que el futuro de la matemática pisara con paso firme, como toda la sociedad matemática deseaba en aquella época.

\chapter{Una mancha en la cortina} \label{ch:Hilbert-2}

\section{Los defectos del programa}

Debido a la gran influencia de David Hilbert, el programa era algo que no se podía refutar ni sacar del plano principal de estudio para las próximas generaciones. Sin embargo,
no por ello significa que el programa no tuviera objeciones. En efecto, muchos de los "problemas" propuestos no fueron más que un disparo al aire; una forma de que la matemática 
de la época mirara hacia donde Hilbert quería mirar. 

Entre los problemas planteados se encuentran tanto problemas del propio estudio de Hilbert, como problemas relevantes de la época, e incluso incluyó intereses que él quería que se
investigasen, pero que desgraciadamente no pertenecían a su campo de estudio. Ésto llevó a que quedaran problemas enunciados de forma demasiado vaga, lo que provocó un desinterés 
general en la sociedad y que, por ello, no fueran influyentes en el desarrollo posterior de la materia. Esto fue lo que ocurrió con el bloque de problemas del 13 al 18, siendo dentro
de los 23 los menos coherentes, y sobre los que Hilbert tenía el tacto menos seguro.\footnote{Esto sin embargo no ocurrió con los últimos 5 problemas, en los que Hilbert tenía terreno
más firme, e incluso había avanzado en algunos de ellos. Véase ~\cite[Pág. 85]{BREZIS199876}.}  


\section{La axiomatización DE LOURDES} MAMA NO QUIERO ESTUDIAR MAS

El problema de la axiomatización de la aritmética (enunciado en el segundo problema), no tuvo la respuesta esperada que Hilbert quería. En efecto, la axiomatización de la aritmética 
era algo estrictamente necesario, pero el escepticismo de la sociedad de la época ponía en duda la mera posibilidad de cualquier avance posible.

Cierto es que en los años siguientes se dieron pequeños pasos en el avance de la consistencia de la aritmética, tales como la obra de Russel y Whitehead con los \textit{Principia Mathematica}
y su teoría de tipos\footnote{Véase \cite{an1910principia}.}, o la emergente teoría de conjuntos de Cantor. Se pudo ver en ambas que, mediante cada una de ellas por separado, se podía probar 
que ambos sistemas formales podían expresar toda la matemática conocida.



Tambien puedo hablar aqui de los avances de hilbert con el libro que se habla en la intro de Godel.


\section{La llegada de Gödel a escena} 
\dictum[Samuel Beckett]{Sam enunció: "Lo que va a decir es falso" \\ Kurt replicó: "Lo que acaba de decir es verdadero"}

% Aquí introducción histórica y formal del cálculo lógico de primer orden
En verano de 1928, el joven y recién licenciado en matemáticas Kurt Gödel, se encuentra con un libro que le hace entrar en escena: \textit{Elementos de la lógica teórica}, de 
Hilbert y Ackermann\footnote{Publicado en 1928, siendo el primer libro de texto de lógica en el sentido actual. Véase \cite{hilbert1962elementos}.}. Dicho libro no sólo se presenta un cálculo deductivo, 
sino que por primera vez se plantea la cuestión de si ese cálculo es completo o no, dejando dicho problema abierto. Debido a la inquietud que causó en la matemática de la época,
y del impacto y la novedad que causó en el propio Gödel, fue motivo suficiente para tomar como tema para su tesis doctoral el tema de la completitud del cálculo lógico de primer orden.

Para Gödel, el campo de estudio, a pesar de ser de su interés, era completamente nuevo para él, el cual le planteaba como reto a largo plazo la resolución del segundo problema
de Hilbert en el intento de formalización de la aritmética. 

Como vimos, Hilbert ya había hecho pequeños avances en el tema, llegando a construir un cálculo deductivo de primer orden, el cual fue celebrado por toda la sociedad matemática 
como un gran avance en la solución del problema. 


%\section{Bibliografía}

%\cite{Zach_2007,Fer_Castro_2020,gray2000hilbert,stewart1987problems,gray2003reto,alma991014322964704990}



\endinput
