% !TeX root = ../libro.tex
% !TeX encoding = utf8

\chapter{La completitud del cálculo lógico de primer orden}

\section{Definiciones y lemas previos}

Definamos previamente el sistema axiomático sobre el que vamos a trabajar:
\footnote{Coincide, exceptuando el principio de asociatividad (que es redundante), con lo expuesto en los apartados 1 y 10 de 
\textit{Principia Mathematica} (Véase \cite{an1910principia})}
\begin{itemize}
    \item Signos básicos primitivos:\footnote{A partir de ellos pueden definirse $\wedge$, $\rightarrow$, $\leftrightarrow$ y 
            $\exists$ del modo habitual.}
            \begin{itemize}
                \item $\lnot$
                \item $\vee$
                \item $\forall$
            \end{itemize}
    \item Axiomas formales:
            \begin{enumerate}
                \item $ \text{X} \vee \text{X} \rightarrow \text{X} $
                \item $ \text{X} \rightarrow \text{X} \vee \text{Y} $
                \item $ \text{Y} \vee \text{X} \rightarrow \text{X} \vee \text{Y} $
                \item $ (\text{X} \rightarrow \text{Y}) \rightarrow (\text{Z} \vee \text{X} \rightarrow \text{Z} \vee \text{Y}) $
                \item $ \forall \text{x Px} \rightarrow \text{Py} $
                \item $ \forall \text{x } (\text{X} \vee \text{Px}) \rightarrow \text{X} \vee \forall \text{x Px} $
            \end{enumerate}
    \item Reglas de inferencia: \footnote{No todas ellas están explícitamente formuladas en los resultados de Russel y Whitehead,
            pero todas ellas son usadas continuamente en sus deducciones.}
            \begin{enumerate}
                \item El esquema de inferencia: De $\alpha$ y $\alpha \rightarrow \beta$ se puede inferir $\beta$.
                \item La regla de sustitución para variables sentenciales y predicativas.
                \item De $\alpha(\text{x})$ puede inferirse $\forall \text{x } \alpha(\text{x})$.
                \item Unas variables individuales (libres o ligadas) pueden ser reemplazadas por cualesquiera otras, con tal de que 
                        ello no produzca ningún solapamiento de los alcances de las variables designadas mediante el mismo signo.
            \end{enumerate}
\end{itemize}

Para las deducciones que van a proceder, también es conveniente establecer algunas abreviaturas en calidad de notación:

\begin{enumerate}
    \item Las notaciones $\pi_1, \thinspace \pi_2, \thinspace \pi_3, \thinspace \rho, $ ... designan prefijos\footnote{Con prefijos 
    se hace referencia a conjuntos de signos básicos primitivos.} cualesquiera, es decir, filas de signos
            finitas de la forma $\forall \text{x} \exists \text{y}, \thinspace \forall \text{x} \forall \text{y} \exists \text{z} \forall \text{u},$ ...
    \item Las letras alemanas minúsculas $\mathfrak{x, \thinspace y, \thinspace u, \thinspace v,} $ etc., designan \textit{n-tuplos} de variables
            individuales, es decir, filas de signos del tipo $\text{xyz}, \thinspace \text{x}_2\text{x}_1\text{x}_2\text{x}_3,$ ..., 
            donde la misma variable puede aparecer varias veces en la misma sentencia.
            \begin{itemize}
                \item[Nota:] Del modo correspondiente hay que entender sentencias como $\forall \mathfrak{x} \thinspace \exists \mathfrak{y},$ etc. 
                            Si una misma variable aparece varias veces en $\mathfrak{x}$, hay que pensar, naturamente, que sólo está escrita una vez
                            en $\forall \mathfrak{x} \thinspace \exists \mathfrak{y},$ ...
            \end{itemize}
\end{enumerate}

Consideremos a continuación una serie de lemas, para las que no presentamos demostración por no ser relevantes para el desarrollo del resto de resultados 
(además de ser demostraciones relativamente sencillas de realizar).

\begin{lema} \label{lem:lema-1}
    Para cada n-tuplo $\mathfrak{x}$ los siguientes resultados son deducibles:
    \begin{enumerate}
        \item[$a)$] $ \forall \mathfrak{x} \text{F} \mathfrak{x} \rightarrow \exists \mathfrak{x} \text{F} \mathfrak{x}  $
        \item[$b)$] $ \forall \mathfrak{x} \text{F} \mathfrak{x} \wedge \exists \mathfrak{x} \text{G} \mathfrak{x} \rightarrow \exists \mathfrak{x} 
                    (\text{F} \mathfrak{x} \wedge \text{G} \mathfrak{x}) $
        \item[$c)$] $ \forall \mathfrak{x} \lnot \text{F} \mathfrak{x} \leftrightarrow \exists \mathfrak{x} \text{F} \mathfrak{x} $
    \end{enumerate}
\end{lema}

\begin{lema} \label{lem:lema-2}
    Si $\mathfrak{x}$ y $\mathfrak{x}'$ sólo se diferencian por el orden en que están escritas las variables, entonces el siguente enunciado es deducible:
    $$\exists \mathfrak{x} \text{F} \mathfrak{x} \rightarrow \exists \mathfrak{x}' \text{F} \mathfrak{x}'$$ 
\end{lema}

\begin{lema} \label{lem:lema-3}
    Si todas las variables $\mathfrak{x}$ son distintas entre sí y $\mathfrak{x}'$ tiene el mismo número de miembros que $\mathfrak{x}$, entonces el siguiente enunciado
    es deducible:
    $$\forall \mathfrak{x} \text{F} \mathfrak{x} \rightarrow \forall \mathfrak{x}' \text{F} \mathfrak{x}'$$
\end{lema}

\begin{lema} \label{lem:lema-4}
    Si $\pi_i$ designa uno de los prefijos $\forall x_i$, $\exists x_i$; y $\rho_i$ designa uno de los prefijos $\forall y_i$, $\exists y_i$, 
    entonces el siguiente es deducible:\footnote{Un resultado análogo vale para $\vee$ en vez de para $\wedge$.}
    $$ \pi_1 \pi_2 \cdots \pi_n Fx_1 x_2 \cdots x_n \wedge \rho_1 \rho_2 \cdots \rho_m G y_1 y_2 \cdots y_m \leftrightarrow \pi(Fx_1 x_2 \cdots x_n \wedge Gy_1 y_2 \cdots y_m ) $$
    para cada prefijo $\pi$ que se componga de los $\pi_i$ y $\rho_i$ y que satisfaga la condición de que $\pi_i$ esté delante de $\pi_k$ 
    para $i < k \leq n$ y de que $\rho_i$ esté delante de $\rho_k$ para $i < k \leq m$ 
\end{lema}

\begin{lema} \label{lem:lema-5}
    Toda fórmula puede ponerse en forma normal prenexa, es decir, para cada fórmula $\alpha$ hay una fórmula prenexa $\gamma$, tal que $\alpha \leftrightarrow \gamma$ es deducible.
\end{lema}

\begin{lema} \label{lem:lema-6}
    Si $\alpha \leftrightarrow \beta$ es deducible, entonces también lo es $\varphi(\alpha) \leftrightarrow \varphi(\beta)$, 
    donde $\varphi(\alpha)$ designa una fórmula cualquiera que contenga $\alpha$ como parte.\footnote{Estos dos últimos resultados se pueden estudiar 
    en detalle en la tercera sección de \cite{hilbert1962elementos}.}  
\end{lema}

\begin{lema} \label{lem:lema-7}
    Cada fórmula conectiva válida es deducible, es decir, los axiomas 1-4 constituyen un sistema suficiente de axiomas para el cálculo conectivo.
\end{lema}

Con estos resultados previos ya estamos en condiciones de afrontar el problema que nos concierne.


%%%%%%%%%%%%%%%%%%%%%%%%%%%%%%%%%%%%%
\section{Exposición y demostraciones}

El teorema de completitud semántica de la lógica de primer orden aparece en el artículo de Gödel de 1930 como Teorema I:

\begin{teorema}\label{thm:teoremaI}
    Cada fórmula válida de la lógica de primer orden es deducible.
\end{teorema}

El presente teorema, objeto principal de estudio de esta sección, sería trivialmente demostrable si pudiesemos probar
el siguiente:
\begin{teorema}\label{thm:teoremaII}
    Cada fórmula de la lógica de primer orden es o refutable
    \footnote{<<$\varphi$ es refutable>> significa <<$\lnot \varphi$ es deducible>>.} o satisfacible (sobre un universo infinito numerable).
\end{teorema}

Y por ello surge el siguiente resultado: 

\begin{proposicion}
    \autoref{thm:teoremaII} $\Rightarrow$ \autoref{thm:teoremaI}
\end{proposicion}
\begin{proof}
    Sea $\alpha$ una fórmula válida. Siendo esto así, $\lnot \alpha$ no es satisfacible, y aplicando \autoref{thm:teoremaII} 
    tenemos que $\alpha$ es refutable. Por tanto, con ello se tiene que $\lnot \lnot \alpha$ (y como consecuencia, también $\alpha$) es una fórmula
    deducible.\footnote{El recíproco del anterior resultado también es cierto y con una demostración igual de simple, aunque no la demostraremos por no ser relevante
    en la demostración del \autoref{thm:teoremaI}.}
\end{proof}

\begin{definicion}
    Una \textit{K-fórmula} es una fórmula $\kappa$ perteneciente a una clase del conjunto de fórmulas $K$ cumpliendo las siguientes condiciones:
    \begin{enumerate}
        \item $\kappa$ es una fórmula prenexa.
        \item $\kappa$ carece de variables individuales libres.
        \item El prefijo de $\kappa$ comienza con un cuantificador universal y termina con un cuantificador particular.
    \end{enumerate}
\end{definicion}

Entonces con la presente definición podemos deducir el siguiente resultado:

\begin{teorema} \label{thm:teoremaIII}
    Si cada $K$-fórmula es refutable o satisfacible, también lo es cualquier fórmula.
\end{teorema}

\begin{proof}
    Sea $\alpha$ una fórmula que no pertenece a $K$. Sea $\mathfrak{x}$ el conjunto de sus variables libres. Como se puede ver
    directamente, si $\alpha$ es refutable (o satisfacible), se sigue la refutabilidad (o satisfacibilidad) de 
    $\exists \mathfrak{x} \alpha$, e igualmente a la inversa. 

    Sea ahora $\pi \varphi$ la forma normal prenexa de $\exists \mathfrak{x} \alpha$, de tal modo que
    \begin{equation}\label{eq:III-1}
        \exists \mathfrak{x} \alpha \leftrightarrow \pi \varphi
    \end{equation}
    es deducible. Además estipulemos que
    \begin{equation}
        \beta = \forall x \pi \exists y (\varphi \wedge \text{Fx} \vee \lnot \text{Fy})
        \footnote{Las variables x e y no deben aparecer en $\pi$.}
    \end{equation}
    
    Entonces
    \begin{equation}\label{eq:III-2}
        \pi \varphi \leftrightarrow \beta
    \end{equation}
    es deducible (por el \autoref{lem:lema-4} y por la deducibilidad de $\forall x \pi \exists y (\varphi \wedge \text{Fx} \vee \lnot \text{Fy})$).
    $\beta$ pertenece a $K$ y, por tanto, es o satisfacible o refutable. Pero por \eqref{eq:III-1} y \eqref{eq:III-2}
    la satisfacibilidad de $\beta$ implica la de $\exists \mathfrak{x} \alpha$ y consiguientemente también la de $\alpha$,
    y lo mismo se puede aplicar para la refutabilidad. Por tanto, concluimos que también $\alpha$ es o satisfacible o refutable.
\end{proof}

Teniendo en cuenta el \autoref{thm:teoremaIII} anterior, basta para demostrar el \autoref{thm:teoremaII} con probar la siguiente:

\begin{proposicion} \label{prop:satisf-refut}
    Cada $K$-fórmula es satisfacible o refutable.
\end{proposicion}

Para ello definimos previamente el concepto de grado de una $K$-fórmula, y probaremos algunos resultados para poder probar la anterior
proposición.

\begin{definicion}
    Llamaremos grado de una $K$-fórmula
    \footnote{También, en el mismo sentido se le puede llamar "grado de un prefijo".}
    al número de series de cuantificaciones universales de su prefijo, separadas unas de otras
    por cuantificadores existenciales.
\end{definicion}

Probamos primeramente el siguiente resultado.

\begin{teorema}\label{thm:teoremaIV}
    Si cada $K$-fórmula de grado $n$ es o refutable o satisfacible, entonces también lo es cada $K$-fórmula de grado $n+1$.
\end{teorema}

\begin{proof}
    Sea $\pi_1 \alpha$ una $K$-fórmula de grado $n+1$. Sea $\pi_1 = \forall \mathfrak{x} \exists \mathfrak{y} \pi_2$ y 
    $\pi_2 = \forall \mathfrak{u} \exists \mathfrak{v} \pi_3$, donde $\pi_2$ tiene grado $n$ y $\pi_3$ el grado $n-1$. Sea además 
    $\text{F}$ una variable predicativa que no aparezca en $\alpha$. Establezcamos:
    \begin{equation}
        \beta = \forall \mathfrak{x}' \exists \mathfrak{y}' \text{F}  \mathfrak{x}' \mathfrak{y}' \wedge 
        \forall \mathfrak{x} \forall \mathfrak{y} (\text{F}  \mathfrak{x} \mathfrak{y} \rightarrow \pi_2 \alpha)
    \end{equation}
    y
    \begin{equation}
        \gamma = \forall \mathfrak{x}' \forall \mathfrak{x} \forall \mathfrak{y} \forall \mathfrak{u} \exists \mathfrak{y}' \exists \mathfrak{v} 
        \pi_3 (\text{F} \mathfrak{x}' \mathfrak{y}' \wedge (\text{F} \mathfrak{x} \mathfrak{y}) \rightarrow \alpha) \footnote{En esta sentencia
        suponemos que las sucesiones de variables $\mathfrak{x}, \mathfrak{x}', \mathfrak{y}, \mathfrak{y}', \mathfrak{u}, \mathfrak{v}$ son disjuntas
        entre sí.} % esta llave sale en rojo y no se por que, pero no da error
    \end{equation}

    Aplicando ahora dos veces el \autoref{lem:lema-4} junto con el \autoref{lem:lema-6}, obtenemos la deducibilidad de 
    \begin{equation} \label{eq:IV-1}
        \beta \leftrightarrow \gamma
    \end{equation}

    Además, es claro que la fórmula 
    \begin{equation} \label{eq:IV-2}
        \beta \rightarrow \pi_1 \alpha
    \end{equation}
    es válida. Ahora bien, $\gamma$ tiene grado $n$, y por tanto es por hipótesis o satisfacible o refutable. Si $\gamma$ es satisfacible, entonces
    también lo es $\pi_1 \alpha$ (por \eqref{eq:IV-1} y \eqref{eq:IV-2}). Si en caso contrario $\gamma$ es refutable, entonces también lo es 
    $\beta$ (por \eqref{eq:IV-1}), es decir, entonces $\lnot \beta$ es deducible. Sustituyendo ahora F por $\pi_2 \alpha$ en $\lnot \beta$, obtenemos
    que en este caso la siguiente sentencia es deducible:
    \begin{equation}
        \lnot (\forall \mathfrak{x}' \exists \mathfrak{y}' \pi_2 \alpha \wedge 
        \forall \mathfrak{x} \forall \mathfrak{y} (\pi_2 \alpha \rightarrow \pi_2 \alpha))
    \end{equation}

    Se puede observar que, naturalmente, 
    \begin{equation}
        \forall \mathfrak{x} \forall \mathfrak{y} (\pi_2 \alpha \rightarrow \pi_2 \alpha)
    \end{equation}
    es deducible, y por ello también lo es $\lnot \forall \mathfrak{x}' \exists \mathfrak{y}' \pi_2 \alpha$, es decir, en este caso $\pi_2 \alpha$ es refutable.
    Por tanto, de hecho $\pi_2 \alpha$ es o refutable o satisfacible.
\end{proof}

Ahora, para acabar de probar la \autoref{prop:satisf-refut}, sólo necesitamos probar el siguiente resultado:

\begin{teorema} \label{thm:teoremaV}
    Cada $K$-fórmula de primer grado es o satisfacible o refutable.
\end{teorema}

La demostración de este teorema requiere algunas definiciones previas, así como algunos resultados derivados de las definiciones que vamos a establecer, por lo que
se deja la presente demostración para más adelante.

Sea $\forall \mathfrak{x} \exists \mathfrak{y} \alpha (\mathfrak{x} ; \mathfrak{y} )$ $-$abreviado como $\pi\alpha$$-$ una fórmula cualquiera de primer grado.
Mediante $\mathfrak{x}$ representamos un r-tuplo de variables, con $\mathfrak{y}$ un s-tuplo. COnsideremos los r-tuplos sacados de la sucesión $\text{x}_0, \text{x}_1,
\text{x}_2,\dots \text{x}_i \dots$ ordenados por la suma creciente de sus índices en una sucesión:
\begin{equation}
    \mathfrak{x}_1 = (\text{x}_0, \text{x}_0, \dots, \text{x}_0), \quad \mathfrak{x}_2 = (\text{x}_1, \text{x}_0, \dots, \text{x}_0), \quad 
    \mathfrak{x}_3 = (\text{x}_0, \text{x}_1, \text{x}_0, \dots, \text{x}_0) \quad \dots
\end{equation}
y definamos una sucesión $\{\alpha_n\}$ de fórmulas derivadas a partir de $\pi\alpha$ del siguiente modo:
\begin{equation}
    \begin{aligned}
        &\alpha_1 = \alpha(\mathfrak{x}_1 ; \text{x}_1, \text{x}_2, \dots, \text{x}_s) \\
        &\alpha_2 = \alpha(\mathfrak{x}_2 ; \text{x}_{s+1}, \text{x}_{s+2}, \dots, \text{x}_{2s}) \wedge \alpha_1 \\
        &\cdots \\ 
        &\alpha_n = \alpha(\mathfrak{x}_n ; \text{x}_{(n-1)s+1}, \text{x}_{(n-1)s+2}, \dots, \text{x}_{ns}) \wedge \alpha_{n-1}    
    \end{aligned}
\end{equation}

Designemos mediante $\mathfrak{y}_n$ el s-tuplo $\text{x}_{(n-1)s+1}, \text{x}_{(n-1)s+2}, \dots, \text{x}_{ns}$ de tal modo que:
\begin{equation} \label{eq:def-V}
    \alpha_n = \alpha(\mathfrak{x}_n ; \mathfrak{y}_n) \wedge \alpha_{n-1}
\end{equation}

Además, definimos $\pi_n\alpha_n$ estableciendo como sigue:
\begin{equation}
    \pi_n\alpha_n = \exists \text{x}_0 \exists \text{x}_1 \cdots \exists \text{x}_{ns} \alpha_n
\end{equation}

Como fácilmente se comprueba, en $\alpha_n$ aparecen precisamente las variables desde $\text{x}_0$ hasta $\text{x}_{ns}$, que están también ligadas por el prefijo $\pi_n$. 
Además, es evidente que las variables del r-tuplo $\mathfrak{x}_{n+1}$ ya aparecen en $\pi_n$ (y por tanto son distintas de las que aparecen en $\mathfrak{y}_{n+1}$).
Designemos mediante $\pi_n'$ lo que queda de $\pi_n$ cuando suprimimos las variables de r-tuplo $\mathfrak{x}_{n+1}$. Si nos olvidamos del orden de aparición de las variables,
tenemos que $\exists \mathfrak{x}_{n+1} \pi_n' = \pi_n$.

Supuestas todas estas notaciones previas, tenemos como resultado directo el teorema siguiente:

\begin{teorema} \label{thm:teoremaVI}
    Para cada $n$ es deducible $\pi\alpha \rightarrow \pi_n\alpha_n$.
\end{teorema}
\begin{proof}
    Probaremos el teorema mediante inducción.
    \begin{itemize}
        \item Para $n = 1$ tenemos que $\pi\alpha \rightarrow \pi_1\alpha_1$ es deducible, ya que, por el \autoref{lem:lema-3} y la cuarta regla de inferencia, tenemos:
                \begin{equation}
                    \forall \mathfrak{x} \exists \mathfrak{y} \alpha (\mathfrak{x} ; \mathfrak{y} ) \rightarrow 
                    \forall \mathfrak{x}_1 \exists \mathfrak{y}_1 \alpha(\mathfrak{x}_1 ; \mathfrak{y}_1 )
                \end{equation}
                y además, por el \autoref{lem:lema-1} tenemos: 
                \begin{equation}
                    \forall \mathfrak{x}_1 \exists \mathfrak{y}_1 \alpha(\mathfrak{x}_1 ; \mathfrak{y}_1 ) \rightarrow  
                    \exists \mathfrak{x}_1 \exists \mathfrak{y}_1 \alpha(\mathfrak{x}_1 ; \mathfrak{y}_1 )
                \end{equation}
        \item Para un $n$ arbitrario tenemos que $\pi\alpha \wedge \pi_n\alpha_n \rightarrow \pi_{n+1}\alpha_{n+1}$ es deducible ya que, al igual que antes, 
                aplicando el \autoref{lem:lema-3} y la cuarta regla de inferencia, tenemos:
                \begin{equation} \label{eq:VI-1}
                    \forall \mathfrak{x} \exists \mathfrak{y} \alpha (\mathfrak{x} ; \mathfrak{y} ) \rightarrow 
                    \forall \mathfrak{x}_{n+1} \exists \mathfrak{y}_{n+1} \alpha(\mathfrak{x}_{n+1} ; \mathfrak{y}_{n+1} )
                \end{equation}
                y además, por el \autoref{lem:lema-2} tenemos:
                \begin{equation} \label{eq:VI-2}
                    \pi_n\alpha_n \rightarrow \exists \mathfrak{x}_{n+1} \pi_n'\alpha_n
                \end{equation}
                Ahora, aplicamos el \autoref{lem:lema-1} y sustituimos F por $\exists \mathfrak{x}_{n+1} \alpha(\mathfrak{x}_{n+1} ; \mathfrak{y}_{n+1} )$
                y G por $\pi'\alpha_{n}$, con lo que obtenemos:
                \begin{equation} \label{eq:VI-3}
                    \forall \mathfrak{x}_{n+1} \exists \mathfrak{y}_{n+1} \alpha (\mathfrak{x}_{n+1} ; \mathfrak{y}_{n+1} ) \wedge \exists \mathfrak{x}_{n+1} \pi_n'\alpha_n
                    \rightarrow  \exists \mathfrak{x}_{n+1} (\exists \mathfrak{y}_{n+1} \alpha (\mathfrak{x}_{n+1} ; \mathfrak{y}_{n+1} ) \wedge \pi_n'\alpha_n)       
                \end{equation} 

                Fijándonos ahora en que el antecedente del condicional \eqref{eq:VI-3} es la conyunción de los consiguientes de \eqref{eq:VI-1} y \eqref{eq:VI-2},
                obtenemos que es deducible:
                \begin{equation} \label{eq:VI-4}
                    \pi\alpha \wedge \pi_n\alpha_n \rightarrow \exists \mathfrak{x}_{n+1} (\exists \mathfrak{y}_{n+1} 
                    \alpha (\mathfrak{x}_{n+1} ; \mathfrak{y}_{n+1} ) \wedge \pi_n'\alpha_n)
                \end{equation}
                Por otro lado, de \eqref{eq:VI-1} y de los lemas 2, 4 y 6 obtenemos la deducibilidad de:
                \begin{equation} \label{eq:VI-5}
                    \exists \mathfrak{x}_{n+1} (\exists \mathfrak{y}_{n+1} \alpha (\mathfrak{x}_{n+1} ; \mathfrak{y}_{n+1} ) \wedge \pi_n'\alpha_n)
                    \leftrightarrow \pi_{n+1}\alpha_{n+1}
                \end{equation}
                Y gracias a \eqref{eq:VI-4} y \eqref{eq:VI-5} obtenemos la inducción, y por tanto la prueba del teorema.
    \end{itemize}

    Con esto tenemos probado el \autoref{thm:teoremaVI}, que como consecuencia queda probado el \autoref{thm:teoremaV}, que junto con resultados anteriores hemos conseguido 
    probar la \autoref{prop:satisf-refut}. Como vimos anteriormente, esta proposición era el resultado que nos faltaba para acabar la demostración del \autoref{thm:teoremaII}, 
    con lo que hemos probado la tesis de este apartado. Es decir, hemos dado una demostración de que toda fórmula válida de primer orden es deducible.
\end{proof}


\section{Corolarios}




%%%%%%%%%%%%%%%%%%%%%%%%%%%%%%%%%%%%%
\chapter{Sobre sentencias formalmente indecidibles}
\dictum[John von Neumann]{El logro de Gödel en la lógica moderna es singular y monumental - más que monumental, es una señal que 
permanecerá visible lejos en el espacio y en el tiempo.\footnote{Palabras de von Neumann en la entrega a Gödel del premio Einstein 
en 1951, recogidas en \textit{New York Times}, 15 de marzo de 1951, pág. 31.}}


\section{Definiciones y conceptos previos}
Tendremos como objetivo principal probar la existencia de sentencias indecidibles para un sistema formal $P$. Dicho sistema $P$ es
esencialmente el sistema que se obtiene cuando a los axiomas de peano se les añade la lógica de \textit{Principia Mathematica} ($PM$ de aquí en adelante).
\footnote{El hecho de de los axiomas de Peano, así como todas las otras modificaciones del sistema $PM$ introducidas en toda la demostración, sólo 
tienen como finalidad simplificar la prueba, y por ellos son prescindibles.}

Los signos primitivos del sistema $P$ son los siguientes:
\begin{enumerate}
    \item Constantes: « $\sim$ » (no), « $\vee$ » (o), « $\Pi$ » (para todo), « $0$ »(cero), « s » (el siguiente de), «()» (paréntesis).
    \item Variables tipo 1 (para individuos\footnote{Cuando tratamos de individuos hacemos referencia al conjunto de los números naturales.}, 
            incluyendo el 0):  «$\text{x}_1$», «$\text{y}_1$», «$\text{z}_1$», \dots \\
          Variables tipo 2 (para clases de individuos): «$\text{x}_2$», «$\text{y}_2$», «$\text{z}_2$», \dots \\
          Variables tipo 3 (para clases de clases de individuos): «$\text{x}_3$», «$\text{y}_3$», «$\text{z}_3$», \dots \\
          \dots \\
          etc., para cada número natural como tipo.\footnote{Suponemos que disponemos de una cantidad infinita numerable de signos para cada tipo de variables.}
\end{enumerate}
\begin{itemize}
    \item[Observación:] No necesitamos disponer de varaibles para relaciones binarias o $n$-arias ($n>2$) como signos primitivos, ya que podemos definir
                        las relaciones como clase de pares ordenados y los pares ordenados a su vez como clases de clases. Por ejemplo, podemos considerar
                        el par ordenado $<a,b>$ como $\{\{a\}, \{a.b\}\}$, donde $\{x,y\}$ denota la clase cuyos únicos elementos son $x$ e $y$, y $\{x\}$ 
                        la clase cuyo único elemento es $x$.\footnote{Las relaciones no homogéneas también pueden definirse de esta manera; por ejemplo, una
                        relación entre individuos y clases puede definirse como una clase de elementos de la forma $\{\{x_2\}, \{\{x_1\},x_2\}\}$. Todos los 
                        teoremas deducibles en $PM$ son también deducibles cuando se los reformula de esta manera.}
\end{itemize}
                           
Llamaremos \textit{signo de primer tipo} a una combinación de signos que tenga una de las siguientes formas:
\begin{equation}
    a,\thinspace sa,\thinspace ssa,\thinspace sssa,\thinspace \dots,\thinspace \text{etc.,}
\end{equation}
Donde $a$ es $0$ ó es variable de tipo 1. En el primer caso llamamos a tal signo un numeral. Para $n>1$ entendemos por \textit{signo de tipo n} lo mismo que por 
\textit{variable de tipo n}. Llamaremos \textit{fórmulas elementales} a las combianciones de signos de la forma $a(b)$, donde $b$ es un signo de tipo $n$, y $a$ 
es un signo de tipo $n+1$. Definimos la calse de las \textit{fórmulas} como la mínima clase que abarca todas las fórmulas elementales y que, siempre que contenga 
$\alpha$ y $\beta$, contiene también $\sim(\alpha), (\alpha)\vee (\beta)$ y $\Pi\text{x}(\alpha)$ (donde x es una variable cualquiera)
\footnote{Por tanto, $\Pi\text{x}(\alpha)$ es también una fórmula cuando x no aparece o no está libre en $\alpha$. Naturalmente, en ester caso $\Pi\text{x}(\alpha)$
significaría lo mismo que $\alpha$.}. Llamamos a $(\alpha)\vee (\beta)$ la \textit{disyunción} de $\alpha$ y $\beta$, a $\sim(\alpha)$ la \textit{negación} de $\alpha$
y a $\Pi\text{x}(\alpha)$ una \textit{generalización} de $\alpha$. Una \textit{sentencia} es una fórmula sin variables libres (donde la noción de variable libre se 
define del modo usual). A una fórmula con exactamente $n$ variables libres (y ninguna otra variable libre) la llamaremos \textit{signo relacional n-ario}; para $n=1$
lo llamaremos también \textit{signo de clase}.

Por $\daleth^b_v \alpha$ (donde $\alpha$ designa una fórmula, $v$ una variable y $b$ un signo del mismo tipo que $v$) entendemos la fórmula que resulta de reemplazar 
en $\alpha$ cada aparición libre de $v$ por $b$\footnote{Si $v$ no aparece libre en $\alpha$, entonces $\daleth^b_v \alpha = \alpha$. Nótese que $\daleth$ es un signo 
matemático.}. Decimos que una fórmula $\alpha$ es una \textit{elevación de tipo} de otra fórmula $\beta$ si $\alpha$ se obtiene a partir de $\beta$ mediante una elevación
por el mismo número de cada variable que aparece en $\beta$. \\

Las siguientes fórmulas (de I a V) se llaman \textit{axiomas} (están escritas con ayuda de abreviaturas: $\wedge, \supset, \equiv, \Sigma, =$\footnote{Como en $PM$, 
consideramos que $\text{x}_1 = \text{y}_1$ está definido por $\Pi \text{x}_2 (\text{x}_2(\text{x}_1) \supset \text{x}_2(\text{y}_1))$; de igual modo para los tipos 
superiores.}, definidas del modo usual y conforme a las conveciones habituales sobre la omisión de paréntesis)\footnote{Para obtener los axiomas a partir de los esquemas
indicados debemos (después de realizar las sustituciones permitidas en II, III y IV), además,
\begin{enumerate}
    \item[(1)] eliminar las abreviaturas
    \item[(2)] añadir los paréntesis omitidos.
\end{enumerate} 
Nótese que las expresiones así obtenidas deben ser "fórmulas" en el sentido arriba definido.}:

\begin{enumerate}
    \item[I.]   %\begin{equation}
                    %\begin{flalign}
                        $1. \quad \sim (\text{sx}_1 = 0)$ \\
                        $2. \quad \text{sx}_1 = \text{sy}_1 \supset \text{x}_1 = \text{y}_1$ \\
                        $3. \quad \text{x}_2 (0) \wedge \Pi \text{x}_1 \thinspace (\text{x}_2 \thinspace(\text{x}_1) \wedge \text{x}_2 (\text{sx}_1)) \supset \Pi \text{x}_1 \thinspace (\text{x}_2 \thinspace(\text{x}_1)) $   
                    %\end{flalign}
                %\end{equation}
    \item[II.] Cada fórmula que resulta de sustituir X, Y por cualesquiera fórmulas en los siguientes esquemas:
                \begin{flalign}
                    &1.\quad \text{X} \vee \text{X} \supset \text{X} \\
                    &2.\quad \text{X} \supset \text{X} \vee \text{Y} \\
                    &3.\quad \text{X} \vee \text{Y} \supset \text{Y} \vee \text{X} \\
                    &4.\quad (\text{X} \supset \text{Y}) \supset (\text{Z} \vee \text{X} \supset \text{Z} \vee \text{Y})
                \end{flalign}
    \item[III.] Cada fórmula que resulta de uno de estos dos esquemas:
                \begin{flalign}
                    &1.\quad \Pi v\alpha \supset \daleth_v^c \alpha \\ 
                    &2.\quad \Pi v(\beta \vee \alpha) \supset \beta \vee \Pi v(\alpha)
                \end{flalign}
                Cuando sustituimos $\alpha, v, \beta, c$ del siguiente modo (y realizamos la operación indicada por «$\daleth$» en $1.$):
                \begin{enumerate}
                    \item[] Sustituimos por $\alpha$ por una fórmula cualquiera, $v$ por una variable cualquiera, $\beta$ por una fórmula en la que no aparezca libre
                            $v$ y $c$ por un signo del mismo tipo que $v$, siempre que $c$ no contenga alguna variable que pase a estar ligada en un lugar de $\alpha$ 
                            donde $v$ estaba libre.\footnote{Por tanto, $c$ es o una variable o el $0$ o un signo de la forma $s\dots su$, donde $u$ es $0$ o una variable
                            de tipo 1. Respecto de la noción de estar (una variable) libre o ligada en un lugar de $\alpha$, véase \cite{v1927hilbertschen}.}
                \end{enumerate}
    \item[IV.]  Cada fórmula que resulta del esquema $$ \Sigma u \Pi v (u(v) \equiv \alpha) $$ cuando usstituimos $v$ por una varaible cualquiera de tipo $n$, sustituimos
                $u$ por una variable cualquiera de tipo $n+1$ y sustituimos $\alpha$ por una fórmula, en la que $u$ no esté libre. Este axioma desempeña el papel de axioma
                de reducibilidad (el axioma de comprensión de la teoría de conjuntos).
    \item[V.]   Cada fórmula que resulta de $$\Pi \text{x}_1 (\text{x}_2 (\text{x}_1) \equiv \text{y}_2(\text{x}_1)) \supset \text{x}_2 = \text{y}_2$$ por elevación de tipo
                (así como esta fórmula misma). Este axioma dice que una clase está completamente determinada por sus elementos.
\end{enumerate}

Una fórmula $\gamma$ se llama una \textit{inferencia inmediata} de $\alpha$ y $\beta$, si $\alpha$ es la fórmula $\sim \beta \vee \gamma$ (y $\gamma$ se llama una 
\textit{inferencia inmediata} de $\alpha$, si $\gamma$ es la fórmula $\Pi v \alpha $, donde $v$ designa una variable cualquiera). La clase de las \textit{fórmulas deducibles}
se define como la mínima clase de fórmulas que contiene los axiomas y está clausurada respecto a la relación de «inferencia inmediata»\footnote{La regla de sustitución 
resulta aquí supreflua, pues en los axiomas mismo ya tenemos realizadas todas las sustituciones posibles (véase \cite{v1927hilbertschen})}.

Ahora asignamos unívocamente números naturas a los signos primitivos del sistema $P$ del siguiente modo:
\begin{equation}
    \begin{aligned}
        &\text{«}0\text{»} \thinspace \dots \thinspace 1 \\
        &\text{«}s\text{»} \thinspace \dots \thinspace 3 \\
        &\text{«}\sim\text{»} \thinspace \dots \thinspace 5 \\
        &\text{«}\vee\text{»} \thinspace \dots \thinspace 7 \\
        &\text{«}\Pi\text{»} \thinspace \dots \thinspace 9 \\
        &\text{«}(\text{»} \thinspace \dots \thinspace 11 \\
        &\text{«})\text{»} \thinspace \dots \thinspace 13 \\    
    \end{aligned}
\end{equation}

A las variables de tipo $n$ asignamos los números de la forma $\rho^n$ (donde $\rho$ es un número primo $> 13$). Mediante esta asignación a cada fila finita de 
signos primitivos (y en especial a cada fórmula) corresponde biunívocamente una secuencia finita de números naturales. Ahora asignamos (de nuevo biunívocamente)
números naturales a las secuencias finitas de números naturales, haciendo corresponder a la secuencia $n_1, n_2, \dots, n_k$ el número 
$2^{n_1} \cdot 3^{n_2} \cdots \rho^{n_k}_k$ donde $\rho_k$ denota el $k$-avo número primo (en orden de magnitud creciente). Así asignamos biunívocamente un
número natural no sólo a cada signo primitivo, sino también a cada secuencia finita de signos primitivos. Mediante $nu(\text{a})$ denotamos
el número natural asignado al signo primitivo (o a la secuencia de signos primitivos) a. Supongamos que esté dada cierta clase o relación $n$-aria $R$ entre signos
primitivos. Le asignamos la clase o relación $n$-aria $R'$ entre números naturales, en la que están los números $x_1, x_2, \dots, x_n$ si y sólo si hay signos primitivos
o secuencias de signos primitivos $\text{a}_1, \text{a}_2, \dots, \text{a}_n$, tales que $x_i = nu(\text{a}_i)$ (para $i = 1, 2, \dots, n)$ y 
$\text{a}_1, \text{a}_2, \dots, \text{a}_n$ están en la relación $R$. Las clases y relaciones de números naturales, que corresponden de este modo a los conceptos
metamatemáticos hasta ahora definidos, como por ejemplo «variable», «fórmula», «sentencia», «axioma», «fórmula deducible», etc., serán designadas por las mismas 
palabras escritas con letras mayúsculas pequeñas. Por ejemplo, el enunciado de que en el sistema $P$ hay problemas indecidibles se convierte en la siguiente afirmación:
Hay SENTENCIAS $a$, tales que ni $a$ ni la NEGACIÓN de $a$ son FÓRMULAS DEDUCIBLES.\\

Ahora vamos a insertar aquí una disgresión que por el momento no tiene nada que ver con el sistema formal $P$. Empecemos por dar la siguiente definición: 
Decimos que una función numérica\footnote{Es decir, su dominio definicional es la clase de los números naturales (o de los $n$-tuplos de números naturales) 
y sus valores son números naturales.} $f(x_1, x_2, \dots, x_n)$ está \textit{recursivamente definida a partir} de las funciones numéricas $h(x_1, x_2, \dots, x_{n-1})$
y $g(x_1, x_2, \dots, x_{n+1})$, si para cada $x_2, \dots, x_n$, $k$ vale lo siguiente:\footnote{En lo sucesivo las letras latinas minúsculas (a veces con subíndices) 
son siempre variables para números naturales (a no ser que se indique explícitamente lo contrario).}
\begin{equation} \label{eq:recdef}
    \begin{aligned}
        &f(0, x_2, \dots, x_n) = h(x_2, \dots, x_n) \\
        &f(k+1, x_2, \dots, x_n) = g(k, f(k, x_2, \dots, x_n), x_2, \dots, x_n) 
    \end{aligned}
\end{equation}

Una función numérica $f$ se llama \textit{recursiva primitiva} si hay una secuencia finita de funciones $f_1, f_2, \dots, f_n$, que acaba con $f$ y que tiene 
la propiedad de que cada función $f_k$ de la secuencia está recursivamente definida a partir de dos de las funciones precedentes o resulta de alguna de las 
funciones precedentes por sustitución\footnote{Más precisamente: por introducción de algunas de las funciones precedentes en los lugares argumentales de una 
de las funciones precedentes, por ejemplo, $f_k(x_1, x_2) = f_p(f_q(x_1, x_2), f_r(x_2))$, con $p, q, r < k$. No es necesario que todas las variables
del lado izquierdo aparezcan también en el derecho.}, o, finalmente, es una constante o la función del siguiente, $x+1$. La longitud de la mínima secuencia de $f_i$
corresponde a una función recursiva primitiva $f$ se llama su \textit{grado}. Una relación $n$-ara $R$ entre números naturales se llama recursiva 
primitiva\footnote{Incluimos las clases entre las relaciones (como relaciones monarias). Las relaciones recursivas primitivas tienen desde luego la propiedad 
de que para cada $n$-tuplo dado de números naturales $x_1, x_2, \dots, x_n$ \textit{se puede decidir si} $R$ $x_1, x_2, \dots, x_n$ o no.} si hay una función $n$-aria
recursiva primitiva $f$ tal que para cada $n$ números naturales $x_1, x_2, \dots, x_n$ 
\begin{equation}
    R x_1, x_2, \dots, x_n \leftrightarrow f(x_1, x_2, \dots, x_n) = 0 
    \footnote{Para todas las consideraciones intuitivas (y especialmente para las metamatemáticas) usamos el simbolismo de Hilbert definido en \cite{hilbert1962elementos}.}
\end{equation}

Los siguientes teoremas valen:

\begin{teorema} \label{teo:TeoremaI}
    Cada función (o relación) obtenida a partir de funciones (o relaciones) recursivas primitivas por sustitución de las variables por funciones recursivas 
    primitivas es recursiva primitiva; igualmente lo es cada función obtenida a partir de funciones recursivas primitivas por definición recursiva según el
    esquema de \eqref{eq:recdef}.
\end{teorema}

\begin{teorema} \label{teo:TeoremaII}
    Si $R$ y $S$ son relaciones recursivas primitivas, también lo son $\lnot R$ y $R \vee S$ (por tanto, también $R \wedge S$).
\end{teorema}

\begin{teorema} \label{teo:TeoremaIII}
    Si las funciones $f(\mathfrak{x})$, $h(\mathfrak{y})$ son recursivas primitivas, entonces también lo es la relación 
    $f(\mathfrak{x}) = h(\mathfrak{y})$\footnote{Usamos las letras alemanas $\mathfrak{x}, \mathfrak{y}$ como abreviaturas para cualesquiera $n$-tuplos de variables, 
    como por ejemplo $x_1, x_2, \dots, x_n$.}.
\end{teorema}

\begin{teorema} \label{teo:TeoremaIV}
    Si la función $f(\mathfrak{x})$ y la relación $Rx, \mathfrak{y}$ son recursivas primitivas, también lo son las relaciones $S$ y $T$ definidas por 
    \begin{equation}
        \begin{aligned}
            &S(\mathfrak{x}, \mathfrak{y}) \leftrightarrow \exists x (x \leq f(\mathfrak{x}) \wedge Rx, \mathfrak{y} ) \\
            &T(\mathfrak{x}, \mathfrak{y}) \leftrightarrow \forall x (x \leq h(\mathfrak{x}) \wedge Rx, \mathfrak{y} )
        \end{aligned}
    \end{equation}
    así como la función
    \begin{equation}
        q(\mathfrak{x}, \mathfrak{y}) = \mu x (x \leq f(\mathfrak{x}) \wedge Rx, \mathfrak{y}),
    \end{equation}
    donde $\mu x \varphi(x)$ significa el mínimo número $x$, para el que vale $\varphi(x)$, si hay algún tal, y $0$, si no lo hay.
\end{teorema}

El \autoref{teo:TeoremaI} se sigue inmediatamente de la definición de «recursivo primitivo». Tanto el \autoref{teo:TeoremaII} como el \autoref{teo:TeoremaIII}
se basan en que las funciones numéricas
\begin{equation}
    ne(x), \\
    di(x,y), \\
    id(y,x)
\end{equation}
correspondientes a las nociones lógicas $\lnot, \vee, =$, a saber
\begin{flalign}
    &ne(x) = 1; \quad ne(x) = 0 \text{ para } x \neq 0 \\
    &di(0,x) = di(x,0) = 0; \quad di(x,y) = 1 \text{ si } x \neq 0, y \neq 0 \\
    &id(x,y) = 0 \text{ si } x=y; \quad id(x,y) = 1 \text{ si } x \neq y 
\end{flalign}

son recursivas primitivas, como fácilmente se comprueba. He aquí la prueba resumida del \autoref{teo:TeoremaIV}:
\begin{proof}
    Por hipótesis hay una función recursiva primitiva $r(x, \mathfrak{y})$, tal que
    \begin{equation}
        R x, \mathfrak{y} \leftrightarrow r(x, \mathfrak{y}) = 0.
    \end{equation}

    Definamos ahroa mediante el esquema de recursión \eqref{eq:recdef} una función $j(x, \mathfrak{y})$ del siguiente modo:
    \begin{equation}
        \begin{aligned}
            &j(0, \mathfrak{y}) = 0 \\
            &j(n+1, \mathfrak{y}) = (n+1) \cdot a + j(n, \mathfrak{y}) \cdot ne(a) 
            \footnote{Presuponemos como ya es sabido que las funciones $x+y$ (adición) y $x\cdot y$ (multiplicación) son recursivas primitivas.}
        \end{aligned}
    \end{equation}
    donde $a = ne(ne(r(0, \mathfrak{y}))) \cdot ne(r(n+1, \mathfrak{y})) \cdot ne(j(n, \mathfrak{y}))$.

    Por tanto, $j(n+1, \mathfrak{y})$ es igual a $n+1$ (si $a=1$) o es igual a $j(n, \mathfrak{y})$ (si $a=0$)\footnote{$a$ no puede tomar otros valores que
    $0$ y $1$, como se sigue en la definición de $ne$.}. Evidentemente, el primer caso ocurre si y sólo si todos los factores de $a$ son $1$, es decir, si ocurre que
    \begin{equation}
        \lnot R \thinspace 0, \mathfrak{y} \wedge R \thinspace n+1, \mathfrak{y} \wedge j(n, \mathfrak{y}) = 0
    \end{equation}

    De aquí se sigue que la función $j(n, \mathfrak{y})$, considerada como función de $n$, da siempre $0$ hasta (pero no incluyendo) el mínimo valor de $n$
    para el que ocurre $R \thinspace n, \mathfrak{y}$, y a partir de ahí siempre da ese valor. (Por consiguiente, si ya ocurre que $R \thinspace 0, \mathfrak{y},
    j(n, \mathfrak{y})$ es constante e igual a $0$). Por tanto, ocurre que
    \begin{flalign}
        &q(\mathfrak{x}, \mathfrak{y}) = j(f(\mathfrak{x}), \mathfrak{y}) \\
        &S \thinspace \mathfrak{x}, \mathfrak{y} \leftrightarrow R \thinspace q(\mathfrak{x}, \mathfrak{y}), \mathfrak{y} 
    \end{flalign}

    La relación $T$ puede ser reducida, por negación, a un caso análogo al de $S$. Con esto queda probado el \autoref{teo:TeoremaIV}.
\end{proof}

Las funciones $x+y, x\cdot y, x^v$, así como las relaciones $x<y$ y $x=y$ son recursivas primitivas, como fácilmente se comprueba. Partiendo de estos conceptos, 
vamos a definir u n a secuencia de funciones (o relaciones) $1$-$45$, cada una de las cuales se define a partir de las precedentes mediante los procedimientos indicados
en los cuatro teoremas anteriores. En la mayor parte de estas definiciones condensamos en un solo paso varios de los pasos permitidos por estos cuatro teoremas. 
Por tanto, cada una de las funciones (o relaciones) $1$-$45$, entre las que se encuentran, por ejemplo, los conceptos «FÓRMULA», «AXIOMA» e «INFERENCIA INMEDIATA»,
es recursiva primitiva.

\begin{enumerate}
    \item $x/y \leftrightarrow \exists z (z \leq x \wedge x = y \cdot z)$\footnote{Después del definiendum, el signo $=$ se usa en el sentido de «igualdad por
            definición»; el signo $\leftrightarrow$, en el de «equivalencia por definición» (por lo demás, el simbolismo es el de Hilbert).}\\
            $x$ es divisible por $y$.\footnote{Cada vez que en las definiciones siguientes aparece uno de los signos $\forall x, \exists x, \mu x$, éste está 
            seguido de una acotación de $x$. Esta acotación sirve meramente para asegurar que la noción definida es recursiva primitiva (véase \autoref{teo:TeoremaIV}).
            La extensión de la noción definida, por el contrario, no cambiaría en la mayor parte de los casos, aunque dejásemos de lado dicha acotación.}
    \item $\text{Prim } x \leftrightarrow \lnot \exists z (z \leq x \wedge z \neq 1 \wedge z \neq x \wedge x/z) \wedge x>1$\\ $x$ es un número primo.
    \item $0 \thinspace Pr \thinspace x = 0$\\ 
            $(n+1) \thinspace Pr \thinspace x = \mu y (y \leq x \wedge \text{Prim } y \wedge x/y \wedge y>n \thinspace Pr \thinspace x)$ \\
            $n \thinspace Pr \thinspace x$ es el $n$-avo número primo (por orden de magnitud creciente) contenido en $x$\footnote{Para $0 < n \leq z$, donde
            $z$ es el número de diferentes factores primos de $x$. Obsérvese que $n \thinspace Pr \thinspace x = 0$ para $n = z+1$.}.
    \item $0! = 1$ \\ $(n+1)! = (n+1) \cdot n!$
    \item $Pr(0) = 0$ \\ $Pr(n+1) = \mu y (y \leq (Pr(n))! + 1 \wedge \text{Prim } y \wedge y > Pr(n))$ \\ $Pr(n)$ es el $n$-avo número primo (por orden de 
            magnitud creciente)
    \item $n \thinspace Gl \thinspace x = \mu y (y \leq x \wedge x/(n \thinspace Pr \thinspace x)^y \wedge \lnot x/(n \thinspace Pr \thinspace x)^{y+1})$ \\ 
            $n \thinspace Gl \thinspace x$ es el miembro $n$-avo de la secuencia numérica correspondiente al número $x$ (para $n>0$ y $n$ no mayor que la 
            longitud de esa secuencia).
    \item $l(x) = \mu y (y \leq x \wedge y \thinspace Pr \thinspace x > 0 \wedge (y+1) \thinspace Pr \thinspace x = 0)$ \\ $l(x)$ es la longitud de la secuencia 
            numérica correspondiente a $x$.
    \item   $\begin{aligned}
                x * y = \mu z (&z \leq (Pr(l(x) + l(y)))^{x+y} \wedge \forall n (n \leq l(x) \rightarrow n \thinspace Gl \thinspace z = n \thinspace Gl \thinspace x) \wedge \\ 
                &\wedge \forall n (0 < n \leq l(y) \rightarrow (n + l(x)) \thinspace Gl \thinspace z = n \thinspace Gl \thinspace y))
            \end{aligned}$ \\
            $x*y$ corresponde a la operación de concatenación de dos secuencias finitas de números.
    \item $R(x) = 2^x$ \\ $R(x)$ corresponde a la secuencia numérica que sólo consta del número $x$ (para $x>0$).
    \item $E(x) = R(11) * x * R(13)$ \\ $E(x)$ corresponde a la operación de poner entre paréntesis (11 y 13 son los números asignados a los 
            signos primitivos «(» y «)» ).
    \item $n text{ Var } x \leftrightarrow \exists z (13 < z \leq x \wedge text{Prim}(z) \wedge x = z^n) \wedge n \neq 0$ \\ $x$ es una VARIABLE DE TIPO $n$.
    \item $text{ Var } x \leftrightarrow \exists n (n \leq x \wedge n text{ Var } x )$ \\ $x$ es una VARIABLE.
    \item $text{Neg}(x) = R(5) * E(x)$ \\ $text{Neg}(x)$ es la NEGACIÓN de $x$.
    \item $x \text{ Dis } y = E(x) * R(7) * E(y)$ \\ $x \text{ Dis } y$ es la DISYUNCIÓN de $x$ e $y$.
    \item $x \text{ Gen } y = R(9) * R(x) * E(y)$ \\ $x \text{ Gen } y$ es la GENERALIZACIÓN de $y$ respecto a la variable $x$ (suponiendo que $x$ sea una variable).
    \item $0 \thinspace N \thinspace x = x$ \\ $(n+1) \thinspace N \thinspace x = R(3) * n \thinspace N \thinspace x$ \\ 
            $n \thinspace N \thinspace x$ corresponde a la operación de poner $n$ veces el signo $s$ delante de $x$.
    \item $Z(n) = n \thinspace N \thinspace (R(1))$ \\ $Z(n)$ es el NUMERAL que designa el número $n$.
    \item $\text{Typ}_1' \thinspace x \leftrightarrow \exists m \thinspace n(m,n \leq x \wedge (m=1 \vee 1 \text{ Var } m) 
            \wedge x = n \thinspace N \thinspace (R(m)))$\footnote{$m,n \leq x$ es una abreviatura de $m \leq x \wedge n \leq x$ (y lo mismo para 
            más de dos variables)} \\ $x$ es un SIGNO DE TIPO $1$.
    \item $\text{Typ}_n \thinspace x \leftrightarrow (n=1 \wedge \text{Typ}_1' \thinspace (x)) \wedge (n>1 \wedge \exists v (v \leq x \wedge n \text{ Var } v \wedge x = R(v)))$ \\
            $x$ es un SIGNO DE TIPO $n$.
    \item $Elf \thinspace x \leftrightarrow \exists yzn (y, z, n \leq x \wedge \text{Typ}_n(y) \wedge \text{Typ}_{n+1}(z) \wedge x = x* E(y))$ \\ $x$ es una FÓRMULA ELEMENTAL.
    \item $Op \thinspace x \thinspace y \thinspace z \leftrightarrow x = \text{Neg}(y) \wedge x = y \text{ Dis } z \vee \exists v (v \leq x \wedge \text{Var } v \wedge 
            x = v \text{ Gen } y)$ 
    \item $\begin{aligned}
            FR \thinspace x \leftrightarrow \forall n (0 < n \leq l(x) \rightarrow Elf(n \thinspace Gl \thinspace x) &\vee \exists pq (0 < p,q < n \wedge \\
            \wedge &Op(n \thinspace Gl \thinspace x, p \thinspace Gl \thinspace x,q \thinspace Gl \thinspace x))) \wedge l(x) > 0
        \end{aligned}$ \\
            $x$ es una SECUENCIA DE FÓRMULAS, cada una de las cuales es o una FÓRMULA ELEMENTAL o se obtiene de las precedentes mediante las operaciones de 
            NEGACIÓN, DISYUNCIÓN o GENERALIZACIÓN.
    \item $\text{Form } x \leftrightarrow \exists n (n \leq (Pr(l(x)^2))^{x \cdot (l(x)^2)} \wedge FR \thinspace n \wedge x = (l(x)) \thinspace Gl \thinspace n)$\footnote{La
            acotación $n \leq (Pr(l(x)^2))^{x \cdot (l(x)^2)}$ puede comprobarse así: La longitud de la más corta secuencia de  fórmulas correspondientes a $x$ puede ser a lo 
            sumo igual al número de subfórmulas de $x$. Pero hay a lo sumo $l(x)$ subfórmulas de longitud $1$, a lo sumo $l(x) -1$ de longitud $2$, \dots, por tanto en conjunto
            a lo sumo $\frac{l(x) \cdot (l(x) + 1)}{2} \leq n \leq (l(x))^2$. Por tanto podemos suponer que todos los factores primos de $n$ son menores que $Pr((l(x))^2)$, 
            que su número es $\leq (l(x))^2$ y que sus exponentes (que son subfórmulas de $x$) son $\leq x$.} \\ 
            $x$ es una FÓRMULA (es decir, el último miembro de una SECUENCIA DE FÓRMULAS $n$).
    \item $v \thinspace \text{Geb} \thinspace n, x \leftrightarrow \text{Var } v \wedge \text{Form } x \wedge \exists a \thinspace b \thinspace c \thinspace 
            (a, b, c \leq x \wedge x = a*(v \text{ Gen } b)* c \wedge \text{Form } x \wedge l(a) + 1 \leq n \leq l(a) + l(v \text{ Gen } b))$ \\ La VARIABLE $v$
            está LIGADA en $x$ en el $n$-avo lugar.
    \item $v \thinspace Fr \thinspace n, x \leftrightarrow \text{Var } v \wedge \text{Form } x \wedge v = n \thinspace Gl \thinspace x \wedge n \leq l(x) 
            \wedge \lnot v \text{ Geb } n, x$ \\ La VARIABLE $v$ está LIBRE en $x$  en el $n$-avo lugar.
    \item $v \thinspace Fr \thinspace x \leftrightarrow \exists n (n \leq l(x) \wedge v \thinspace Fr \thinspace n, x)$ \\ $v$ aparece como VARIABLE LIBRE EN $x$.
    \item $Su \thinspace x\tbinom{n}{y} = \mu z (z \leq (Pr(l(x) + l(y)))^{x+y} \wedge \exists u \thinspace v(u,v \leq x \wedge x = u * R(n \thinspace Gl \thinspace x) * v \wedge \\ 
            \wedge z =  u * y * v \wedge n = l(u) + 1))$ \\
            $Su \thinspace x\tbinom{n}{y}$ se obtiene a partir de $x$ cuando sustituimos el $n$-avo miembro de $x$ por $y$ (suponiendo que $0 < n \leq (x)$).
    \item $\qquad \quad 0 \thinspace St \thinspace v, x = \mu n (n \leq l(x) \wedge v \thinspace Fr \thinspace n, x \wedge \lnot \exists p(n < p \leq l(x) \wedge v \thinspace Fr \thinspace p, x))$\\
            $(k+1) \thinspace St \thinspace v, x = \mu n (n < k \thinspace St \thinspace v, x \wedge v \thinspace Fr \thinspace n, x \wedge \lnot \exists 
            p(n < p \leq k \thinspace St \thinspace v, x \wedge v \thinspace Fr \thinspace p, x))$\\
            $k \thinspace St \thinspace v, x$ es el $k+1$-avo lugar de $x$ (contado a partir del extremo derecho de la FÓRMULA $x$), en el que $v$ aparece LIBRE en $x$ 
            (y es $0$ si no hay tal lugar).
    \item $A(v,x) = \mu n (n \leq l(x) \wedge n \thinspace St \thinspace v, x = 0)$ \\ $A(v,x)$ es el número de lugares en que $v$ aparece LIBRE en $x$.
    \item $Sb_0 (x^v_y) = x$ \\ $Sb_{k+1} (x^v_y) = Su \thinspace (Sb_k (x^v_y)) \tbinom{\text{\textit{k St v,x}}}{y}$
    \item $Sb (x^v_y) = Sb_{A(v,x)} (x^v_y)$\footnote{Si $v$ no es una VARIABLE  o $x$ no es una FÓRMULA, enotnces $Sb (x^v_y) = x$.}\\
            $Sb (x^v_y)$ es la noción anteriormente definida de $\daleth^b_v \alpha$\footnote{En vez de $Sb(Sb(x^v_w)^w_z)$ escribimos $Sb \thinspace x^{v \thinspace w}_{y \thinspace z}$, 
            (y análogamente para más de dos VARIABLES).}.
    \item $x \text{ Imp } y = (\text{Neg}(x)) \text{ Dis } y \\ x \text{ Con } y = \text{Neg } ((\text{Neg}(x)) \text{ Dis } (\text{Neg}(y))) \\$
\end{enumerate}




\endinput
%------------------------------------------------------------------------------------
% FIN DEL CAPÍTULO. 
%------------------------------------------------------------------------------------

