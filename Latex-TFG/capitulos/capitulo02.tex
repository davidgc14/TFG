% !TeX root = ../libro.tex
% !TeX encoding = utf8

\chapter{La completitud del cálculo lógico de primer orden}


\section{Exposición y demostraciones}

El teorema de completitud semántica de la lógica de primer orden aparece en el artículo de Gödel de 1930 como Teorema I:

\begin{teorema}\label{thmI:teorema}
    Cada fórmula válida de la lógica de primer orden es deducible.
\end{teorema}

El presente teorema, objeto principal de estudio de esta sección, sería trivialmente demostrable si pudiesemos probar
el siguiente:
\begin{teorema}\label{thmII:teorema}
    Cada fórmula de la lógica de primer orden es o refutable
    \footnote{<<$\varphi$ es refutable>> significa <<$\lnot \varphi$ es deducible>>.} o satisfacible (sobre un universo infinito numerable).
\end{teorema}

Y por ello surge el siguiente resultado: 

\begin{proposicion}
    \autoref{thmII:teorema} $\Rightarrow$ \autoref{thmI:teorema}
\end{proposicion}
\begin{proof}
    Sea $\alpha$ una fórmula válida. Siendo esto así, $\lnot \alpha$ no es satisfacible, y aplicando \autoref{thmII:teorema} 
    tenemos que $\alpha$ es refutable. Por tanto, con ello se tiene que $\lnot \lnot \alpha$ (y como consecuencia, también $\alpha$) es una fórmula
    deducible.\footnote{El recíproco del anterior resultado también es cierto y con una demostración igual de simple, aunque no la demostraremos por no ser relevante
    en la demostración del \autoref{thmI:teorema}.}
\end{proof}

\begin{definicion}
    Una \textit{K-fórmula} es una fórmula $\kappa$ perteneciente a una clase del conjunto de fórmulas $K$ cumpliendo las siguientes condiciones:
    \begin{enumerate}
        \item $\kappa$ es una fórmula prenexa.
        \item $\kappa$ carece de variables individuales libres.
        \item El prefijo de $\kappa$ comienza con un cuantificador universal y termina con un cuantificador particular.
    \end{enumerate}
\end{definicion}

Entonces con la presente definición podemos deducir el siguiente resultado:

\begin{teorema} \label{thmIII:teorema}
    Si cada $K$-fórmula es refutable o satisfacible, también lo es cualquier fórmula.
\end{teorema}

\begin{proof}
    Sea $\alpha$ una fórmula que no pertenece a $K$. Sea $\mathfrak{x}$ el conjunto de sus variables libres. Como se puede ver
    directamente, si $\alpha$ es refutable (o satisfacible), se sigue la refutabilidad (o satisfacibilidad) de 
    $\exists \mathfrak{x} \alpha$, e igualmente a la inversa. 

    Sea ahora $\pi \varphi$ la forma normal prenexa de $\exists \mathfrak{x} \alpha$, de tal modo que
    \begin{equation}\label{eq:prenexa}
        \exists \mathfrak{x} \alpha \leftrightarrow \mathfrak{x}
    \end{equation}
    \begin{equation}
        \exists \mathfrak{x} \alpha \leftrightarrow \mathfrak{x}
    \end{equation}

    \autoref{eq:prenexa}

\end{proof}



\chapter{El teorema de Incompletitud}

\endinput
%------------------------------------------------------------------------------------
% FIN DEL CAPÍTULO. 
%------------------------------------------------------------------------------------

