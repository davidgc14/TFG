% !TeX root = ../libro.tex
% !TeX encoding = utf8

\chapter{La completitud del cálculo lógico de primer orden}

\section{Definiciones y lemas previos}

Definamos previamente el sistema axiomático sobre el que vamos a trabajar:
\footnote{Coincide, exceptuando el principio de asociatividad (que es redundante), con lo expuesto en los apartados 1 y 10 de 
\textit{Principia Mathematica} (Véase \cite{an1910principia})}
\begin{itemize}
    \item Signos básicos primitivos:\footnote{A partir de ellos pueden definirse $\wedge$, $\rightarrow$, $\leftrightarrow$ y 
            $\exists$ del modo habitual.}
            \begin{itemize}
                \item $\lnot$
                \item $\vee$
                \item $\forall$
            \end{itemize}
    \item Axiomas formales:
            \begin{enumerate}
                \item $ \text{X} \vee \text{X} \rightarrow \text{X} $
                \item $ \text{X} \rightarrow \text{X} \vee \text{Y} $
                \item $ \text{Y} \vee \text{X} \rightarrow \text{X} \vee \text{Y} $
                \item $ (\text{X} \rightarrow \text{Y}) \rightarrow (\text{Z} \vee \text{X} \rightarrow \text{Z} \vee \text{Y}) $
                \item $ \forall \text{x Px} \rightarrow \text{Py} $
                \item $ \forall \text{x } (\text{X} \vee \text{Px}) \rightarrow \text{X} \vee \forall \text{x Px} $
            \end{enumerate}
    \item Reglas de inferencia: \footnote{No todas ellas están explícitamente formuladas en los resultados de Russel y Whitehead,
            pero todas ellas son usadas continuamente en sus deducciones.}
            \begin{enumerate}
                \item El esquema de inferencia: De $\alpha$ y $\alpha \rightarrow \beta$ se puede inferir $\beta$.
                \item La regla de sustitución para variables sentenciales y predicativas.
                \item De $\alpha(\text{x})$ puede inferirse $\forall \text{x } \alpha(\text{x})$.
                \item Unas variables individuales (libres o ligadas) pueden ser reemplazadas por cualesquiera otras, con tal de que 
                        ello no produzca ningún solapamiento de los alcances de las variables designadas mediante el mismo signo.
            \end{enumerate}
\end{itemize}

Para las deducciones que van a proceder, también es conveniente establecer algunas abreviaturas en calidad de notación:

\begin{enumerate}
    \item Las notaciones $\pi_1, \thinspace \pi_2, \thinspace \pi_3, \thinspace \rho, $ ... designan prefijos\footnote{Con prefijos 
    se hace referencia a conjuntos de signos básicos primitivos.} cualesquiera, es decir, filas de signos
            finitas de la forma $\forall \text{x} \exists \text{y}, \thinspace \forall \text{x} \forall \text{y} \exists \text{z} \forall \text{u},$ ...
    \item Las letras alemanas minúsculas $\mathfrak{x, \thinspace y, \thinspace u, \thinspace v,} $ etc., designan \textit{n-tuplos} de variables
            individuales, es decir, filas de signos del tipo $\text{xyz}, \thinspace \text{x}_2\text{x}_1\text{x}_2\text{x}_3,$ ..., 
            donde la misma variable puede aparecer varias veces en la misma sentencia.
            \begin{itemize}
                \item[Nota:] Del modo correspondiente hay que entender sentencias como $\forall \mathfrak{x} \thinspace \exists \mathfrak{y},$ etc. 
                            Si una misma variable aparece varias veces en $\mathfrak{x}$, hay que pensar, naturamente, que sólo está escrita una vez
                            en $\forall \mathfrak{x} \thinspace \exists \mathfrak{y},$ ...
            \end{itemize}
\end{enumerate}

Consideremos a continuación una serie de lemas, para las que no presentamos demostración por no ser relevantes para el desarrollo del resto de resultados 
(además de ser demostraciones relativamente sencillas de realizar).

\begin{lema} \label{lem:lema-1}
    Para cada n-tuplo $\mathfrak{x}$ los siguientes resultados son deducibles:
    \begin{enumerate}
        \item[$a)$] $ \forall \mathfrak{x} \text{F} \mathfrak{x} \rightarrow \exists \mathfrak{x} \text{F} \mathfrak{x}  $
        \item[$b)$] $ \forall \mathfrak{x} \text{F} \mathfrak{x} \wedge \exists \mathfrak{x} \text{G} \mathfrak{x} \rightarrow \exists \mathfrak{x} 
                    (\text{F} \mathfrak{x} \wedge \text{G} \mathfrak{x}) $
        \item[$c)$] $ \forall \mathfrak{x} \lnot \text{F} \mathfrak{x} \leftrightarrow \exists \mathfrak{x} \text{F} \mathfrak{x} $
    \end{enumerate}
\end{lema}

\begin{lema} \label{lem:lema-2}
    Si $\mathfrak{x}$ y $\mathfrak{x}'$ sólo se diferencian por el orden en que están escritas las variables, entonces el siguente enunciado es deducible:
    $$\exists \mathfrak{x} \text{F} \mathfrak{x} \rightarrow \exists \mathfrak{x}' \text{F} \mathfrak{x}'$$ 
\end{lema}

\begin{lema} \label{lem:lema-3}
    Si todas las variables $\mathfrak{x}$ son distintas entre sí y $\mathfrak{x}'$ tiene el mismo número de miembros que $\mathfrak{x}$, entonces el siguiente enunciado
    es deducible:
    $$\forall \mathfrak{x} \text{F} \mathfrak{x} \rightarrow \forall \mathfrak{x}' \text{F} \mathfrak{x}'$$
\end{lema}

\begin{lema} \label{lem:lema-4}
    Si $\pi_i$ designa uno de los prefijos $\forall x_i$, $\exists x_i$; y $\rho_i$ designa uno de los prefijos $\forall y_i$, $\exists y_i$, 
    entonces el siguiente es deducible:\footnote{Un resultado análogo vale para $\vee$ en vez de para $\wedge$.}
    $$ \pi_1 \pi_2 \cdots \pi_n Fx_1 x_2 \cdots x_n \wedge \rho_1 \rho_2 \cdots \rho_m G y_1 y_2 \cdots y_m \leftrightarrow \pi(Fx_1 x_2 \cdots x_n \wedge Gy_1 y_2 \cdots y_m ) $$
    para cada prefijo $\pi$ que se componga de los $\pi_i$ y $\rho_i$ y que satisfaga la condición de que $\pi_i$ esté delante de $\pi_k$ 
    para $i < k \leq n$ y de que $\rho_i$ esté delante de $\rho_k$ para $i < k \leq m$ 
\end{lema}

\begin{lema} \label{lem:lema-5}
    Toda fórmula puede ponerse en forma normal prenexa, es decir, para cada fórmula $\alpha$ hay una fórmula prenexa $\gamma$, tal que $\alpha \leftrightarrow \gamma$ es deducible.
\end{lema}

\begin{lema} \label{lem:lema-6}
    Si $\alpha \leftrightarrow \beta$ es deducible, entonces también lo es $\varphi(\alpha) \leftrightarrow \varphi(\beta)$, 
    donde $\varphi(\alpha)$ designa una fórmula cualquiera que contenga $\alpha$ como parte.\footnote{Estos dos últimos resultados se pueden estudiar 
    en detalle en la tercera sección de \cite{hilbert1962elementos}.}  
\end{lema}

\begin{lema} \label{lem:lema-7}
    Cada fórmula conectiva válida es deducible, es decir, los axiomas 1-4 constituyen un sistema suficiente de axiomas para el cálculo conectivo.
\end{lema}

Con estos resultados previos ya estamos en condiciones de afrontar el problema que nos concierne.


%%%%%%%%%%%%%%%%%%%%%%%%%%%%%%%%%%%%%
\section{Exposición y demostraciones}

El teorema de completitud semántica de la lógica de primer orden aparece en el artículo de Gödel de 1930 como Teorema I:

\begin{teorema}\label{thm:teoremaI}
    Cada fórmula válida de la lógica de primer orden es deducible.
\end{teorema}

El presente teorema, objeto principal de estudio de esta sección, sería trivialmente demostrable si pudiesemos probar
el siguiente:
\begin{teorema}\label{thm:teoremaII}
    Cada fórmula de la lógica de primer orden es o refutable
    \footnote{<<$\varphi$ es refutable>> significa <<$\lnot \varphi$ es deducible>>.} o satisfacible (sobre un universo infinito numerable).
\end{teorema}

Y por ello surge el siguiente resultado: 

\begin{proposicion}
    \autoref{thm:teoremaII} $\Rightarrow$ \autoref{thm:teoremaI}
\end{proposicion}
\begin{proof}
    Sea $\alpha$ una fórmula válida. Siendo esto así, $\lnot \alpha$ no es satisfacible, y aplicando \autoref{thm:teoremaII} 
    tenemos que $\alpha$ es refutable. Por tanto, con ello se tiene que $\lnot \lnot \alpha$ (y como consecuencia, también $\alpha$) es una fórmula
    deducible.\footnote{El recíproco del anterior resultado también es cierto y con una demostración igual de simple, aunque no la demostraremos por no ser relevante
    en la demostración del \autoref{thm:teoremaI}.}
\end{proof}

\begin{definicion}
    Una \textit{K-fórmula} es una fórmula $\kappa$ perteneciente a una clase del conjunto de fórmulas $K$ cumpliendo las siguientes condiciones:
    \begin{enumerate}
        \item $\kappa$ es una fórmula prenexa.
        \item $\kappa$ carece de variables individuales libres.
        \item El prefijo de $\kappa$ comienza con un cuantificador universal y termina con un cuantificador particular.
    \end{enumerate}
\end{definicion}

Entonces con la presente definición podemos deducir el siguiente resultado:

\begin{teorema} \label{thm:teoremaIII}
    Si cada $K$-fórmula es refutable o satisfacible, también lo es cualquier fórmula.
\end{teorema}

\begin{proof}
    Sea $\alpha$ una fórmula que no pertenece a $K$. Sea $\mathfrak{x}$ el conjunto de sus variables libres. Como se puede ver
    directamente, si $\alpha$ es refutable (o satisfacible), se sigue la refutabilidad (o satisfacibilidad) de 
    $\exists \mathfrak{x} \alpha$, e igualmente a la inversa. 

    Sea ahora $\pi \varphi$ la forma normal prenexa de $\exists \mathfrak{x} \alpha$, de tal modo que
    \begin{equation}\label{eq:III-1}
        \exists \mathfrak{x} \alpha \leftrightarrow \pi \varphi
    \end{equation}
    es deducible. Además estipulemos que
    \begin{equation}
        \beta = \forall x \pi \exists y (\varphi \wedge \text{Fx} \vee \lnot \text{Fy})
        \footnote{Las variables x e y no deben aparecer en $\pi$.}
    \end{equation}
    
    Entonces
    \begin{equation}\label{eq:III-2}
        \pi \varphi \leftrightarrow \beta
    \end{equation}
    es deducible (por el \autoref{lem:lema-4} y por la deducibilidad de $\forall x \pi \exists y (\varphi \wedge \text{Fx} \vee \lnot \text{Fy})$).
    $\beta$ pertenece a $K$ y, por tanto, es o satisfacible o refutable. Pero por \eqref{eq:III-1} y \eqref{eq:III-2}
    la satisfacibilidad de $\beta$ implica la de $\exists \mathfrak{x} \alpha$ y consiguientemente también la de $\alpha$,
    y lo mismo se puede aplicar para la refutabilidad. Por tanto, concluimos que también $\alpha$ es o satisfacible o refutable.
\end{proof}

Teniendo en cuenta el \autoref{thm:teoremaIII} anterior, basta para demostrar el \autoref{thm:teoremaII} con probar la siguiente:

\begin{proposicion} \label{prop:satisf-refut}
    Cada $K$-fórmula es satisfacible o refutable.
\end{proposicion}

Para ello definimos previamente el concepto de grado de una $K$-fórmula, y probaremos algunos resultados para poder probar la anterior
proposición.

\begin{definicion}
    Llamaremos grado de una $K$-fórmula
    \footnote{También, en el mismo sentido se le puede llamar "grado de un prefijo".}
    al número de series de cuantificaciones universales de su prefijo, separadas unas de otras
    por cuantificadores existenciales.
\end{definicion}

Probamos primeramente el siguiente resultado.

\begin{teorema}\label{thm:teoremaIV}
    Si cada $K$-fórmula de grado $n$ es o refutable o satisfacible, entonces también lo es cada $K$-fórmula de grado $n+1$.
\end{teorema}

\begin{proof}
    Sea $\pi_1 \alpha$ una $K$-fórmula de grado $n+1$. Sea $\pi_1 = \forall \mathfrak{x} \exists \mathfrak{y} \pi_2$ y 
    $\pi_2 = \forall \mathfrak{u} \exists \mathfrak{v} \pi_3$, donde $\pi_2$ tiene grado $n$ y $\pi_3$ el grado $n-1$. Sea además 
    $\text{F}$ una variable predicativa que no aparezca en $\alpha$. Establezcamos:
    \begin{equation}
        \beta = \forall \mathfrak{x}' \exists \mathfrak{y}' \text{F}  \mathfrak{x}' \mathfrak{y}' \wedge 
        \forall \mathfrak{x} \forall \mathfrak{y} (\text{F}  \mathfrak{x} \mathfrak{y} \rightarrow \pi_2 \alpha)
    \end{equation}
    y
    \begin{equation}
        \gamma = \forall \mathfrak{x}' \forall \mathfrak{x} \forall \mathfrak{y} \forall \mathfrak{u} \exists \mathfrak{y}' \exists \mathfrak{v} 
        \pi_3 (\text{F} \mathfrak{x}' \mathfrak{y}' \wedge (\text{F} \mathfrak{x} \mathfrak{y}) \rightarrow \alpha) \footnote{En esta sentencia
        suponemos que las sucesiones de variables $\mathfrak{x}, \mathfrak{x}', \mathfrak{y}, \mathfrak{y}', \mathfrak{u}, \mathfrak{v}$ son disjuntas
        entre sí.} % esta llave sale en rojo y no se por que, pero no da error
    \end{equation}

    Aplicando ahora dos veces el \autoref{lem:lema-4} junto con el \autoref{lem:lema-6}, obtenemos la deducibilidad de 
\end{proof}



\chapter{El teorema de Incompletitud}

\endinput
%------------------------------------------------------------------------------------
% FIN DEL CAPÍTULO. 
%------------------------------------------------------------------------------------

