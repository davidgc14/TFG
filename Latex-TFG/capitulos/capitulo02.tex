% !TeX root = ../libro.tex
% !TeX encoding = utf8

\chapter{La completitud del cálculo lógico de primer orden}

\section{Definiciones y lemas previos}

Definamos previamente el sistema axiomático sobre el que vamos a trabajar:
\footnote{Coincide, exceptuando el principio de asociatividad (que es redundante), con lo expuesto en los apartados 1 y 10 de 
\textit{Principia Mathematica} (Véase \cite{an1910principia})}
\begin{itemize}
    \item Signos básicos primitivos:\footnote{A partir de ellos pueden definirse $\wedge$, $\rightarrow$, $\leftrightarrow$ y 
            $\exists$ del modo habitual.}
            \begin{itemize}
                \item $\lnot$
                \item $\vee$
                \item $\forall$
            \end{itemize}
    \item Axiomas formales:
            \begin{enumerate}
                \item $ \text{X} \vee \text{X} \rightarrow \text{X} $
                \item $ \text{X} \rightarrow \text{X} \vee \text{Y} $
                \item $ \text{Y} \vee \text{X} \rightarrow \text{X} \vee \text{Y} $
                \item $ (\text{X} \rightarrow \text{Y}) \rightarrow (\text{Z} \vee \text{X} \rightarrow \text{Z} \vee \text{Y}) $
                \item $ \forall \text{x Px} \rightarrow \text{Py} $
                \item $ \forall \text{x } (\text{X} \vee \text{Px}) \rightarrow \text{X} \vee \forall \text{x Px} $
            \end{enumerate}
    \item Reglas de inferencia: \footnote{No todas ellas están explícitamente formuladas en los resultados de Russel y Whitehead,
            pero todas ellas son usadas continuamente en sus deducciones.}
            \begin{enumerate}
                \item El esquema de inferencia: De $\alpha$ y $\alpha \rightarrow \beta$ se puede inferir $\beta$.
                \item La regla de sustitución para variables sentenciales y predicativas.
                \item De $\alpha(\text{x})$ puede inferirse $\forall \text{x } \alpha(\text{x})$.
                \item Unas variables individuales (libres o ligadas) pueden ser reemplazadas por cualesquiera otras, con tal de que 
                        ello no produzca ningún solapamiento de los alcances de las variables designadas mediante el mismo signo.
            \end{enumerate}
\end{itemize}

Para las deducciones que van a proceder, también es conveniente establecer algunas abreviaturas en calidad de notación:

\begin{enumerate}
    \item Las notaciones $\pi_1, \thinspace \pi_2, \thinspace \pi_3, \thinspace \rho, $ ... designan prefijos\footnote{Con prefijos 
    se hace referencia a conjuntos de signos básicos primitivos.} cualesquiera, es decir, filas de signos
            finitas de la forma $\forall \text{x} \exists \text{y}, \thinspace \forall \text{x} \forall \text{y} \exists \text{z} \forall \text{u},$ ...
    \item Las letras alemanas minúsculas $\mathfrak{x, \thinspace y, \thinspace u, \thinspace v,} $ etc., designan \textit{n-tuplos} de variables
            individuales, es decir, filas de signos del tipo $\text{xyz}, \thinspace \text{x}_2\text{x}_1\text{x}_2\text{x}_3,$ ..., 
            donde la misma variable puede aparecer varias veces en la misma sentencia.
            \begin{itemize}
                \item[Nota:] Del modo correspondiente hay que entender sentencias como $\forall \mathfrak{x} \thinspace \exists \mathfrak{y},$ etc. 
                            Si una misma variable aparece varias veces en $\mathfrak{x}$, hay que pensar, naturamente, que sólo está escrita una vez
                            en $\forall \mathfrak{x} \thinspace \exists \mathfrak{y},$ ...
            \end{itemize}
\end{enumerate}

Consideremos a continuación una serie de lemas, para las que no presentamos demostración por no ser relevantes para el desarrollo del resto de resultados 
(además de ser demostraciones relativamente sencillas de realizar).

\begin{lema} \label{lem:lema-1}
    Para cada n-tuplo $\mathfrak{x}$ los siguientes resultados son deducibles:
    \begin{enumerate}
        \item[$a)$] $ \forall \mathfrak{x} \text{F} \mathfrak{x} \rightarrow \exists \mathfrak{x} \text{F} \mathfrak{x}  $
        \item[$b)$] $ \forall \mathfrak{x} \text{F} \mathfrak{x} \wedge \exists \mathfrak{x} \text{G} \mathfrak{x} \rightarrow \exists \mathfrak{x} 
                    (\text{F} \mathfrak{x} \wedge \text{G} \mathfrak{x}) $
        \item[$c)$] $ \forall \mathfrak{x} \lnot \text{F} \mathfrak{x} \leftrightarrow \exists \mathfrak{x} \text{F} \mathfrak{x} $
    \end{enumerate}
\end{lema}

\begin{lema} \label{lem:lema-2}
    Si $\mathfrak{x}$ y $\mathfrak{x}'$ sólo se diferencian por el orden en que están escritas las variables, entonces el siguente enunciado es deducible:
    $$\exists \mathfrak{x} \text{F} \mathfrak{x} \rightarrow \exists \mathfrak{x}' \text{F} \mathfrak{x}'$$ 
\end{lema}

\begin{lema} \label{lem:lema-3}
    Si todas las variables $\mathfrak{x}$ son distintas entre sí y $\mathfrak{x}'$ tiene el mismo número de miembros que $\mathfrak{x}$, entonces el siguiente enunciado
    es deducible:
    $$\forall \mathfrak{x} \text{F} \mathfrak{x} \rightarrow \forall \mathfrak{x}' \text{F} \mathfrak{x}'$$
\end{lema}

\begin{lema} \label{lem:lema-4}
    Si $\pi_i$ designa uno de los prefijos $\forall x_i$, $\exists x_i$; y $\rho_i$ designa uno de los prefijos $\forall y_i$, $\exists y_i$, 
    entonces el siguiente es deducible:\footnote{Un resultado análogo vale para $\vee$ en vez de para $\wedge$.}
    $$ \pi_1 \pi_2 \cdots \pi_n Fx_1 x_2 \cdots x_n \wedge \rho_1 \rho_2 \cdots \rho_m G y_1 y_2 \cdots y_m \leftrightarrow \pi(Fx_1 x_2 \cdots x_n \wedge Gy_1 y_2 \cdots y_m ) $$
    para cada prefijo $\pi$ que se componga de los $\pi_i$ y $\rho_i$ y que satisfaga la condición de que $\pi_i$ esté delante de $\pi_k$ 
    para $i < k \leq n$ y de que $\rho_i$ esté delante de $\rho_k$ para $i < k \leq m$ 
\end{lema}

\begin{lema} \label{lem:lema-5}
    Toda fórmula puede ponerse en forma normal prenexa, es decir, para cada fórmula $\alpha$ hay una fórmula prenexa $\gamma$, tal que $\alpha \leftrightarrow \gamma$ es deducible.
\end{lema}

\begin{lema} \label{lem:lema-6}
    Si $\alpha \leftrightarrow \beta$ es deducible, entonces también lo es $\varphi(\alpha) \leftrightarrow \varphi(\beta)$, 
    donde $\varphi(\alpha)$ designa una fórmula cualquiera que contenga $\alpha$ como parte.\footnote{Estos dos últimos resultados se pueden estudiar 
    en detalle en la tercera sección de \cite{hilbert1962elementos}.}  
\end{lema}

\begin{lema} \label{lem:lema-7}
    Cada fórmula conectiva válida es deducible, es decir, los axiomas 1-4 constituyen un sistema suficiente de axiomas para el cálculo conectivo.
\end{lema}

Con estos resultados previos ya estamos en condiciones de afrontar el problema que nos concierne.


%%%%%%%%%%%%%%%%%%%%%%%%%%%%%%%%%%%%%
\section{Exposición y demostraciones}

El teorema de completitud semántica de la lógica de primer orden aparece en el artículo de Gödel de 1930 como Teorema I:

\begin{teorema}\label{thm:teoremaI}
    Cada fórmula válida de la lógica de primer orden es deducible.
\end{teorema}

El presente teorema, objeto principal de estudio de esta sección, sería trivialmente demostrable si pudiesemos probar
el siguiente:
\begin{teorema}\label{thm:teoremaII}
    Cada fórmula de la lógica de primer orden es o refutable
    \footnote{<<$\varphi$ es refutable>> significa <<$\lnot \varphi$ es deducible>>.} o satisfacible (sobre un universo infinito numerable).
\end{teorema}

Y por ello surge el siguiente resultado: 

\begin{proposicion}
    \autoref{thm:teoremaII} $\Rightarrow$ \autoref{thm:teoremaI}
\end{proposicion}
\begin{proof}
    Sea $\alpha$ una fórmula válida. Siendo esto así, $\lnot \alpha$ no es satisfacible, y aplicando \autoref{thm:teoremaII} 
    tenemos que $\alpha$ es refutable. Por tanto, con ello se tiene que $\lnot \lnot \alpha$ (y como consecuencia, también $\alpha$) es una fórmula
    deducible.\footnote{El recíproco del anterior resultado también es cierto y con una demostración igual de simple, aunque no la demostraremos por no ser relevante
    en la demostración del \autoref{thm:teoremaI}.}
\end{proof}

\begin{definicion}
    Una \textit{K-fórmula} es una fórmula $\kappa$ perteneciente a una clase del conjunto de fórmulas $K$ cumpliendo las siguientes condiciones:
    \begin{enumerate}
        \item $\kappa$ es una fórmula prenexa.
        \item $\kappa$ carece de variables individuales libres.
        \item El prefijo de $\kappa$ comienza con un cuantificador universal y termina con un cuantificador particular.
    \end{enumerate}
\end{definicion}

Entonces con la presente definición podemos deducir el siguiente resultado:

\begin{teorema} \label{thm:teoremaIII}
    Si cada $K$-fórmula es refutable o satisfacible, también lo es cualquier fórmula.
\end{teorema}

\begin{proof}
    Sea $\alpha$ una fórmula que no pertenece a $K$. Sea $\mathfrak{x}$ el conjunto de sus variables libres. Como se puede ver
    directamente, si $\alpha$ es refutable (o satisfacible), se sigue la refutabilidad (o satisfacibilidad) de 
    $\exists \mathfrak{x} \alpha$, e igualmente a la inversa. 

    Sea ahora $\pi \varphi$ la forma normal prenexa de $\exists \mathfrak{x} \alpha$, de tal modo que
    \begin{equation}\label{eq:III-1}
        \exists \mathfrak{x} \alpha \leftrightarrow \pi \varphi
    \end{equation}
    es deducible. Además estipulemos que
    \begin{equation}
        \beta = \forall x \pi \exists y (\varphi \wedge \text{Fx} \vee \lnot \text{Fy})
        \footnote{Las variables x e y no deben aparecer en $\pi$.}
    \end{equation}
    
    Entonces
    \begin{equation}\label{eq:III-2}
        \pi \varphi \leftrightarrow \beta
    \end{equation}
    es deducible (por el \autoref{lem:lema-4} y por la deducibilidad de $\forall x \pi \exists y (\varphi \wedge \text{Fx} \vee \lnot \text{Fy})$).
    $\beta$ pertenece a $K$ y, por tanto, es o satisfacible o refutable. Pero por \eqref{eq:III-1} y \eqref{eq:III-2}
    la satisfacibilidad de $\beta$ implica la de $\exists \mathfrak{x} \alpha$ y consiguientemente también la de $\alpha$,
    y lo mismo se puede aplicar para la refutabilidad. Por tanto, concluimos que también $\alpha$ es o satisfacible o refutable.
\end{proof}

Teniendo en cuenta el \autoref{thm:teoremaIII} anterior, basta para demostrar el \autoref{thm:teoremaII} con probar la siguiente:

\begin{proposicion} \label{prop:satisf-refut}
    Cada $K$-fórmula es satisfacible o refutable.
\end{proposicion}

Para ello definimos previamente el concepto de grado de una $K$-fórmula, y probaremos algunos resultados para poder probar la anterior
proposición.

\begin{definicion}
    Llamaremos grado de una $K$-fórmula
    \footnote{También, en el mismo sentido se le puede llamar "grado de un prefijo".}
    al número de series de cuantificaciones universales de su prefijo, separadas unas de otras
    por cuantificadores existenciales.
\end{definicion}

Probamos primeramente el siguiente resultado.

\begin{teorema}\label{thm:teoremaIV}
    Si cada $K$-fórmula de grado $n$ es o refutable o satisfacible, entonces también lo es cada $K$-fórmula de grado $n+1$.
\end{teorema}

\begin{proof}
    Sea $\pi_1 \alpha$ una $K$-fórmula de grado $n+1$. Sea $\pi_1 = \forall \mathfrak{x} \exists \mathfrak{y} \pi_2$ y 
    $\pi_2 = \forall \mathfrak{u} \exists \mathfrak{v} \pi_3$, donde $\pi_2$ tiene grado $n$ y $\pi_3$ el grado $n-1$. Sea además 
    $\text{F}$ una variable predicativa que no aparezca en $\alpha$. Establezcamos:
    \begin{equation}
        \beta = \forall \mathfrak{x}' \exists \mathfrak{y}' \text{F}  \mathfrak{x}' \mathfrak{y}' \wedge 
        \forall \mathfrak{x} \forall \mathfrak{y} (\text{F}  \mathfrak{x} \mathfrak{y} \rightarrow \pi_2 \alpha)
    \end{equation}
    y
    \begin{equation}
        \gamma = \forall \mathfrak{x}' \forall \mathfrak{x} \forall \mathfrak{y} \forall \mathfrak{u} \exists \mathfrak{y}' \exists \mathfrak{v} 
        \pi_3 (\text{F} \mathfrak{x}' \mathfrak{y}' \wedge (\text{F} \mathfrak{x} \mathfrak{y}) \rightarrow \alpha) \footnote{En esta sentencia
        suponemos que las sucesiones de variables $\mathfrak{x}, \mathfrak{x}', \mathfrak{y}, \mathfrak{y}', \mathfrak{u}, \mathfrak{v}$ son disjuntas
        entre sí.} % esta llave sale en rojo y no se por que, pero no da error
    \end{equation}

    Aplicando ahora dos veces el \autoref{lem:lema-4} junto con el \autoref{lem:lema-6}, obtenemos la deducibilidad de 
    \begin{equation} \label{eq:IV-1}
        \beta \leftrightarrow \gamma
    \end{equation}

    Además, es claro que la fórmula 
    \begin{equation} \label{eq:IV-2}
        \beta \rightarrow \pi_1 \alpha
    \end{equation}
    es válida. Ahora bien, $\gamma$ tiene grado $n$, y por tanto es por hipótesis o satisfacible o refutable. Si $\gamma$ es satisfacible, entonces
    también lo es $\pi_1 \alpha$ (por \eqref{eq:IV-1} y \eqref{eq:IV-2}). Si en caso contrario $\gamma$ es refutable, entonces también lo es 
    $\beta$ (por \eqref{eq:IV-1}), es decir, entonces $\lnot \beta$ es deducible. Sustituyendo ahora F por $\pi_2 \alpha$ en $\lnot \beta$, obtenemos
    que en este caso la siguiente sentencia es deducible:
    \begin{equation}
        \lnot (\forall \mathfrak{x}' \exists \mathfrak{y}' \pi_2 \alpha \wedge 
        \forall \mathfrak{x} \forall \mathfrak{y} (\pi_2 \alpha \rightarrow \pi_2 \alpha))
    \end{equation}

    Se puede observar que, naturalmente, 
    \begin{equation}
        \forall \mathfrak{x} \forall \mathfrak{y} (\pi_2 \alpha \rightarrow \pi_2 \alpha)
    \end{equation}
    es deducible, y por ello también lo es $\lnot \forall \mathfrak{x}' \exists \mathfrak{y}' \pi_2 \alpha$, es decir, en este caso $\pi_2 \alpha$ es refutable.
    Por tanto, de hecho $\pi_2 \alpha$ es o refutable o satisfacible.
\end{proof}

Ahora, para acabar de probar la \autoref{prop:satisf-refut}, sólo necesitamos probar el siguiente resultado:

\begin{teorema} \label{thm:teoremaV}
    Cada $K$-fórmula de primer grado es o satisfacible o refutable.
\end{teorema}

La demostración de este teorema requiere algunas definiciones previas, así como algunos resultados derivados de las definiciones que vamos a establecer, por lo que
se deja la presente demostración para más adelante.

Sea $\forall \mathfrak{x} \exists \mathfrak{y} \alpha (\mathfrak{x} ; \mathfrak{y} )$ $-$abreviado como $\pi\alpha$$-$ una fórmula cualquiera de primer grado.
Mediante $\mathfrak{x}$ representamos un r-tuplo de variables, con $\mathfrak{y}$ un s-tuplo. COnsideremos los r-tuplos sacados de la sucesión $\text{x}_0, \text{x}_1,
\text{x}_2,\dots \text{x}_i \dots$ ordenados por la suma creciente de sus índices en una sucesión:
\begin{equation}
    \mathfrak{x}_1 = (\text{x}_0, \text{x}_0, \dots, \text{x}_0), \quad \mathfrak{x}_2 = (\text{x}_1, \text{x}_0, \dots, \text{x}_0), \quad 
    \mathfrak{x}_3 = (\text{x}_0, \text{x}_1, \text{x}_0, \dots, \text{x}_0) \quad \dots
\end{equation}
y definamos una sucesión $\{\alpha_n\}$ de fórmulas derivadas a partir de $\pi\alpha$ del siguiente modo:
\begin{equation}
    \begin{aligned}
        &\alpha_1 = \alpha(\mathfrak{x}_1 ; \text{x}_1, \text{x}_2, \dots, \text{x}_s) \\
        &\alpha_2 = \alpha(\mathfrak{x}_2 ; \text{x}_{s+1}, \text{x}_{s+2}, \dots, \text{x}_{2s}) \wedge \alpha_1 \\
        &\cdots \\ 
        &\alpha_n = \alpha(\mathfrak{x}_n ; \text{x}_{(n-1)s+1}, \text{x}_{(n-1)s+2}, \dots, \text{x}_{ns}) \wedge \alpha_{n-1}    
    \end{aligned}
\end{equation}

Designemos mediante $\mathfrak{y}_n$ el s-tuplo $\text{x}_{(n-1)s+1}, \text{x}_{(n-1)s+2}, \dots, \text{x}_{ns}$ de tal modo que:
\begin{equation} \label{eq:def-V}
    \alpha_n = \alpha(\mathfrak{x}_n ; \mathfrak{y}_n) \wedge \alpha_{n-1}
\end{equation}

Además, definimos $\pi_n\alpha_n$ estableciendo como sigue:
\begin{equation}
    \pi_n\alpha_n = \exists \text{x}_0 \exists \text{x}_1 \cdots \exists \text{x}_{ns} \alpha_n
\end{equation}

Como fácilmente se comprueba, en $\alpha_n$ aparecen precisamente las variables desde $\text{x}_0$ hasta $\text{x}_{ns}$, que están también ligadas por el prefijo $\pi_n$. 
Además, es evidente que las variables del r-tuplo $\mathfrak{x}_{n+1}$ ya aparecen en $\pi_n$ (y por tanto son distintas de las que aparecen en $\mathfrak{y}_{n+1}$).
Designemos mediante $\pi_n'$ lo que queda de $\pi_n$ cuando suprimimos las variables de r-tuplo $\mathfrak{x}_{n+1}$. Si nos olvidamos del orden de aparición de las variables,
tenemos que $\exists \mathfrak{x}_{n+1} \pi_n' = \pi_n$.

Supuestas todas estas notaciones previas, tenemos como resultado directo el teorema siguiente:

\begin{teorema} \label{thm:teoremaVI}
    Para cada $n$ es deducible $\pi\alpha \rightarrow \pi_n\alpha_n$.
\end{teorema}
\begin{proof}
    Probaremos el teorema mediante inducción.
    \begin{itemize}
        \item Para $n = 1$ tenemos que $\pi\alpha \rightarrow \pi_1\alpha_1$ es deducible, ya que, por el \autoref{lem:lema-3} y la cuarta regla de inferencia, tenemos:
                \begin{equation}
                    \forall \mathfrak{x} \exists \mathfrak{y} \alpha (\mathfrak{x} ; \mathfrak{y} ) \rightarrow 
                    \forall \mathfrak{x}_1 \exists \mathfrak{y}_1 \alpha(\mathfrak{x}_1 ; \mathfrak{y}_1 )
                \end{equation}
                y además, por el \autoref{lem:lema-1} tenemos: 
                \begin{equation}
                    \forall \mathfrak{x}_1 \exists \mathfrak{y}_1 \alpha(\mathfrak{x}_1 ; \mathfrak{y}_1 ) \rightarrow  
                    \exists \mathfrak{x}_1 \exists \mathfrak{y}_1 \alpha(\mathfrak{x}_1 ; \mathfrak{y}_1 )
                \end{equation}
        \item Para un $n$ arbitrario tenemos que $\pi\alpha \wedge \pi_n\alpha_n \rightarrow \pi_{n+1}\alpha_{n+1}$ es deducible ya que, al igual que antes, 
                aplicando el \autoref{lem:lema-3} y la cuarta regla de inferencia, tenemos:
                \begin{equation} \label{eq:VI-1}
                    \forall \mathfrak{x} \exists \mathfrak{y} \alpha (\mathfrak{x} ; \mathfrak{y} ) \rightarrow 
                    \forall \mathfrak{x}_{n+1} \exists \mathfrak{y}_{n+1} \alpha(\mathfrak{x}_{n+1} ; \mathfrak{y}_{n+1} )
                \end{equation}
                y además, por el \autoref{lem:lema-2} tenemos:
                \begin{equation} \label{eq:VI-2}
                    \pi_n\alpha_n \rightarrow \exists \mathfrak{x}_{n+1} \pi_n'\alpha_n
                \end{equation}
                Ahora, aplicamos el \autoref{lem:lema-1} y sustituimos F por $\exists \mathfrak{x}_{n+1} \alpha(\mathfrak{x}_{n+1} ; \mathfrak{y}_{n+1} )$
                y G por $\pi'\alpha_{n}$, con lo que obtenemos:
                \begin{equation} \label{eq:VI-3}
                    \forall \mathfrak{x}_{n+1} \exists \mathfrak{y}_{n+1} \alpha (\mathfrak{x}_{n+1} ; \mathfrak{y}_{n+1} ) \wedge \exists \mathfrak{x}_{n+1} \pi_n'\alpha_n
                    \rightarrow  \exists \mathfrak{x}_{n+1} (\exists \mathfrak{y}_{n+1} \alpha (\mathfrak{x}_{n+1} ; \mathfrak{y}_{n+1} ) \wedge \pi_n'\alpha_n)       
                \end{equation} 

                Fijándonos ahora en que el antecedente del condicional \eqref{eq:VI-3} es la conyunción de los consiguientes de \eqref{eq:VI-1} y \eqref{eq:VI-2},
                obtenemos que es deducible:
                \begin{equation} \label{eq:VI-4}
                    \pi\alpha \wedge \pi_n\alpha_n \rightarrow \exists \mathfrak{x}_{n+1} (\exists \mathfrak{y}_{n+1} 
                    \alpha (\mathfrak{x}_{n+1} ; \mathfrak{y}_{n+1} ) \wedge \pi_n'\alpha_n)
                \end{equation}
                Por otro lado, de \eqref{eq:VI-1} y de los lemas 2, 4 y 6 obtenemos la deducibilidad de:
                \begin{equation} \label{eq:VI-5}
                    \exists \mathfrak{x}_{n+1} (\exists \mathfrak{y}_{n+1} \alpha (\mathfrak{x}_{n+1} ; \mathfrak{y}_{n+1} ) \wedge \pi_n'\alpha_n)
                    \leftrightarrow \pi_{n+1}\alpha_{n+1}
                \end{equation}
                Y gracias a \eqref{eq:VI-4} y \eqref{eq:VI-5} obtenemos la inducción, y por tanto la prueba del teorema.
    \end{itemize}

    Con esto tenemos probado el \autoref{thm:teoremaVI}, que como consecuencia queda probado el \autoref{thm:teoremaV}, que junto con resultados anteriores hemos conseguido 
    probar la \autoref{prop:satisf-refut}. Como vimos anteriormente, esta proposición era el resultado que nos faltaba para acabar la demostración del \autoref{thm:teoremaII}, 
    con lo que hemos probado la tesis de este apartado. Es decir, hemos dado una demostración de que toda fórmula válida de primer orden es deducible.
\end{proof}


\section{Corolarios}




%%%%%%%%%%%%%%%%%%%%%%%%%%%%%%%%%%%%%
\chapter{Sobre sentencias formalmente indecidibles}
\dictum[John von Neumann]{El logro de Gödel en la lógica moderna es singular y monumental - más que monumental, es una señal que 
permanecerá visible lejos en el espacio y en el tiempo.\footnote{Palabras de von Neumann en la entrega a Gödel del premio Einstein 
en 1951, recogidas en \textit{New York Times}, 15 de marzo de 1951, pág. 31.}}


\section{Definiciones y conceptos previos}
Tendremos como objetivo principal probar la existencia de sentencias indecidibles para un sistema formal $P$. Dicho sistema $P$ es
esencialmente el sistema que se obtiene cuando a los axiomas de peano se les añade la lógica de \textit{Principia Mathematica} ($PM$ de aquí en adelante).
\footnote{El hecho de de los axiomas de Peano, así como todas las otras modificaciones del sistema $PM$ introducidas en toda la demostración, sólo 
tienen como finalidad simplificar la prueba, y por ellos son prescindibles.}

Los signos primitivos del sistema $P$ son los siguientes:
\begin{enumerate}
    \item Constantes: « $\sim$ » (no), « $\vee$ » (o), « $\Pi$ » (para todo), « $0$ »(cero), « s » (el siguiente de), «()» (paréntesis).
    \item Variables tipo 1 (para individuos\footnote{Cuando tratamos de individuos hacemos referencia al conjunto de los números naturales.}, 
            incluyendo el 0):  «$\text{x}_1$», «$\text{y}_1$», «$\text{z}_1$», \dots \\
          Variables tipo 2 (para clases de individuos): «$\text{x}_2$», «$\text{y}_2$», «$\text{z}_2$», \dots \\
          Variables tipo 3 (para clases de clases de individuos): «$\text{x}_3$», «$\text{y}_3$», «$\text{z}_3$», \dots \\
          \dots \\
          etc., para cada número natural como tipo.\footnote{Suponemos que disponemos de una cantidad infinita numerable de signos para cada tipo de variables.}
\end{enumerate}
\begin{itemize}
    \item[Observación:] No necesitamos disponer de variables para relaciones binarias o $n$-arias ($n>2$) como signos primitivos, ya que podemos definir
                        las relaciones como clase de pares ordenados y los pares ordenados a su vez como clases de clases. Por ejemplo, podemos considerar
                        el par ordenado $<a,b>$ como $\{\{a\}, \{a.b\}\}$, donde $\{x,y\}$ denota la clase cuyos únicos elementos son $x$ e $y$, y $\{x\}$ 
                        la clase cuyo único elemento es $x$.\footnote{Las relaciones no homogéneas también pueden definirse de esta manera; por ejemplo, una
                        relación entre individuos y clases puede definirse como una clase de elementos de la forma $\{\{x_2\}, \{\{x_1\},x_2\}\}$. Todos los 
                        teoremas deducibles en $PM$ son también deducibles cuando se los reformula de esta manera.}
\end{itemize}
                           
Llamaremos \textit{signo de primer tipo} a una combinación de signos que tenga una de las siguientes formas:
\begin{equation}
    a,\thinspace sa,\thinspace ssa,\thinspace sssa,\thinspace \dots,\thinspace \text{etc.,}
\end{equation}
Donde $a$ es $0$ ó es variable de tipo 1. En el primer caso llamamos a tal signo un numeral. Para $n>1$ entendemos por \textit{signo de tipo n} lo mismo que por 
\textit{variable de tipo n}. Llamaremos \textit{fórmulas elementales} a las combianciones de signos de la forma $a(b)$, donde $b$ es un signo de tipo $n$, y $a$ 
es un signo de tipo $n+1$. Definimos la calse de las \textit{fórmulas} como la mínima clase que abarca todas las fórmulas elementales y que, siempre que contenga 
$\alpha$ y $\beta$, contiene también $\sim(\alpha), (\alpha)\vee (\beta)$ y $\Pi\text{x}(\alpha)$ (donde x es una variable cualquiera)
\footnote{Por tanto, $\Pi\text{x}(\alpha)$ es también una fórmula cuando x no aparece o no está libre en $\alpha$. Naturalmente, en ester caso $\Pi\text{x}(\alpha)$
significaría lo mismo que $\alpha$.}. Llamamos a $(\alpha)\vee (\beta)$ la \textit{disyunción} de $\alpha$ y $\beta$, a $\sim(\alpha)$ la \textit{negación} de $\alpha$
y a $\Pi\text{x}(\alpha)$ una \textit{generalización} de $\alpha$. Una \textit{sentencia} es una fórmula sin variables libres (donde la noción de variable libre se 
define del modo usual). A una fórmula con exactamente $n$ variables libres (y ninguna otra variable libre) la llamaremos \textit{signo relacional n-ario}; para $n=1$
lo llamaremos también \textit{signo de clase}.

Por $\daleth^b_v \alpha$ (donde $\alpha$ designa una fórmula, $v$ una variable y $b$ un signo del mismo tipo que $v$) entendemos la fórmula que resulta de reemplazar 
en $\alpha$ cada aparición libre de $v$ por $b$\footnote{Si $v$ no aparece libre en $\alpha$, entonces $\daleth^b_v \alpha = \alpha$. Nótese que $\daleth$ es un signo 
matemático.}. Decimos que una fórmula $\alpha$ es una \textit{elevación de tipo} de otra fórmula $\beta$ si $\alpha$ se obtiene a partir de $\beta$ mediante una elevación
por el mismo número de cada variable que aparece en $\beta$. \\

Las siguientes fórmulas (de I a V) se llaman \textit{axiomas} (están escritas con ayuda de abreviaturas: $\wedge, \supset, \equiv, \Sigma, =$\footnote{Como en $PM$, 
consideramos que $\text{x}_1 = \text{y}_1$ está definido por $\Pi \text{x}_2 (\text{x}_2(\text{x}_1) \supset \text{x}_2(\text{y}_1))$; de igual modo para los tipos 
superiores.}, definidas del modo usual y conforme a las conveciones habituales sobre la omisión de paréntesis)\footnote{Para obtener los axiomas a partir de los esquemas
indicados debemos (después de realizar las sustituciones permitidas en II, III y IV), además,
\begin{enumerate}
    \item[(1)] eliminar las abreviaturas
    \item[(2)] añadir los paréntesis omitidos.
\end{enumerate} 
Nótese que las expresiones así obtenidas deben ser "fórmulas" en el sentido arriba definido.}:

\begin{enumerate}
    \item[I.]   %\begin{equation}
                    %\begin{flalign}
                        $1. \quad \sim (\text{sx}_1 = 0)$ \\
                        $2. \quad \text{sx}_1 = \text{sy}_1 \supset \text{x}_1 = \text{y}_1$ \\
                        $3. \quad \text{x}_2 (0) \wedge \Pi \text{x}_1 \thinspace (\text{x}_2 \thinspace(\text{x}_1) \wedge \text{x}_2 (\text{sx}_1)) \supset \Pi \text{x}_1 \thinspace (\text{x}_2 \thinspace(\text{x}_1)) $   
                    %\end{flalign}
                %\end{equation}
    \item[II.] Cada fórmula que resulta de sustituir X, Y por cualesquiera fórmulas en los siguientes esquemas:
                \begin{flalign}
                    &1.\quad \text{X} \vee \text{X} \supset \text{X} \\
                    &2.\quad \text{X} \supset \text{X} \vee \text{Y} \\
                    &3.\quad \text{X} \vee \text{Y} \supset \text{Y} \vee \text{X} \\
                    &4.\quad (\text{X} \supset \text{Y}) \supset (\text{Z} \vee \text{X} \supset \text{Z} \vee \text{Y})
                \end{flalign}
    \item[III.] Cada fórmula que resulta de uno de estos dos esquemas:
                \begin{flalign}
                    &1.\quad \Pi v\alpha \supset \daleth_v^c \alpha \\ 
                    &2.\quad \Pi v(\beta \vee \alpha) \supset \beta \vee \Pi v(\alpha)
                \end{flalign}
                Cuando sustituimos $\alpha, v, \beta, c$ del siguiente modo (y realizamos la operación indicada por «$\daleth$» en $1.$):
                \begin{enumerate}
                    \item[] Sustituimos por $\alpha$ por una fórmula cualquiera, $v$ por una variable cualquiera, $\beta$ por una fórmula en la que no aparezca libre
                            $v$ y $c$ por un signo del mismo tipo que $v$, siempre que $c$ no contenga alguna variable que pase a estar ligada en un lugar de $\alpha$ 
                            donde $v$ estaba libre.\footnote{Por tanto, $c$ es o una variable o el $0$ o un signo de la forma $s\dots su$, donde $u$ es $0$ o una variable
                            de tipo 1. Respecto de la noción de estar (una variable) libre o ligada en un lugar de $\alpha$, véase \cite{v1927hilbertschen}.}
                \end{enumerate}
    \item[IV.]  Cada fórmula que resulta del esquema $$ \Sigma u \Pi v (u(v) \equiv \alpha) $$ cuando usstituimos $v$ por una varaible cualquiera de tipo $n$, sustituimos
                $u$ por una variable cualquiera de tipo $n+1$ y sustituimos $\alpha$ por una fórmula, en la que $u$ no esté libre. Este axioma desempeña el papel de axioma
                de reducibilidad (el axioma de comprensión de la teoría de conjuntos).
    \item[V.]   Cada fórmula que resulta de $$\Pi \text{x}_1 (\text{x}_2 (\text{x}_1) \equiv \text{y}_2(\text{x}_1)) \supset \text{x}_2 = \text{y}_2$$ por elevación de tipo
                (así como esta fórmula misma). Este axioma dice que una clase está completamente determinada por sus elementos.
\end{enumerate}

Una fórmula $\gamma$ se llama una \textit{inferencia inmediata} de $\alpha$ y $\beta$, si $\alpha$ es la fórmula $\sim \beta \vee \gamma$ (y $\gamma$ se llama una 
\textit{inferencia inmediata} de $\alpha$, si $\gamma$ es la fórmula $\Pi v \alpha $, donde $v$ designa una variable cualquiera). La clase de las \textit{fórmulas deducibles}
se define como la mínima clase de fórmulas que contiene los axiomas y está clausurada respecto a la relación de «inferencia inmediata»\footnote{La regla de sustitución 
resulta aquí supreflua, pues en los axiomas mismo ya tenemos realizadas todas las sustituciones posibles (véase \cite{v1927hilbertschen})}.

Ahora asignamos unívocamente números naturas a los signos primitivos del sistema $P$ del siguiente modo:
\begin{equation}
    \begin{aligned}
        &\text{«}0\text{»} \thinspace \dots \thinspace 1 \\
        &\text{«}s\text{»} \thinspace \dots \thinspace 3 \\
        &\text{«}\sim\text{»} \thinspace \dots \thinspace 5 \\
        &\text{«}\vee\text{»} \thinspace \dots \thinspace 7 \\
        &\text{«}\Pi\text{»} \thinspace \dots \thinspace 9 \\
        &\text{«}(\text{»} \thinspace \dots \thinspace 11 \\
        &\text{«})\text{»} \thinspace \dots \thinspace 13 \\    
    \end{aligned}
\end{equation}

A las variables de tipo $n$ asignamos los números de la forma $\rho^n$ (donde $\rho$ es un número primo $> 13$). Mediante esta asignación a cada fila finita de 
signos primitivos (y en especial a cada fórmula) corresponde biunívocamente una secuencia finita de números naturales. Ahora asignamos (de nuevo biunívocamente)
números naturales a las secuencias finitas de números naturales, haciendo corresponder a la secuencia $n_1, n_2, \dots, n_k$ el número 
$2^{n_1} \cdot 3^{n_2} \cdots \rho^{n_k}_k$ donde $\rho_k$ denota el $k$-avo número primo (en orden de magnitud creciente). Así asignamos biunívocamente un
número natural no sólo a cada signo primitivo, sino también a cada secuencia finita de signos primitivos. Mediante $nu(\text{a})$ denotamos
el número natural asignado al signo primitivo (o a la secuencia de signos primitivos) a. Supongamos que esté dada cierta clase o relación $n$-aria $R$ entre signos
primitivos. Le asignamos la clase o relación $n$-aria $R'$ entre números naturales, en la que están los números $x_1, x_2, \dots, x_n$ si y sólo si hay signos primitivos
o secuencias de signos primitivos $\text{a}_1, \text{a}_2, \dots, \text{a}_n$, tales que $x_i = nu(\text{a}_i)$ (para $i = 1, 2, \dots, n)$ y 
$\text{a}_1, \text{a}_2, \dots, \text{a}_n$ están en la relación $R$. Las clases y relaciones de números naturales, que corresponden de este modo a los conceptos
metamatemáticos hasta ahora definidos, como por ejemplo «variable», «fórmula», «sentencia», «axioma», «fórmula deducible», etc., serán designadas por las mismas 
palabras escritas con letras mayúsculas pequeñas. Por ejemplo, el enunciado de que en el sistema $P$ hay problemas indecidibles se convierte en la siguiente afirmación:
Hay SENTENCIAS $a$, tales que ni $a$ ni la NEGACIÓN de $a$ son FÓRMULAS DEDUCIBLES.

%%%%%%%%%%%%%%%%%%%%%%%%%%%%%%%%%%%%%%%%%%%%%%%%%%%%%%%%%%%
%%%%%%%%%%%%%%%%%%%%%%%%%%%%%%%%%%%%%%%%%%%%%%%%%%%%%%%%%%%
%%%%%%%%%%%%%%%%%%%%%%%%%%%%%%%%%%%%%%%%%%%%%%%%%%%%%%%%%%%
\section{La indecidibilidad}

Ahora vamos a insertar aquí una disgresión que por el momento no tiene nada que ver con el sistema formal $P$. Empecemos por dar la siguiente definición: 
Decimos que una función numérica\footnote{Es decir, su dominio definicional es la clase de los números naturales (o de los $n$-tuplos de números naturales) 
y sus valores son números naturales.} $f(x_1, x_2, \dots, x_n)$ está \textit{recursivamente definida a partir} de las funciones numéricas $h(x_1, x_2, \dots, x_{n-1})$
y $g(x_1, x_2, \dots, x_{n+1})$, si para cada $x_2, \dots, x_n$, $k$ vale lo siguiente:\footnote{En lo sucesivo las letras latinas minúsculas (a veces con subíndices) 
son siempre variables para números naturales (a no ser que se indique explícitamente lo contrario).}
\begin{equation} \label{eq:recdef}
    \begin{aligned}
        &f(0, x_2, \dots, x_n) = h(x_2, \dots, x_n) \\
        &f(k+1, x_2, \dots, x_n) = g(k, f(k, x_2, \dots, x_n), x_2, \dots, x_n) 
    \end{aligned}
\end{equation}

Una función numérica $f$ se llama \textit{recursiva primitiva} si hay una secuencia finita de funciones $f_1, f_2, \dots, f_n$, que acaba con $f$ y que tiene 
la propiedad de que cada función $f_k$ de la secuencia está recursivamente definida a partir de dos de las funciones precedentes o resulta de alguna de las 
funciones precedentes por sustitución\footnote{Más precisamente: por introducción de algunas de las funciones precedentes en los lugares argumentales de una 
de las funciones precedentes, por ejemplo, $f_k(x_1, x_2) = f_p(f_q(x_1, x_2), f_r(x_2))$, con $p, q, r < k$. No es necesario que todas las variables
del lado izquierdo aparezcan también en el derecho.}, o, finalmente, es una constante o la función del siguiente, $x+1$. La longitud de la mínima secuencia de $f_i$
corresponde a una función recursiva primitiva $f$ se llama su \textit{grado}. Una relación $n$-ara $R$ entre números naturales se llama recursiva 
primitiva\footnote{Incluimos las clases entre las relaciones (como relaciones monarias). Las relaciones recursivas primitivas tienen desde luego la propiedad 
de que para cada $n$-tuplo dado de números naturales $x_1, x_2, \dots, x_n$ \textit{se puede decidir si} $R$ $x_1, x_2, \dots, x_n$ o no.} si hay una función $n$-aria
recursiva primitiva $f$ tal que para cada $n$ números naturales $x_1, x_2, \dots, x_n$ 
\begin{equation}
    R x_1, x_2, \dots, x_n \leftrightarrow f(x_1, x_2, \dots, x_n) = 0 
    \footnote{Para todas las consideraciones intuitivas (y especialmente para las metamatemáticas) usamos el simbolismo de Hilbert definido en \cite{hilbert1962elementos}.}
\end{equation}

Los siguientes teoremas valen:

\begin{teorema} \label{teo:TeoremaI}
    Cada función (o relación) obtenida a partir de funciones (o relaciones) recursivas primitivas por sustitución de las variables por funciones recursivas 
    primitivas es recursiva primitiva; igualmente lo es cada función obtenida a partir de funciones recursivas primitivas por definición recursiva según el
    esquema de \eqref{eq:recdef}.
\end{teorema}

\begin{teorema} \label{teo:TeoremaII}
    Si $R$ y $S$ son relaciones recursivas primitivas, también lo son $\lnot R$ y $R \vee S$ (por tanto, también $R \wedge S$).
\end{teorema}

\begin{teorema} \label{teo:TeoremaIII}
    Si las funciones $f(\mathfrak{x})$, $h(\mathfrak{y})$ son recursivas primitivas, entonces también lo es la relación 
    $f(\mathfrak{x}) = h(\mathfrak{y})$\footnote{Usamos las letras alemanas $\mathfrak{x}, \mathfrak{y}$ como abreviaturas para cualesquiera $n$-tuplos de variables, 
    como por ejemplo $x_1, x_2, \dots, x_n$.}.
\end{teorema}

\begin{teorema} \label{teo:TeoremaIV}
    Si la función $f(\mathfrak{x})$ y la relación $Rx, \mathfrak{y}$ son recursivas primitivas, también lo son las relaciones $S$ y $T$ definidas por 
    \begin{equation}
        \begin{aligned}
            &S(\mathfrak{x}, \mathfrak{y}) \leftrightarrow \exists x (x \leq f(\mathfrak{x}) \wedge Rx, \mathfrak{y} ) \\
            &T(\mathfrak{x}, \mathfrak{y}) \leftrightarrow \forall x (x \leq h(\mathfrak{x}) \wedge Rx, \mathfrak{y} )
        \end{aligned}
    \end{equation}
    así como la función
    \begin{equation}
        q(\mathfrak{x}, \mathfrak{y}) = \mu x (x \leq f(\mathfrak{x}) \wedge Rx, \mathfrak{y}),
    \end{equation}
    donde $\mu x \varphi(x)$ significa el mínimo número $x$, para el que vale $\varphi(x)$, si hay algún tal, y $0$, si no lo hay.
\end{teorema}

El \autoref{teo:TeoremaI} se sigue inmediatamente de la definición de «recursivo primitivo». Tanto el \autoref{teo:TeoremaII} como el \autoref{teo:TeoremaIII}
se basan en que las funciones numéricas
\begin{equation}
    ne(x), \\
    di(x,y), \\
    id(y,x)
\end{equation}
correspondientes a las nociones lógicas $\lnot, \vee, =$, a saber
\begin{flalign}
    &ne(x) = 1; \quad ne(x) = 0 \text{ para } x \neq 0 \\
    &di(0,x) = di(x,0) = 0; \quad di(x,y) = 1 \text{ si } x \neq 0, y \neq 0 \\
    &id(x,y) = 0 \text{ si } x=y; \quad id(x,y) = 1 \text{ si } x \neq y 
\end{flalign}

son recursivas primitivas, como fácilmente se comprueba. He aquí la prueba resumida del \autoref{teo:TeoremaIV}:
\begin{proof}
    Por hipótesis hay una función recursiva primitiva $r(x, \mathfrak{y})$, tal que
    \begin{equation}
        R x, \mathfrak{y} \leftrightarrow r(x, \mathfrak{y}) = 0.
    \end{equation}

    Definamos ahroa mediante el esquema de recursión \eqref{eq:recdef} una función $j(x, \mathfrak{y})$ del siguiente modo:
    \begin{equation}
        \begin{aligned}
            &j(0, \mathfrak{y}) = 0 \\
            &j(n+1, \mathfrak{y}) = (n+1) \cdot a + j(n, \mathfrak{y}) \cdot ne(a) 
            \footnote{Presuponemos como ya es sabido que las funciones $x+y$ (adición) y $x\cdot y$ (multiplicación) son recursivas primitivas.}
        \end{aligned}
    \end{equation}
    donde $a = ne(ne(r(0, \mathfrak{y}))) \cdot ne(r(n+1, \mathfrak{y})) \cdot ne(j(n, \mathfrak{y}))$.

    Por tanto, $j(n+1, \mathfrak{y})$ es igual a $n+1$ (si $a=1$) o es igual a $j(n, \mathfrak{y})$ (si $a=0$)\footnote{$a$ no puede tomar otros valores que
    $0$ y $1$, como se sigue en la definición de $ne$.}. Evidentemente, el primer caso ocurre si y sólo si todos los factores de $a$ son $1$, es decir, si ocurre que
    \begin{equation}
        \lnot R \thinspace 0, \mathfrak{y} \wedge R \thinspace n+1, \mathfrak{y} \wedge j(n, \mathfrak{y}) = 0
    \end{equation}

    De aquí se sigue que la función $j(n, \mathfrak{y})$, considerada como función de $n$, da siempre $0$ hasta (pero no incluyendo) el mínimo valor de $n$
    para el que ocurre $R \thinspace n, \mathfrak{y}$, y a partir de ahí siempre da ese valor. (Por consiguiente, si ya ocurre que $R \thinspace 0, \mathfrak{y},
    j(n, \mathfrak{y})$ es constante e igual a $0$). Por tanto, ocurre que
    \begin{flalign}
        &q(\mathfrak{x}, \mathfrak{y}) = j(f(\mathfrak{x}), \mathfrak{y}) \\
        &S \thinspace \mathfrak{x}, \mathfrak{y} \leftrightarrow R \thinspace q(\mathfrak{x}, \mathfrak{y}), \mathfrak{y} 
    \end{flalign}

    La relación $T$ puede ser reducida, por negación, a un caso análogo al de $S$. Con esto queda probado el \autoref{teo:TeoremaIV}.
\end{proof}

Las funciones $x+y, x\cdot y, x^v$, así como las relaciones $x<y$ y $x=y$ son recursivas primitivas, como fácilmente se comprueba. Partiendo de estos conceptos, 
vamos a definir u n a secuencia de funciones (o relaciones) $1$-$45$, cada una de las cuales se define a partir de las precedentes mediante los procedimientos indicados
en los cuatro teoremas anteriores. En la mayor parte de estas definiciones condensamos en un solo paso varios de los pasos permitidos por estos cuatro teoremas. 
Por tanto, cada una de las funciones (o relaciones) $1$-$45$, entre las que se encuentran, por ejemplo, los conceptos «FÓRMULA», «AXIOMA» e «INFERENCIA INMEDIATA»,
es recursiva primitiva.

\begin{enumerate}
    \item $x/y \leftrightarrow \exists z (z \leq x \wedge x = y \cdot z)$\footnote{Después del definiendum, el signo $=$ se usa en el sentido de «igualdad por
            definición»; el signo $\leftrightarrow$, en el de «equivalencia por definición» (por lo demás, el simbolismo es el de Hilbert).}\\
            $x$ es divisible por $y$.\footnote{Cada vez que en las definiciones siguientes aparece uno de los signos $\forall x, \exists x, \mu x$, éste está 
            seguido de una acotación de $x$. Esta acotación sirve meramente para asegurar que la noción definida es recursiva primitiva (véase \autoref{teo:TeoremaIV}).
            La extensión de la noción definida, por el contrario, no cambiaría en la mayor parte de los casos, aunque dejásemos de lado dicha acotación.}
    \item $\text{Prim } x \leftrightarrow \lnot \exists z (z \leq x \wedge z \neq 1 \wedge z \neq x \wedge x/z) \wedge x>1$\\ $x$ es un número primo.
    \item $0 \thinspace Pr \thinspace x = 0$\\ 
            $(n+1) \thinspace Pr \thinspace x = \mu y (y \leq x \wedge \text{Prim } y \wedge x/y \wedge y>n \thinspace Pr \thinspace x)$ \\
            $n \thinspace Pr \thinspace x$ es el $n$-avo número primo (por orden de magnitud creciente) contenido en $x$\footnote{Para $0 < n \leq z$, donde
            $z$ es el número de diferentes factores primos de $x$. Obsérvese que $n \thinspace Pr \thinspace x = 0$ para $n = z+1$.}.
    \item $0! = 1$ \\ $(n+1)! = (n+1) \cdot n!$
    \item $Pr(0) = 0$ \\ $Pr(n+1) = \mu y (y \leq (Pr(n))! + 1 \wedge \text{Prim } y \wedge y > Pr(n))$ \\ $Pr(n)$ es el $n$-avo número primo (por orden de 
            magnitud creciente)
    \item $n \thinspace Gl \thinspace x = \mu y (y \leq x \wedge x/(n \thinspace Pr \thinspace x)^y \wedge \lnot x/(n \thinspace Pr \thinspace x)^{y+1})$ \\ 
            $n \thinspace Gl \thinspace x$ es el miembro $n$-avo de la secuencia numérica correspondiente al número $x$ (para $n>0$ y $n$ no mayor que la 
            longitud de esa secuencia).
    \item $l(x) = \mu y (y \leq x \wedge y \thinspace Pr \thinspace x > 0 \wedge (y+1) \thinspace Pr \thinspace x = 0)$ \\ $l(x)$ es la longitud de la secuencia 
            numérica correspondiente a $x$.
    \item   $\begin{aligned}
                x * y = \mu z (&z \leq (Pr(l(x) + l(y)))^{x+y} \wedge \forall n (n \leq l(x) \rightarrow n \thinspace Gl \thinspace z = n \thinspace Gl \thinspace x) \wedge \\ 
                &\wedge \forall n (0 < n \leq l(y) \rightarrow (n + l(x)) \thinspace Gl \thinspace z = n \thinspace Gl \thinspace y))
            \end{aligned}$ \\
            $x*y$ corresponde a la operación de concatenación de dos secuencias finitas de números.
    \item $R(x) = 2^x$ \\ $R(x)$ corresponde a la secuencia numérica que sólo consta del número $x$ (para $x>0$).
    \item $E(x) = R(11) * x * R(13)$ \\ $E(x)$ corresponde a la operación de poner entre paréntesis (11 y 13 son los números asignados a los 
            signos primitivos «(» y «)» ).
    \item $n text{ Var } x \leftrightarrow \exists z (13 < z \leq x \wedge text{Prim}(z) \wedge x = z^n) \wedge n \neq 0$ \\ $x$ es una VARIABLE DE TIPO $n$.
    \item $text{ Var } x \leftrightarrow \exists n (n \leq x \wedge n text{ Var } x )$ \\ $x$ es una VARIABLE.
    \item $text{Neg}(x) = R(5) * E(x)$ \\ $text{Neg}(x)$ es la NEGACIÓN de $x$.
    \item $x \text{ Dis } y = E(x) * R(7) * E(y)$ \\ $x \text{ Dis } y$ es la DISYUNCIÓN de $x$ e $y$.
    \item $x \text{ Gen } y = R(9) * R(x) * E(y)$ \\ $x \text{ Gen } y$ es la GENERALIZACIÓN de $y$ respecto a la variable $x$ (suponiendo que $x$ sea una variable).
    \item $0 \thinspace N \thinspace x = x$ \\ $(n+1) \thinspace N \thinspace x = R(3) * n \thinspace N \thinspace x$ \\ 
            $n \thinspace N \thinspace x$ corresponde a la operación de poner $n$ veces el signo $s$ delante de $x$.
    \item $Z(n) = n \thinspace N \thinspace (R(1))$ \\ $Z(n)$ es el NUMERAL que designa el número $n$.
    \item $\text{Typ}_1' \thinspace x \leftrightarrow \exists m \thinspace n(m,n \leq x \wedge (m=1 \vee 1 \text{ Var } m) 
            \wedge x = n \thinspace N \thinspace (R(m)))$\footnote{$m,n \leq x$ es una abreviatura de $m \leq x \wedge n \leq x$ (y lo mismo para 
            más de dos variables)} \\ $x$ es un SIGNO DE TIPO $1$.
    \item $\text{Typ}_n \thinspace x \leftrightarrow (n=1 \wedge \text{Typ}_1' \thinspace (x)) \wedge (n>1 \wedge \exists v (v \leq x \wedge n \text{ Var } v \wedge x = R(v)))$ \\
            $x$ es un SIGNO DE TIPO $n$.
    \item $Elf \thinspace x \leftrightarrow \exists yzn (y, z, n \leq x \wedge \text{Typ}_n(y) \wedge \text{Typ}_{n+1}(z) \wedge x = x* E(y))$ \\ $x$ es una FÓRMULA ELEMENTAL.
    \item $Op \thinspace x \thinspace y \thinspace z \leftrightarrow x = \text{Neg}(y) \wedge x = y \text{ Dis } z \vee \exists v (v \leq x \wedge \text{Var } v \wedge 
            x = v \text{ Gen } y)$ 
    \item $\begin{aligned}
            FR \thinspace x \leftrightarrow \forall n (0 < n \leq l(x) \rightarrow Elf(n \thinspace Gl \thinspace x) &\vee \exists pq (0 < p,q < n \wedge \\
            \wedge &Op(n \thinspace Gl \thinspace x, p \thinspace Gl \thinspace x,q \thinspace Gl \thinspace x))) \wedge l(x) > 0
        \end{aligned}$ \\
            $x$ es una SECUENCIA DE FÓRMULAS, cada una de las cuales es o una FÓRMULA ELEMENTAL o se obtiene de las precedentes mediante las operaciones de 
            NEGACIÓN, DISYUNCIÓN o GENERALIZACIÓN.
    \item $\text{Form } x \leftrightarrow \exists n (n \leq (Pr(l(x)^2))^{x \cdot (l(x)^2)} \wedge FR \thinspace n \wedge x = (l(x)) \thinspace Gl \thinspace n)$\footnote{La
            acotación $n \leq (Pr(l(x)^2))^{x \cdot (l(x)^2)}$ puede comprobarse así: La longitud de la más corta secuencia de  fórmulas correspondientes a $x$ puede ser a lo 
            sumo igual al número de subfórmulas de $x$. Pero hay a lo sumo $l(x)$ subfórmulas de longitud $1$, a lo sumo $l(x) -1$ de longitud $2$, \dots, por tanto en conjunto
            a lo sumo $\frac{l(x) \cdot (l(x) + 1)}{2} \leq n \leq (l(x))^2$. Por tanto podemos suponer que todos los factores primos de $n$ son menores que $Pr((l(x))^2)$, 
            que su número es $\leq (l(x))^2$ y que sus exponentes (que son subfórmulas de $x$) son $\leq x$.} \\ 
            $x$ es una FÓRMULA (es decir, el último miembro de una SECUENCIA DE FÓRMULAS $n$).
    \item $v \thinspace \text{Geb} \thinspace n, x \leftrightarrow \text{Var } v \wedge \text{Form } x \wedge \exists a \thinspace b \thinspace c \thinspace 
            (a, b, c \leq x \wedge x = a*(v \text{ Gen } b)* c \wedge \text{Form } x \wedge l(a) + 1 \leq n \leq l(a) + l(v \text{ Gen } b))$ \\ La VARIABLE $v$
            está LIGADA en $x$ en el $n$-avo lugar.
    \item $v \thinspace Fr \thinspace n, x \leftrightarrow \text{Var } v \wedge \text{Form } x \wedge v = n \thinspace Gl \thinspace x \wedge n \leq l(x) 
            \wedge \lnot v \text{ Geb } n, x$ \\ La VARIABLE $v$ está LIBRE en $x$  en el $n$-avo lugar.
    \item $v \thinspace Fr \thinspace x \leftrightarrow \exists n (n \leq l(x) \wedge v \thinspace Fr \thinspace n, x)$ \\ $v$ aparece como VARIABLE LIBRE EN $x$.
    \item $Su \thinspace x\tbinom{n}{y} = \mu z (z \leq (Pr(l(x) + l(y)))^{x+y} \wedge \exists u \thinspace v(u,v \leq x \wedge x = u * R(n \thinspace Gl \thinspace x) * v \wedge \\ 
            \wedge z =  u * y * v \wedge n = l(u) + 1))$ \\
            $Su \thinspace x\tbinom{n}{y}$ se obtiene a partir de $x$ cuando sustituimos el $n$-avo miembro de $x$ por $y$ (suponiendo que $0 < n \leq (x)$).
    \item $\qquad \quad 0 \thinspace St \thinspace v, x = \mu n (n \leq l(x) \wedge v \thinspace Fr \thinspace n, x \wedge \lnot \exists p(n < p \leq l(x) \wedge v \thinspace Fr \thinspace p, x))$\\
            $(k+1) \thinspace St \thinspace v, x = \mu n (n < k \thinspace St \thinspace v, x \wedge v \thinspace Fr \thinspace n, x \wedge \lnot \exists 
            p(n < p \leq k \thinspace St \thinspace v, x \wedge v \thinspace Fr \thinspace p, x))$\\
            $k \thinspace St \thinspace v, x$ es el $k+1$-avo lugar de $x$ (contado a partir del extremo derecho de la FÓRMULA $x$), en el que $v$ aparece LIBRE en $x$ 
            (y es $0$ si no hay tal lugar).
    \item $A(v,x) = \mu n (n \leq l(x) \wedge n \thinspace St \thinspace v, x = 0)$ \\ $A(v,x)$ es el número de lugares en que $v$ aparece LIBRE en $x$.
    \item $Sb_0 (x^v_y) = x$ \\ $Sb_{k+1} (x^v_y) = Su \thinspace (Sb_k (x^v_y)) \tbinom{\text{\textit{k St v,x}}}{y}$
    \item $Sb (x^v_y) = Sb_{A(v,x)} (x^v_y)$\footnote{Si $v$ no es una VARIABLE  o $x$ no es una FÓRMULA, enotnces $Sb (x^v_y) = x$.}\\
            $Sb (x^v_y)$ es la noción anteriormente definida de $\daleth^b_v \alpha$\footnote{En vez de $Sb(Sb(x^v_w)^w_z)$ escribimos $Sb \thinspace x^{v \thinspace w}_{y \thinspace z}$, 
            (y análogamente para más de dos VARIABLES).}.
    \item $x \text{ Imp } y = (\text{Neg}(x)) \text{ Dis } y \\ x \text{ Con } y = \text{Neg } ((\text{Neg}(x)) \text{ Dis } (\text{Neg}(y))) \\
            x \text{ Aeq } y = (x \text{ Imp } y) \text{ Cong}(y \text{ Imp } x) \\ v \text{ Ex } y = \text{Neg } (v \text{ Gen} (\text{Neg}(y))).$
    \item $ n \thinspace Th \thinspace x = \mu y (y \leq x^{(x^n)} \wedge \forall k (k \leq l(x) \rightarrow (k \thinspace Gl \thinspace x > 13 
            \wedge k \thinspace Gl \thinspace y = k \thinspace Gl \thinspace x \cdot (1 \thinspace Pr(k \thinspace Gl \thinspace x))^n)))$ \\
            $ n \thinspace Th \thinspace x $ es la $n$-ava ELEVACIÓN DE TIPO de $x$ (suponiendo que $x$ y $ n \thinspace Th \thinspace x $ sean fórmulas).
            \begin{itemize}
                \item[Nota. ] A los axiomas I,$1$-$3$, corresponden tres números determinados, que designamos mediante $z_1$, $z_2$ y $z_3$. 
                                Procedemos a definir los siguientes:
            \end{itemize}
    \item $Z - Ax \thinspace x \leftrightarrow x = z_1 \vee x = z_2 \vee x = z_3$
    \item $A_1 - Ax \thinspace \leftrightarrow \exists y (y \leq x \wedge \text{Form } y \wedge x = (y \text{ Dis } y) \text{ Imp } y)$ \\
            $x$ es una FÓRMULA que se obtiene a partir del esquema axiomático II,$1$ por sustitución. De modo análogo se definen $A_2-Ax$, 
            $A_3-Ax$ y $A_4-Ax$, correspondientes a los axiomas II,$2$-$4$. 
    \item $A-Ax \thinspace x \leftrightarrow A_1-Ax \thinspace x \vee A_2-Ax \thinspace x \vee A_3-Ax \thinspace x \vee A_4-Ax \thinspace x$\\
            $x$ es una FÓRMULA que se obtiene por sustitución a partir de un esquema axiomático conectivo. 
    \item $Q \thinspace z, y, v \leftrightarrow \lnot \exists n \thinspace m \thinspace w \thinspace (n \leq l(y) \wedge m \leq l(z) \wedge 
            w \leq z \wedge w = m \thinspace Gl \thinspace z \wedge w \text{ Geb } n,y \wedge v \thinspace Fr \thinspace n, y)$ \\
            $z$ no contiene VARIABLE alguna que esté LIGADA en $y$ en un lugar, en el cual $v$ esté LIBRE.
    \item $L_1 - Ax \thinspace x \leftrightarrow \exists v\thinspace y \thinspace z \thinspace n (v,y,z,n \leq x \wedge n \text{ Var } v 
            \wedge \text{Typ}_n z \wedge \text{Form } y \wedge Q \thinspace z,y,v \wedge \\ \wedge x = (v \text{ Gen } y) \text{ Imp } (Sb(y_z^v)))$\\ 
            $x$ es una FÓRMULA que se obtiene por sustitución a partir del esquema axiomático III,$1$.
    \item $L_2 - Ax \thinspace x \leftrightarrow \exists v\thinspace q \thinspace p \thinspace (v,p,q \leq x \wedge n \text{ Var } v 
            \wedge \text{Form } p \wedge \lnot v \thinspace Fr \thinspace p \wedge \text{Form } q \wedge \\
            \wedge x = (v \text{ Gen } (p \text{ Dis } q)) \text{ Imp } (p \text{ Dis } (v \text{ Gen } q)))$\\ 
            $x$ es una FÓRMULA que se obtiene por sustitución a partir del esquema axiomático III,$2$.
    \item $R-Ax \thinspace x \leftrightarrow \exists u \thinspace v \thinspace y \thinspace n \thinspace (u,v,y,n \leq x \wedge 
            n \text{ Var } v \wedge (n+1) \text{ Var } u \wedge \lnot u \thinspace Fr \thinspace y \wedge \text{Form } y \wedge 
            x = u \text{ Ex} (v \text{ Gen } ((R(u) * E(R(v))) \text{ Aeq } y)))$ \\ 
            $x$ es una FÓRMULA que se obtiene por sustitución a partir del esquema axiomático IV,$1$.
            \begin{itemize}
                \item[Nota. ] Al axioma V,$1$ le corresponde un número determinado $z_4$. Así, procedemos a definir:
            \end{itemize}
    \item $M-Ax \thinspace x \leftrightarrow \exists n(n\leq x \wedge x = n \thinspace Th \thinspace z_4)$
    \item $Ax \thinspace x \leftrightarrow Z-Ax \thinspace x \vee L_1-Ax \thinspace x \vee L_2-Ax \thinspace x \vee R-Ax \thinspace x \vee M-Ax \thinspace x$\\ 
            $x$ es un AXIOMA.
    \item $Fl \thinspace x \thinspace y \thinspace z \leftrightarrow y = z \text{ Imp } x \vee \exists v (v \leq x 
            \wedge \text{Var } v \wedge x = v \text{ Gen } y)$\\ 
            $x$ es una INFERENCIA INMEDIATA de $y$ y $z$.
    \item $Bw \thinspace x \leftrightarrow \forall n (0 < n \leq l(x) \rightarrow Ax(n \thinspace Gl \thinspace x) \wedge \exists pq (0 < p,q < n \wedge 
            Fl \thinspace (n \thinspace Gl \thinspace x, p \thinspace Gl \thinspace x, q \thinspace Gl \thinspace x))) \wedge l(x) > 0 $\\
            $x$ es una DEDUCCIÓN (una secuencia finita de FÓRMULAS, cada una de las cuales es o un AXIOMA o una INFERENCIA INMEDIATA 
            de dos de las FÓRMULAS precedentes).
    \item $x \thinspace B \thinspace y \leftrightarrow Bw \thinspace x \wedge (l(x)) \thinspace Gl \thinspace x = y$ \\ $x$ es una DEDUCCIÓN DE LA FÓRMULA y.
    \item $\text{Bew } x \leftrightarrow \exists y \thinspace yBx$ \\ $x$ es una FÓRMULA DEDUCIBLE. ($\text{Bew } x$ es la única de las nociones 
            $1$-$46$ de la que no podemos afirmar que sea recursiva primitiva).
\end{enumerate}

El hecho puede ser formulado vagamente diciendo que cada relación recursiva primitiva es definible en el sistema $P$ (interpretado en cuanto a su 
contenido del modo habitual) puede ser expresado con precisión y sin referencia a ninguna interpretación natural de las fórmulas de $P$, 
mediante el siguiente teorema:

\begin{teorema} \label{teo:TeoremaV}
    Para cada relación recursiva primitiva $n$-aria $R$ hay un SIGNO RELACIONAL $r$ (con las VARIABLES LIBRES\footnote{Las VARIABLES $u_1, \dots, u_n$ pueden
    estar elegidas de cualquier manera.} $u_1, u_2, \dots, u_n$ tal que para cada $n$-tuplo de números naturales $(x_1, \dots, x_n)$ vale:
    
    \begin{equation} \label{eq:Teo5-3}
        R x_1, \dots, x_n \rightarrow \text{Bew} \left(Sb \left(r 
        \begin{aligned}
            &u_1, \dots, u_n\\
            &Z(x_1), \dots, Z(x_n)
        \end{aligned}\right)\right)
    \end{equation}

    \begin{equation} \label{eq:Teo5-4}
        \lnot R x_1, \dots, x_n \rightarrow \text{Bew} \left(\text{Neg } Sb \left(r 
        \begin{aligned}
            &u_1, \dots, u_n\\
            &Z(x_1), \dots, Z(x_n)
        \end{aligned}\right)\right)
    \end{equation}
\end{teorema}

\begin{proof}
    Aquí nos limitaremos a esbozar la prueba de este teorema\footnote{Desde luego, el \autoref{teo:TeoremaV} se basa en el hecho de que, dada una relación
    recursiva primitiva $R$, para cada $n$-tuplo de números naturales podemos decidir en base a los axiomas del sistema $P$ si esos números están en la 
    relación $R$ o no.}, pues no ofrece ninguna dificultad de principio, pero es bastante larga. Probamos el teorema para todas las relaciones
    $R x_1, \dots, x_n $ de la forma $x_1 = f(x_2, \dots, x_n)$\footnote{De ahí se sigue inmediatamente su validez para toda relación recursiva
    primitiva, pues cada tal relación es equivalente a $0 = f(x_1, \dots, n_n)$, donde $f$ es recursiva primitiva.} (donde $f$ es una función recursiva 
    primitiva) por inducción completa sobre el grado de $f$. Para funciones de grado $1$ (es decir, constantes y la función $x + 1$) el teorema es trivial. 
    Sea $f$ de grado $m$. Entonces $f$ se obtiene a partir de las funciones $f_1, \dots, f_k$ de menor grado mediante las operaciones de sustitución o de 
    definición recursiva. Puesto que el teorema ya está probado para $f_1, \dots, f_k$, por hipótesis inductiva, hay SIGNOS RELACIONALES correspondientes
    $r_1, \dots, r_k$, tales que \eqref{eq:Teo5-3} y \eqref{eq:Teo5-4} valen. Los procesos de definición (sustitución y definición recursiva) mediante los 
    que se obtiene $f$ a partir de $f_1, \dots, f_k$ pueden ser ambos formalmente deducidos en el sistema $P$. Si se hace esto se obtiene a partir de $r_1, \dots, r_k$
    un nuevo SIGNO RELACIONAL $r$\footnote{Desde luego, cuando esta prueba se lleva a cabo con todo detalle, $r$ no se define indirectamente con ayuda 
    de su interpretación intuitiva, sino sólo por su estructura puramente formal.} para el cual puede probarse sin dificultad que valen \eqref{eq:Teo5-3} 
    y \eqref{eq:Teo5-4}, haciendo uso de la hipótesis inductiva. Un SIGNO RELACIONAL $r$, que corresponde\footnote{Que, por tanto, en la interpretación 
    usual expresa que esa relación se da.} de este modo a una relación recursiva primitiva, se llama recursivo primitivo.
\end{proof}

Ahora llegamos a la meta de nuestras consideraciones. Sea $K$ una clase cualquiera de FÓRMULAS. Designamos mediante $\text{Flg}(K)$\footnote{conjunto de 
inferencias a partir de $K$.} el mínimo conjunto de FÓRMULAS que contiene todas las FORMULAS de $K$ y todos los AXIOMAS y está clausurado respecto a 
la relación de «INFERENCIA INMEDIATA». Decimos que $K$ es $\omega$-consistente si no hay ningún SIGNO DE CLASE $a$, tal que
\begin{equation}
    \forall n \left(
        Sb \left( a
            \begin{aligned}
                v \\
                Z(n)
            \end{aligned}
        \right)
        \in \text{Flg}(K)
    \right)
    \wedge (\text{Neg}(v \text{ Gen } a)) \in \text{Flg}(K)
\end{equation}
donde $v$ es la VARIABLE LIBRE del SIGNO DE CLASE $a$. 

Cada sistema $\omega$-consistente es también consistente, desde luego. Pero la inversa no vale, como más adelante veremos.

El resultado general sobre la existencia de sentencias indecidibles dice como sigue:

\begin{teorema} \label{teo:TeoremaVI}
    Para cada clase recursiva primitiva y $\omega$-consistente $K$ de FORMULAS hay un SIGNO DE CLASE $r$ tal que ni $v \text{ Gen } a$ ni 
    $\text{Neg}(v \text{ Gen } a)$ pertenecen a $\text{Flg}(K)$ (donde $v$ es la VARIABLE LIBRE de $r$).
\end{teorema}

\begin{proof}
    Sea $K$ una clase recursiva primitiva y $\omega$-consistente cualquiera de FORMULAS. Definimos:
    \begin{flalign} \label{eq:TeoVI-5}
        Bw_K \thinspace x \leftrightarrow \forall n (n \leq l(x) \rightarrow Ax(&n \thinspace Gl \thinspace x) \vee (n \thinspace Gl \thinspace x) \in K 
        \vee \exists p,q (0 < p,q < n \wedge \\
        &\wedge Fl \thinspace (n \thinspace Gl \thinspace x, p \thinspace Gl \thinspace x, q \thinspace Gl \thinspace x))) \wedge l(x) > 0
    \end{flalign}
    (véase la análoga noción $44$).

    \begin{equation} \label{eq:TeoVI-6}
        x \thinspace B_K \thinspace y \leftrightarrow Bw_K \thinspace x \wedge (l(x)) \thinspace Gl \thinspace x = y
    \end{equation}
    
    \begin{equation} \label{eq:TeoVI-6.1}
        \text{Bew}_K \thinspace x \leftrightarrow \exists y \thinspace y B_K x
    \end{equation}
    (véanse las análogas nociones $45$ y $46$).

    Evidentemente, ocurre que
    \begin{equation} \label{eq:TeoVI-7}
        \forall x (\text{Bew}_K \thinspace x \leftrightarrow x \in \text{Flg}(K))
    \end{equation}
    
    \begin{equation} \label{eq:TeoVI-8}
        \forall (\text{Bew} \thinspace x \rightarrow \text{Bew}_K \thinspace x )
    \end{equation}

    Ahora definimos la relación 
    \begin{equation} \label{eq:TeoVI-8.1}
        Q \thinspace x \thinspace y \leftrightarrow \lnot B_K \left( Sb\left(y
            \begin{aligned}
                &19 \\
                &Z(y)
            \end{aligned}
        \right)\right)
    \end{equation}

    Puesto que $x B_K y$ y $Sb\left(y 
        \begin{aligned}
            &19 \\
            &Z(y)
        \end{aligned}
    \right)$ son recursivas primitivas (la primera por \eqref{eq:TeoVI-5} y \eqref{eq:TeoVI-6}; la segunda por los apartados $17$ y $31$), también lo es 
    $Q \thinspace x \thinspace y$. Por el \autoref{teo:TeoremaV} y por \eqref{eq:TeoVI-8} hay por tanto un SIGNO RELACIONAL $q$ (con las VARIABLES LIBRES 
    de $17$ y $19$), tal que
    \begin{equation} \label{eq:TeoVI-9}
        \lnot x B_K \left( Sb\left(y 
        \begin{aligned}
            &19 \\
            &Z(y)
        \end{aligned}
        \right)\right) \rightarrow \text{Bew}_K \thinspace \left( Sb\left(q 
        \begin{aligned}
            &17 \qquad 19 \\
            &Z(x)\quad Z(y)
        \end{aligned}
        \right)\right)
    \end{equation}

    \begin{equation} \label{eq:TeoVI-10}
        x B_K \left( Sb\left(y 
        \begin{aligned}
            &19 \\
            &Z(y)
        \end{aligned}
        \right)\right) \rightarrow \text{Bew}_K \thinspace \left(\text{Neg } Sb\left(q 
        \begin{aligned}
            &17 \qquad 19 \\
            &Z(x)\quad Z(y)
        \end{aligned}
        \right)\right)
    \end{equation}

    Sea 
    \begin{equation} \label{eq:TeoVI-11}
        p = 17 \text{ Gen } q
    \end{equation}
    ($p$ es un SIGNO DE CLASE con variable libre $19$) y sea también
    \begin{equation} \label{eq:TeoVI-12}
        r = Sb \left(q 
        \begin{aligned}
            &19 \\
            &Z(y)
        \end{aligned}
        \right)
    \end{equation}
    ($r$ es un SIGNO DE CLASE recursivo primitivo con la VARIABLE LIBRE $17$)\footnote{Pues $r$ se obtiene a partir del SIGNO RELACIONAL recursivo primitivo
    $q$ mediante SUSTITUCÓN de una VARIABLE por el NUMERAL de $p$.}. Entonces tenemos

    \begin{equation} \label{eq:TeoVI-13}
        Sb \left(p
        \begin{aligned}
            &19 \\
            &Z(y)
        \end{aligned}
        \right) = \left( (17 \text{ Gen } q)
        \begin{aligned}
            &19 \\
            &Z(y)
        \end{aligned}
        \right) = 17 \text{ Gen } Sb \left(q 
        \begin{aligned}
            &19 \\
            &Z(y)
        \end{aligned}
        \right) = 17 \text{ Gen } r
    \end{equation}
    (por \eqref{eq:TeoVI-11} y \eqref{eq:TeoVI-12})\footnote{Las operaciones Gen, $Sb$ son intercambiables, si se refieren a distintas VARIABLES.}; además

    \begin{equation} \label{eq:TeoVI-14}
        Sb\left(q 
        \begin{aligned}
            &17 \qquad 19 \\
            &Z(x)\quad Z(y)
        \end{aligned}
        \right) = Sb\left(q 
        \begin{aligned}
            &17 \\
            &Z(x)
        \end{aligned}
        \right)
    \end{equation}
    (por \eqref{eq:TeoVI-12}). Si ahora sustituimos $y$ por $p$ en \eqref{eq:TeoVI-9} y \eqref{eq:TeoVI-10}, entonces teniendo en cuenta  \eqref{eq:TeoVI-13} y 
    \eqref{eq:TeoVI-14} obtenemos que 

    \begin{equation} \label{eq:TeoVI-15}
        \lnot x \thinspace B_K \thinspace (17 \text{ Gen } r) \rightarrow \text{Bew}_K \thinspace \left(Sb\left(r
        \begin{aligned}
            &17 \\
            &Z(x)
        \end{aligned}
        \right)\right)
    \end{equation}

    \begin{equation} \label{eq:TeoVI-16}
        x \thinspace B_K \thinspace (17 \text{ Gen } r) \rightarrow \text{Bew}_K \thinspace \left(\text{Neg } Sb\left(r
        \begin{aligned}
            &17 \\
            &Z(x)
        \end{aligned}
        \right)\right)
    \end{equation}

    De aquí se sigue:
    \begin{enumerate}
        \item   $17 \text{ Gen } r$ no es $K$-DEDUCIBLE\footnote{Con «x es $K$-DEDUCIBLE» queremos decir que $x \in \text{Flg}(K)$, lo que, por \eqref{eq:TeoVI-7},
                significa los mismo que $\text{Bew}_K \thinspace x$.}. Pues si lo fuera, habría (según \eqref{eq:TeoVI-6.1}) un $n$ tal que 
                $n \thinspace B_K (17 \text{ Gen } r)$. Pero entonces, por \autoref{eq:TeoVI-16} ocurriría que 
                $\text{Bew}_K \thinspace \left(\text{Neg } Sb\left(r
                \begin{aligned}
                    &17 \\
                    &Z(x)
                \end{aligned}
                \right)\right)$, 
                mientras que, por otro lado, de la $K$-DEDUCIBILIDAD de $(17 \text{ Gen } r)$ se sigue también la de 
                $Sb\left(r
                \begin{aligned}
                    &17 \\
                    &Z(x)
                \end{aligned}
                \right)$.
                Por otro lado, $K$ sería inconsistente (y en especial $\omega$-consistente).
        \item Neg $(17 \text{ Gen } r)$ no es $K$-DEDUCIBLE.
                \begin{proof}
                    Como acabamos de probar, $(17 \text{ Gen } r)$ no es $K$-DEDUCIBLE, es decir, por \eqref{eq:TeoVI-6.1} ocurre que 
                    $\forall n \lnot (n \thinspace B_K (17 \text{ Gen } r))$. De ahí se sigue por \eqref{eq:TeoVI-15} que
                    $$\forall n \text{ Bew}_K Sb\left(r
                    \begin{aligned}
                        &17 \\
                        &Z(x)
                    \end{aligned}
                    \right)$$
                    lo que junto con $\text{Bew}_K(\text{Neg }  (17 \text{ Gen } r))$, es incompatible con la $\omega$-consistencia de $K$.
                \end{proof}
    \end{enumerate}

    Por tanto, $(17 \text{ Gen } r)$ es indecidible en base a $K$, con lo que el \autoref{teo:TeoremaVI} queda probado.
\end{proof}

Fácilmente se ve que la prueba que acabamos de presentar es constructiva\footnote{Pues todas las afirmaciones existenciales que aparecen en la prueba se
basan en el \autoref{teo:TeoremaV}, que, como fácilmente se ve, es intuicionistamente aceptable.}, es decir, hemos probado de un modo intuicionistamente 
del todo aceptable lo siguiente: Sea dada una clase cualquiera de FORMULAS, definida de un modo recursivo primitivo. Entonces, si se nos presentara una 
decisión formal (en base a $K$) de la SENTENCIA $17 \text{ Gen } r$ (que puede ser efectivamente escrita) nosotros podríamos ofrecer efectivamente:

\begin{enumerate}
    \item Una DEDUCCIÓN de $\text{Neg }(17 \text{ Gen } r)$.
    \item Para cada $n$ dado, una DEDUCCIÓN de $Sb\left(r
            \begin{aligned}
                &17 \\
                &Z(x)
            \end{aligned}
            \right)$.
\end{enumerate}
Es decir, una decisión formal de $17 \text{ Gen } r$ tendría como consecuencia la exhibición efectiva de una $\omega$-inconsistencia.

Diremos que una relación (o clase) entre números naturales $Rx_1, \dots, x_n$ es \textit{decidible} si existe un SIGNO RELACIONAL $n$-ario $r$, tal que 
\eqref{eq:Teo5-3} y \eqref{eq:Teo5-4} valen para él\footnote{Véase el \autoref{teo:TeoremaV}.}. En especial, por tanto, del teorema V resulta que cada 
relación recursiva primitiva es decidible. Análogamente diremos que un SIGNO RELACIONAL es \textit{decidible} si corresponde de este modo a una relación decidible. 
Ahora bien, para la existencia de sentencias indecidibles basta con que la clase $K$ sea $\omega$-consistente y decidible. Pues la decidibilidad se transmite de
$K$ a $xB_K y$ y a $Q\thinspace x \thinspace y$\footnote{Véanse para la primera transmisión \eqref{eq:TeoVI-5} y \eqref{eq:TeoVI-6}; para la segunda véase 
\eqref{eq:TeoVI-8.1}.}, y esto es todo lo que se utilizó en la prueba antes expuesta. La sentencia indecidible tiene en este caso la forma $v \text{ Gen } r$, 
donde $r$ es un SIGNO DE CLASE decidible. (Incluso basta con que $K$ sea decidible en el sistema ampliado con $K$).

Si en vez de suponer que $K$ es $\omega$-consistente nos limitamos a suponer que es consistente, entonces ya no se sigue (por la prueba anterior) que exista 
una sentencia indecidible, pero sí se sigue que existe una propiedad $r$, para la que no se puede dar un contraejemplo, y para la que tampoco se puede probar que
todos los números la tengan. Pues en la prueba de que $17 \text{ Gen } r$ no es $K$-DEDUCIBLE habíamos utilizado sólo el hecho de que $K$ era consistente y por
\eqref{eq:TeoVI-15} de $\lnot (\text{Bew}_K (17 \text{ Gen } r))$ se sigue que para cada número $x$ se cumpla $Sb\left(r
\begin{aligned}
    &17 \\
    &Z(x)
\end{aligned}
\right)$
y, por tanto, que $\text{Neg } Sb\left(r
\begin{aligned}
    &17 \\
    &Z(x)
\end{aligned}
\right)$ no es $K$-DEDUCIBLE para ningún número.

Si añadimos $\text{Neg } (17 \text{ Gen } r)$ a $K$, obtenemos una CLASE DE FORMULAS $K'$ que es consistente, pero no $\omega$-consistente. $K'$ es consistente, 
pues si no, $17 \text{ Gen } r$ sería $K$-DEDUCIBLE. Pero $K'$ no es $\omega$-consistente, pues por $\lnot (\text{Bew}_K (17 \text{ Gen } r))$ y \eqref{eq:TeoVI-15}
ocurre que $\forall x \text{ Bew }_K Sb\left(r
\begin{aligned}
    &17 \\
    &Z(x)
\end{aligned}
\right)$ y por tanto que $\forall x \text{ Bew }_{K'} Sb\left(r
\begin{aligned}
    &17 \\
    &Z(x)
\end{aligned}
\right)$, mientras que por otro lado ocurre, claro está, que $\text{ Bew }_{K'} (\text{Neg } (17 \text{ Gen } r))$\footnote{Desde luego, con esto sólo hemos probado 
la existencia de clases $K$, que son consistentes, pero no $\omega$-consistentes, bajo el supuesto de que hay algún $K$ consistente (es decir, de que $P$ es 
consistente).}.

Un caso especial del \autoref{teo:TeoremaVI} se da cuando la clase $K$ consta de un número finito de FORMULAS (y, si se quiere, de las que se obtiene a partir 
de ellas por ELEVACIÓN DE TIPO). Naturalmente, cada clase finita $K$ es recursiva primitiva. Sea $a$ el mayor número contenido en $K$. Entonces ocurre para $K$ que
\begin{equation}
    x \in K \leftrightarrow \exists n,m \thinspace (m \leq x \wedge n \leq a \wedge n \in K \wedge x = m \thinspace Th \thinspace n)
\end{equation}

Por tanto, $K$ es recursiva primitiva. Esto nos permite inferir que incluso con ayuda del axioma de elección (para todos los tipos) o de la hipótesis generalizada 
del continuo no todas las sentencias son decidibles, suponiendo que estas hipótesis sean $\omega$-consistentes.

En la prueba del \autoref{teo:TeoremaVI} no se utilizaron otras propiedades del sistema $P$ que las siguientes:

\begin{enumerate}
    \item La clase de los axiomas y de las reglas de inferencia (es decir, la relación de «inferencia inmediata») son recursivamente definibles (tan pronto como 
            reemplazamos de algún modo los signos primitivos por números naturales).
    \item Cada relación recursiva primitiva es definible (en el sentido del \autoref{teo:TeoremaV}) en el sistema $P$.
\end{enumerate}

Por eso en cada sistema formal, que satisface los supuestos $1$ y $2$ y es $\omega$-consistente, hay sentencias indecidibles de la forma $\forall x\thinspace Fx$,
donde $F$ es una propiedad recursiva primitiva de los números naturales, y lo mismo ocurre en cada extensión de un tal sistema resultante de añadir una clase 
recursivamente definible y $\omega$-consistente de axiomas. Como fácilmente se comprueba, entre los sistemas que satisfacen los supuestos $1$ y $2$ se cuentan
las teorías axiomáticas de conjuntos de Zermelo-Faenkel y de von Neumann\footnote{La prueba del supuesto 1 resulta aquí incluso más sencilla que en el caso
del sistema $P$, pues sólo hay un tipo de variable (o dos en el sistema de von Neumann).}, así como el sistema axiomático de la teoría de números que consta de 
los axiomas de Peano, la definición recursiva (según el esquema \eqref{eq:recdef}) y las reglas lógicas. El supuesto $1$ es satisfecho en general por cada sistema, 
cuyas reglas de inferencia son las usuales y cuyos axiomas (análogamente a los de $P$) se obtienen por sustitución a partir de un número finito de 
esquemas.\footnote{Como se mostrará en la segunda parte de este artículo, la verdadera razón de la incompletud inherente a todos los sistemas formales de la matemática 
es que la formación de tipos cada vez mayores puede continuarse hasta lo transfinito (véase \cite{hilbert1926unendliche}), mientras que en cada sistema formal a lo 
sumo disponemos de un número infinito numerable de ellos. En efecto, se puede mostrar que las sentencias indecidibles aquí construidas se vuelven decidibles si 
añadimos tipos más altos adecuados (por ejemplo, el tipo $\omega$ al sistema $P$). Algo parecido ocurre con la teoría axiomática de conjuntos.}

%%%%%%%%%%%%%%%%%%%%%%%%%%%%%%%%%%%%%%%%%%%%%%%%%%%%%%%%%%%%%%%%%%%%%%%%%%%%%%%
%%%%%%%%%%%%%%%%%%%%%%%%%%%%%%%%%%%%%%%%%%%%%%%%%%%%%%%%%%%%%%%%%%%%%%%%%%%%%%%
%%%%%%%%%%%%%%%%%%%%%%%%%%%%%%%%%%%%%%%%%%%%%%%%%%%%%%%%%%%%%%%%%%%%%%%%%%%%%%%
\section{Consistencia}

Ahora vamos a sacar algunas consecuencias del \autoref{teo:TeoremaVI} y para ello damos la siguiente definición:

\begin{definicion}
    Una relación (o clase) se llama \textit{aritmética} si puede ser definida con la sola ayuda de las nociones $+$ y $\cdot$ (adición y multiplicación de números 
    naturales)\footnote{Aquí y en lo sucesivo consideramos siempre que el cero es un número natural.} y de las constantes lógicas $\vee, \lnot, \forall x$ y $=$,
    donde $\forall x$ y $=$ sólo pueden referirse a números naturales\footnote{de un tal concepto debe constar exclusivamente de los signos indicados, de variables 
    $x, y, \dots$, para números naturales y de las constantes $0$ y $1$ (no pueden aparecer en él variables para funciones o conjuntos). En los prefijos, desde luego, 
    puede aparecer cualquier otra variable para números en vez de $x$.}. 
\end{definicion}

De modo análogo se define la noción de «sentencia aritmética». En especial son aritméticas las relaciones «mayor que» y «congruente módulo $n$», pues ocurre

\begin{flalign}
    &x > y \leftrightarrow \lnot \exists z (y = x+z) \\ 
    &x \equiv y \pmod{n} \leftrightarrow  \exists z (x=y+z \cdot n \vee y = x+z\cdot n)
\end{flalign}

Ahora tenemos el siguiente:
\begin{teorema} \label{teo:TeoremaVII}
    Cada relación recursiva primitiva es aritmética.
\end{teorema}
    
\begin{proof}
    Probaremos la siguiente versión de este teorema: \textit{Cada relación de la forma $x_0 = f(x_1 , \dots, x_n)$, donde $f$ es recursiva primitiva, es aritmética}.
    Procedamos por inducción completa sobre el grado de $f$. Tenga $f$ el grado $s$ $( s > 1 )$ . Entonces ocurre que o bien
    \begin{enumerate}
        \item $f(x_1 , \dots, x_n) = h(j_1(x_1 , \dots, x_n), j_2(x_1 , \dots, x_n), \dots, j_m(x_1 , \dots, x_n))$\footnote{Naturalmente, no es necesario que todos 
                los $x_1 , \dots, x_n$ aparezcan de hecho en los $j_i$.}\\ (donde $h$ y todos los $j_i$ son de grado menor que $s$) o bien se da
        \item $f(0, x_2, \dots, x_n) = p(x_2 , \dots, x_n)$\\
                $f(k+1, x_2, \dots, x_n) = q(k, f(k, x_2, \dots, x_n), x_2 , \dots, x_n)$\\ (donde $p$ y $q$ so nde grado menor que $s$).
    \end{enumerate}
    En el primer caso tenemos 
    \begin{flalign}
        x_0 = f(x_1, \dots, x_n) \leftrightarrow &\exists y_1, \dots, y_m \thinspace (R x_0 y_1, \dots, y_m \wedge \\ 
        &\wedge S_1 y_1, x_1, \dots, x_n \wedge \cdots \wedge S_m y_m, x_1, \dots, x_n)
    \end{flalign}
    donde $R$ y $S_i$ son las relaciones aritméticas, existentes por hipótesis inductiva, que son equivalentes con $x_0 = h(y_1 , \dots, y_n)$ y 
    $y = j_i(x_1 , \dots, x_n)$. Por eso en este caso $x_0 = f(x_1 , \dots, x_n)$ es aritmética.

    En el segundo caso vamos a aplicar el siguiente procedimiento. Podemos expresar la relación $x_0 = f(x_1 , \dots, x_n)$ con ayuda del concepto «sucesión de 
    números» $(t)$\footnote{$t$ es aquí una variable que toma como valores las sucesiones de números naturales. $t_k$ designa el miembro $k+1$-avo de una sucesión 
    $t$; $t_0$ designa su primer miembro.} del siguiente modo:
    \begin{flalign}
        x_0 = &f(x_1 , \dots, x_n) \leftrightarrow \exists t (t_0 = p(x_2, \dots, x_n) \wedge \\
        &\wedge \forall k (k < x_1 \rightarrow t_{k+1} = q(k, t_k, x_2, \dots, x_n)) \wedge x_0 = t_{x_1})
    \end{flalign}

    Si $S$ y $x_2, \dots, x_n$ y $Tz \thinspace x_1, \dots, x_{n+1}$ son las relaciones aritméticas, existentes por hipótesis inductiva, que son equivalentes a
    $y = p(x_2, \dots, x_n)$ y $z = q(x_1, \dots, x_{n+1})$, respectivamente, entonces
    \begin{flalign} \label{eq:TeoVII-17}
        x_0 = &f(x_1 , \dots, x_n) \leftrightarrow \exists t (St_0 \thinspace x_2, \dots, x_n \wedge \\
        &\wedge \forall k (k < x_1 \rightarrow T(t_{k+1}, k, t_k, x_2, \dots, x_n)) \wedge x_0 = t_{x_1})
    \end{flalign}
    
    Ahora sustituimos la noción de «sucesión de números» por la de «par de números», asignando al par de números $n$, $d$ la secuencia numérica $t^(n,d)$ (tal que
    $t^(n,d)_k = [n]_{1 + (k+1)d}$, donde $[n]_m$ denota el mínimo resto no-negativo de $n$ módulo $m$).

    Entonces vale el siguiente lema:
    \begin{itemize}
        \item[] \begin{lema} \label{lem:lema-01}
                    Si $t$ es una sucesión cualquiera de números naturales y $k$ es un número natural cualquiera, entonces hay un par de números naturales $n$, $d$, 
                    tales que $t^(n,d)$ y $t$ coinciden en los primeros $k$ miembros.
                \end{lema}
        \item[] \begin{proof} (Demostración del Lema) \\
                    Sea $l$ el máximo de los números $k, t_0, t_1, \dots, t_{k-1}$. Determinemos $n$, de tal modo que
                    \begin{equation}
                        n \equiv t_i \pmod{(1 + (i+1)l!)} \qquad \text{para } i = 0, 1, \dots, k-1
                    \end{equation}
                    lo cual es posible, pues cada dos números $1 + (i+1)l!) \thinspace (i = 0, 1, \dots, k-1)$ son primos entre sí. En efecto, un número primo contenido
                    en dos de esos números debería estar contenido en la diferencia $(i_1 - i_2)l!$, y puesto que $| i_1 - i_2 | < l$, también debería estar contenido
                    en $l!$, lo que es imposible. Así pues, el par de números $n$, $l!$ tiene la propiedad deseada.
                \end{proof}
    \end{itemize}

    Puesto que la relación $x = [n]_m$ está definida por 
    \begin{equation}
        x \equiv n \pmod{m} \wedge x < m
    \end{equation}
    y por tanto es aritmética, también es aritmética la relación
    \begin{equation}
        P x_0, x_1, \dots, x_n
    \end{equation}
    definida del siguiente modo:
    \begin{flalign}
        P &x_0, \dots, x_n \leftrightarrow \exists n, d \thinspace (S[n]_{d+1}, x_2, \dots, x_n \wedge \\
        \wedge \forall k(k <x_1 \rightarrow &T[n]_{1+d(k+2)}, k, [n]_{1+d(k+1)}, x_2, \dots, x_n) \wedge x_0 = [n]_{1+d(x_1+1)})
    \end{flalign}

    Pero por \eqref{eq:TeoVII-17} y el \autoref{lem:lema-01} esta relación es equivalente a $x_0 = f(x_1 , \dots, x_n)$ (en la sucesión $t$, tal como interviene
    en \eqref{eq:TeoVII-17}, sólo importan sus primeros $x_1 + 1$ miembros).

    Con esto queda probado el \autoref{teo:TeoremaVII}.
\end{proof}

Conforme al \autoref{teo:TeoremaVII} para cada problema de la forma $\forall x F x$ (donde $F$ es recursiva primitiva) hay un problema aritmético equivalente, 
y puesto que toda la prueba del \autoref{teo:TeoremaVII} (para cada $F$ particular) se puede formalizar en el sistema $P$, esta equivalencia es deducible en $P$. 
Por eso vale el siguiente

\begin{teorema} \label{teo:TeoremaVIII}
    En cada uno de los sistemas formales\footnote{Se trata de los sistemas formales $\omega$-consistentes que resultan de añadir a $P$ una clase recursivamente 
    definible de axiomas.} mencionados en el \autoref{teo:TeoremaVI} hay sentencias aritméticas indecidibles.
\end{teorema}

Lo mismo vale (según el \autoref{lem:lema-01}) para la teoría axiomática de conjuntos y sus extensiones mediante clases recursivas primitivas y $\omega$-consistentes 
de axiomas.

Finalmente derivamos todavía el siguiente resultado:

\begin{teorema} \label{teo:TeoremaIX}
    En todos los sistemas formales mencionados en el \autoref{teo:TeoremaVI} hay problemas indecidibles de la lógica pura de predicados de primer orden\footnote{Véase 
    \cite{hilbert1999principles}. En el sistema $P$ entendemos por fórmulas de la lógica pura de predicados de primer orden las fórmulas que se obtienen al sustituir 
    las relaciones por clases de tipo superior en las fórmulas del cálculo de predicados de primer orden de $PM$.} (es decir, fórmulas de la lógica pura de primer 
    orden, respecto a las cuales no podemos probar ni su validez ni la existencia de un contraejemplo).
\end{teorema}

Esto se basa en el siguiente teorema (y por tanto el presente será demostrado como consecuencia del siguiente):
\begin{teorema} \label{teo:TeoremaX}
    Cada problema de la forma $\forall x F x$ (donde $F$ es recursiva primitiva) es reducible a la cuestión de si una determinada fórmula de la lógica pura de primer 
    orden es satisfacible o no (es decir, para cada F recursiva primitiva podemos encontrar una fórmula de la lógica pura de primer orden, cuya satisfacibilidad es
    equivalente a la verdad de $\forall x F x$).
\end{teorema}

Consideramos como fórmulas de la lógica pura de primer orden las fórmulas formadas con los signos primitivos: 
$\lnot, \vee, \forall\text{x}, =; \text{x}, \text{y}, \dots \text{(variables individuales)}, \text{Hx}, \text{Gxy}, \text{Lxyz}$ (variables relacionales y de 
propiedades), donde $\forall \text{x y } =$ sólo pueden referirse a individuos\footnote{Hilbert y W. Ackermann no incluyen el signo $=$
entre los del cálculo lógico de primer orden. Pero para cada fórmula en que aparece el signo $=$ existe otra fórmula en que ese signo no aparece y que es
satisfacible si y sólo si la primera fórmula lo es.}. A estos signos añadimos todavía una tercera especie de variables $\text{g(x), l(x,y), j(x,y,z)}$, etc., 
que representan funciones de objetos (es decir, $\text{g(x), l(x,y)}$, etc., designan funciones cuyos argumentos y valores son individuos)\footnote{Además, el dominio 
de definición debe ser siempre el dominio entero de individuos.}. Una fórmula que, además de los signos de la lógica pura de primer orden, contenga variables de la 
tercera especie ($\text{g(x), l(x,y)}$, etc.), será llamada una fórmula en sentido amplio\footnote{Variables de la tercera especie pueden aparecer en todos los 
lugares ocupados por variables individuales, por ejemplo, y=l(x), Gxl(y), Gl(x,g(y)), x, etc.}. Las nociones de «satisfacible» y «válido» pueden extenderse sin
más a las fórmulas en sentido amplio, y tenemos un teorema que dice que para cada fórmula $\alpha$ en sentido amplio podemos encontrar una fórmula $\beta$ de la lógica 
pura de primer orden, tal que la satisfacibilidad de $\alpha$ es equivalente a la de $\beta$. $\beta$ se obtiene a partir de $\alpha$, sustituyendo las variables de 
la tercera especie ($\text{g(x), l(x,y)}$, etc.), \dots que aparecen en $\alpha$ por expresiones de la forma $\iota$zGzx, $\iota$zLzxy, \dots eliminando las funciones 
«descriptivas» por el método usado en $PM$, y uniendo conyuntivamente\footnote{Es decir, formando la conyunción.} la fórmula así obtenida con una expresión que diga que
todos los G, L, \dots que hemos puesto en vez de los g, l, \dots son unívocos respecto a su primer lugar.

Ahora mostramos que para cada problema de la forma $\forall x F x$ (donde $F$ es recursiva primitiva) hay un problema equivalente referente a la satisfacibilidad de 
una fórmula en sentido amplio, de donde -según la observación que acabamos de hacer- se sigue el \autoref{teo:TeoremaX}. 

Puesto que $F$ es recursiva primitiva, hay una función recursiva primitiva $f(x)$, tal que $Fx \leftrightarrow f(x) = 0$, y para $f$ hay una secuencia de funciones
$f_1, f_2, \dots, f_n$, tales que $f_n = f$, $f_1(x) = x+1$ y para cada $f_k(1 < k \leq n)$ o bien

\begin{enumerate}
    \item   \begin{flalign} \label{eq:TeoX-18}
                &\forall x_2, \dots, x_m (f_k(0, x_2, \dots, x_m) = f_p(x_2, \dots, x_m)) \\
                \forall x, x_2, \dots, x_m& (f_k(f_1(x), x_2, \dots, x_m) = f_q(x, f_k(x_1, x_2, \dots, x_m), x_2, \dots, x_m))
            \end{flalign}
            donde $p,q < k$, o bien
    \item   \begin{equation}\label{eq:TeoX-19}
                \forall x_1, \dots, x_m (f_k(x_1, \dots, x_m) = f_r(f_{i_1}(\mathfrak{x}_1), \dots, f_{i_s}(\mathfrak{x}_s)))\footnote{Los $\mathfrak{x_i}$ representan
                cualesquiera secuencias finitas de las variables $x_1, x_2, \dots, x_m$; por ejemplo $x_1, x_3, x_2$.}
            \end{equation}
            donde $r<k, i_v <k$ (para cada $v = 1,2,\dots,s)$, o bien
    \item   \begin{equation}\label{eq:TeoX-20}
            \forall x_1, \dots, x_m (f_k(x_1, \dots, x_m) = f_1(f_1(\dots(f_1(0))\dots)))
            \end{equation}
\end{enumerate}

Además formamos las sentencias:
\begin{equation} \label{eq:TeoX-21}
    \forall x \lnot f_1(x) = 0 \wedge \forall x, y \thinspace (f_1(x) = f_1(y) \rightarrow x = y)
\end{equation}

\begin{equation} \label{eq:TeoX-22}
    \forall x \thinspace (f_n(x) = 0)
\end{equation}

En todas las fórmulas \eqref{eq:TeoX-18}, \eqref{eq:TeoX-19}, \eqref{eq:TeoX-20} (para $ k=2,3, \dots, n$) y en \eqref{eq:TeoX-21} y \eqref{eq:TeoX-22} sustituimos
ahora las funciones $f_1$ por las variables funcionales $g_i$ y el número $0$ por una variable individual $\text{x}_0$ que no haya aparecido hasta ahora, y a continuación 
formamos la conyunción $\gamma$ de todas las fórmulas así obtenidas.
\begin{corolario}
    La fórmula $\exists \text{x}_0 \thinspace \gamma$ tiene la propiedad requerida, es decir,
    $$\forall x f(x) = 0 \qquad \leftrightarrow \qquad \exists \text{x}_0 \thinspace \gamma \text{ es satisfacible.}$$
\end{corolario}
\begin{proof}
    La implicación $\rightarrow$ es directa, pues si las funciones $f_1, f_2, \dots, f_n$ se ponen en vez de las variables $g_1, g_2, \dots, g_n$ en 
    $\exists \text{x}_0 \thinspace \gamma$, evidentemente resulta un enunciado verdadero.

    Veamos ahora el recíproco. Sean $h_1, h_2, \dots, h_n$ las funciones, existentes por hipótesis, que producen un enunciado verdadero cuando se ponen en vez de
    $g_1, g_2, \dots, g_n$. Su dominio de individuos sea $D$. Puesto que $\exists \text{x}_0 \thinspace \gamma$ vale para las funciones $h_i$ hay un individuo $a$ 
    (de $D$), tal que todas las fórmulas \eqref{eq:TeoX-18} a \eqref{eq:TeoX-22}, si reemplazamos las $f_i$ por $h_1$ y $0$ por $a$. Ahora formamos la mínima subclase de $D$, que
    contiene $a$ y está clausurada respecto a la operación $h_1(x)$. Esta subclase $D'$ tiene la propiedad de que cada una de las funciones $h_i$, aplicada a elementos de $D'$, 
    proporciona de nuevo elementos de $D'$. Pues para $h_1$ vale esto por definición de $D'$, y por \eqref{eq:TeoX-18}, \eqref{eq:TeoX-19} y \eqref{eq:TeoX-20} esta propiedad se 
    transmite de los $h_1$ con índice menor a los de índice mayor. Designemos mediante $h_i'$las funciones que resultan de restringir los $h_1$ al dominio de individuos $D'$.
    También para éstas funciones valen todas las fórmulas \eqref{eq:TeoX-18} a \eqref{eq:TeoX-22} (reemplazando las $f_i$ por $h_1$ y $0$ por $a$).

    Puesto que \eqref{eq:TeoX-21} vale para $h_i'$ y $a$, podemos aplicar biunívocamente los individuos de $D'$ a los números naturales, de tal modo que $a$ se convierte en $0$
    y la función $h_1'$ en la función $f_1$ del siguiente. Pero mediante esta aplicación todas las la funciones $h_i'$ se convierten en la funciones $f_i$, y puesto que 
    \eqref{eq:TeoX-22} vale para $h_n'$ y $a$, ocurre que $\forall x \thinspace f_n(x) = 0$ o $\forall x \thinspace f(x) = 0$, que es lo que queríamos probar\footnote{Del
    \autoref{teo:TeoremaX} se sigue, por ejemplo, que los problemas de Fermat y de Goldbach podrían ser resueltos, si se pudiera resolver el problema de la decisión para la 
    lógica de primer orden.}.
\end{proof}

Puesto que (para cada $F$ especial) la argumentación que conduce al \autoref{teo:TeoremaX} puede también ser llevada a cabo en el sistema formal $P$, también la equivalencia 
entre un enunciado de la forma $\forall x F x$ (donde $F$ es recursivo primitivo) y la satisfacibilidad de la fórmula correspondiente de la lógica de primer orden es demostrable en $P$, 
y por tanto de la indecidibilidad de lo uno se sigue la de lo otro, con lo que queda probado el \autoref{teo:TeoremaIX}\footnote{Naturalmente, el \autoref{teo:TeoremaIX} también vale 
para la teoría axiomática de conjuntos, así como para sus extensiones obtenidas mediante el añadido de clases recursivas primitivas y $@o$-consistentes de axiomas, pues también en esos
sistemas hay enunciados indecidibles de la forma $\forall xFx$ (donde $F$ es recursivo primitivo).}.


%%%%%%%%%%%%%%%%%%%%%%%%%%%%%%%%%%%
%%%%%%%%%%%%%%%%%%%%%%%%%%%%%%%%%%%
\section{Consecuencias}
De los resultados del apartado de la Decibilidad, se sigue una sorprendente consecuencia relativa a la prueba de la consistencia del sistema $P$ (y de sus extensiones), que queda 
expresada en el siguiente teorema:

\begin{teorema} \label{teo:TeoremaXI}
    Sea $K$ una clase recursiva primitiva y consistente\footnote{$K$ es consistente (abreviadamente, $\text{Wid } K$) se define así: 
    $\text{Wid } K \leftrightarrow \exists x \thinspace (\text{Form } x \wedge \lnot \text{Bew}_K x)$.} cualquiera de FORMULAS. Entonces ocurre que la SENTENCIA que dice que $K$ es 
    consistente no es $K$-DEDUCIBLE. En especial, la consistencia de $P$ no es deducible\footnote{Esto se sigue al tomar como K la clase vacía de FORMULAS.} en $P$, suponiendo que $P$ sea
    consistente (en caso contrario, naturalmente, toda fórmula sería deducible).
\end{teorema}
\begin{proof}
    La prueba (meramente esbozada) es la siguiente: Sea $K$ una clase recursiva primitiva cualquiera (pero elegida de una vez por todas para las siguientes consideraciones) de FORMULAS 
    (en el caso más simple, la clase vacía). Para probar el hecho de que $17 \text{ Gen } r$\footnote{Naturalmente, $r$ depende (como $p$) de $K$.} no es K-DEDUCIBLE, sólo habíamos utilizado 
    la consistencia de $K$, es decir, ocurre que
    \begin{equation} \label{eq:teoXI-23}
        \text{Wid } K \rightarrow \lnot \text{Bew}_K (17 \text{ Gen } r)
    \end{equation}
    es decir, según \eqref{eq:TeoVI-6.1}:
    \begin{equation}
        \text{Wid } K \rightarrow \forall x \lnot (x \thinspace B_K (17 \text{ Gen } r))
    \end{equation}

    Ahora, por \eqref{eq:TeoVI-13} se tiene 
    \begin{equation}
        17 \text{ Gen } r = Sb\left(p
        \begin{aligned}
            &19 \\
            &Z(p)
        \end{aligned}
        \right)
    \end{equation}
    y por tanto
    \begin{equation}
        \text{Wid } K \rightarrow \forall x \lnot \left(x \thinspace B_K Sb\left(p
        \begin{aligned}
            &19 \\
            &Z(p)
        \end{aligned}
        \right)\right)
    \end{equation}
    es decir, por \eqref{eq:TeoVI-8.1}
    \begin{equation} \label{eq:teoXI-24}
        \text{Wid } K \rightarrow \forall x \thinspace Qxp
    \end{equation}

    Ahora constatamos que todas las nociones definidas hasta ahora (y todas las afirmaciones probadas) son también definibles (o deducibles) en $P$. Pues no hemos utilizado más que los 
    métodos usuales de definición y prueba de la matemática clásica, tal y como están formalizados en $P$. Especialmente ocurre que $K$ (como cada clase recursiva primitiva) es definible en $P$. 
    Sea $w$ la SENTENCIA mediante la que se expresa en $P$ que $\text{Wid } K$ La relación $Qxy$ expresa (conforme a \eqref{eq:TeoVI-8.1}, \eqref{eq:TeoVI-9} y \eqref{eq:TeoVI-10}) mediante 
    el SIGNO RELACIONAL $q$, por tanto, $Qxp$ por $r$ (pues por \eqref{eq:TeoVI-12} ocurre que 
    $r = Sb\left(p
    \begin{aligned}
        &19 \\
        &Z(p)
    \end{aligned}
    \right)$), el enunciado $\forall x \thinspace Qxp$ por $17 \text{ Gen } r$.

    Por \eqref{eq:teoXI-24}, se tiene que $w \text{ Imp } (17 \text{ Gen } r)$ es deducible\footnote{Que de \eqref{eq:teoXI-23} podríamos inferir la verdad de $w \text{ Imp } (17 \text{ Gen } r)$ 
    se debe a que la sentencia indecidible $17 \text{ Gen } r$ afirma su propia indeducibilidad, como ya observamos al principio.} en $P$ (y, por tanto, $K$-DEDUCIBLE). Si $w$ fuese $K$-DEDUCIBLE,
    también $17 \text{ Gen } r$ sería $K$-DEDUCIBLE, y de ahí se seguiría, por \eqref{eq:teoXI-23}, que $K$ no es consistente.
\end{proof}

Nótese que también esta prueba es constructiva, es decir, caso de que nos presenten una DEDUCCIÓN de $w$ a partir de $K$, nuestra prueba permite obtener efectivamente una contradicción a partir 
de $K$. La prueba entera del \autoref{teo:TeoremaXI} puede trasladarse literalmente a la teoría axiomática de conjuntos $M$ y a la matemática clásica axiomática\footnote{Véase \cite{v1927hilbertschen}.} 
$A$ y también aquí obtenemos el mismo resultado: No hay prueba alguna de la consistencia de $M$ (o de $A$), que pudiera ser formalizada en $M$ (o en $A$), suponiendo que $M$(o $A$) sea consistente. 
Hagamos notar explícitamente que el \autoref{teo:TeoremaXI} (y los resultados correspondientes sobre $M$ y $A$) no se oponen al punto de vista formalista de Hilbert. En efecto, este punto de vista 
sólo supone la existencia de una prueba de consistencia llevada a cabo por medios finitarios y sería concebible que hubiera pruebas finitarias que no fuesen representables en $P$ (ni en $M$ o $A$).
Puesto que para cada clase consistente $K$ $w$ no es $K$-DEDUCIBLE, ocurre que siempre que $\text{Neg}(w)$ no es $K$-DEDUCIBLE ya tenemos sentencias (a saber, $w$) no decidibles en base a $K$.
Con otras palabras en el \autoref{teo:TeoremaVI} podemos sustituir la hipótesis de la $\omega$-consistencia por esta otra: la sentencia «$K$ es inconsistente» no es $K$-DEDUCIBLE. (Nótese que existen 
clases consistentes $K$ para las que esa sentencia es $K$-DEDUCIBLE.)

\subsection{La noción de sistema formal}
En este trabajo se ha limitado básicamente al sistema $P$ y sólo se han indicado las aplicaciones a otros sistemas. Sin embargo, como consecuencia de avances posteriores, en particular del
hecho de que gracias a la obra de A. M. Turing\footnote{Véase \cite{turing1936computable}} ahora disponemos de una definición precisa e indudablemente adecuada de la noción general de sistema formal, 
ahora es posible dar una versión completamente general de los teoremas \autoref{teo:TeoremaVI} y \autoref{teo:TeoremaXI}. Es decir, se puede probar rigurosamente que en cada sistema formal consistente 
que contenga una cierta porción de teoría finitaria de números hay sentencias aritméticas indecidibles y que, además, la consistencia de cualquiera de esos sistemas no puede ser probada en el sistema mismo.
\endinput
%------------------------------------------------------------------------------------
% FIN DEL CAPÍTULO. 
%------------------------------------------------------------------------------------

