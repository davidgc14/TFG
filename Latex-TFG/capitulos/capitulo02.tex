% !TeX root = ../libro.tex
% !TeX encoding = utf8

\chapter{La completitud del cálculo lógico de primer orden}

Definamos previamente el sistema axiomático sobre el que vamos a trabajar:
\footnote{Coincide, exceptuando el principio de asociatividad (que es redundante), con lo expuesto en los apartados 1 y 10 de 
\textit{Principia Mathematica} (Véase \cite{an1910principia})}
\begin{itemize}
    \item Signos básicos primitivos:\footnote{A partir de ellos pueden definirse $\wedge$, $\rightarrow$, $\leftrightarrow$ y 
            $\exists$ del modo habitual.}
            \begin{itemize}
                \item $\lnot$
                \item $\vee$
                \item $\forall$
            \end{itemize}
    \item Axiomas formales:
            \begin{enumerate}
                \item $ \text{X} \vee \text{X} \rightarrow \text{X} $
                \item $ \text{X} \rightarrow \text{X} \vee \text{Y} $
                \item $ \text{Y} \vee \text{X} \rightarrow \text{X} \vee \text{Y} $
                \item $ (\text{X} \rightarrow \text{Y}) \rightarrow (\text{Z} \vee \text{X} \rightarrow \text{Z} \vee \text{Y}) $
                \item $ \forall \text{x Px} \rightarrow \text{Py} $
                \item $ \forall \text{x } (\text{X} \vee \text{Px}) \rightarrow \text{X} \vee \forall \text{x Px} $
            \end{enumerate}
    \item Reglas de inferencia: \footnote{No todas ellas están explícitamente formuladas en los resultados de Russel y Whitehead,
            pero todas ellas son usadas continuamente en sus deducciones.}
            \begin{enumerate}
                \item El esquema de inferencia: De $\alpha$ y $\alpha \rightarrow \beta$ se puede inferir $\beta$.
                \item La regla de sustitución para variables sentenciales y predicativas.
                \item De $\alpha(\text{x})$ puede inferirse $\forall \text{x } \alpha(\text{x})$.
                \item Unas variables individuales (libres o ligadas) pueden ser reemplazadas por cualesquiera otras, con tal de que 
                        ello no produzca ningún solapamiento de los alcances de las variables designadas mediante el mismo signo.
            \end{enumerate}
\end{itemize}

\section{Exposición y demostraciones}

El teorema de completitud semántica de la lógica de primer orden aparece en el artículo de Gödel de 1930 como Teorema I:

\begin{teorema}\label{thm:teoremaI}
    Cada fórmula válida de la lógica de primer orden es deducible.
\end{teorema}

El presente teorema, objeto principal de estudio de esta sección, sería trivialmente demostrable si pudiesemos probar
el siguiente:
\begin{teorema}\label{thm:teoremaII}
    Cada fórmula de la lógica de primer orden es o refutable
    \footnote{<<$\varphi$ es refutable>> significa <<$\lnot \varphi$ es deducible>>.} o satisfacible (sobre un universo infinito numerable).
\end{teorema}

Y por ello surge el siguiente resultado: 

\begin{proposicion}
    \autoref{thm:teoremaII} $\Rightarrow$ \autoref{thm:teoremaI}
\end{proposicion}
\begin{proof}
    Sea $\alpha$ una fórmula válida. Siendo esto así, $\lnot \alpha$ no es satisfacible, y aplicando \autoref{thm:teoremaII} 
    tenemos que $\alpha$ es refutable. Por tanto, con ello se tiene que $\lnot \lnot \alpha$ (y como consecuencia, también $\alpha$) es una fórmula
    deducible.\footnote{El recíproco del anterior resultado también es cierto y con una demostración igual de simple, aunque no la demostraremos por no ser relevante
    en la demostración del \autoref{thm:teoremaI}.}
\end{proof}

\begin{definicion}
    Una \textit{K-fórmula} es una fórmula $\kappa$ perteneciente a una clase del conjunto de fórmulas $K$ cumpliendo las siguientes condiciones:
    \begin{enumerate}
        \item $\kappa$ es una fórmula prenexa.
        \item $\kappa$ carece de variables individuales libres.
        \item El prefijo de $\kappa$ comienza con un cuantificador universal y termina con un cuantificador particular.
    \end{enumerate}
\end{definicion}

Entonces con la presente definición podemos deducir el siguiente resultado:

\begin{teorema} \label{thm:teoremaIII}
    Si cada $K$-fórmula es refutable o satisfacible, también lo es cualquier fórmula.
\end{teorema}

\begin{proof}
    Sea $\alpha$ una fórmula que no pertenece a $K$. Sea $\mathfrak{x}$ el conjunto de sus variables libres. Como se puede ver
    directamente, si $\alpha$ es refutable (o satisfacible), se sigue la refutabilidad (o satisfacibilidad) de 
    $\exists \mathfrak{x} \alpha$, e igualmente a la inversa. 

    Sea ahora $\pi \varphi$ la forma normal prenexa de $\exists \mathfrak{x} \alpha$, de tal modo que
    \begin{equation}\label{eq:III-1}
        \exists \mathfrak{x} \alpha \leftrightarrow \pi \varphi
    \end{equation}
    es deducible. Además estipulemos que
    \begin{equation}
        \beta = \forall x \pi \exists y (\varphi \wedge \text{Fx} \vee \lnot \text{Fy})
        \footnote{Las variables x e y no deben aparecer en $\pi$.}
    \end{equation}
    
    Entonces
    \begin{equation}\label{eq:III-2}
        \pi \varphi \leftrightarrow \beta
    \end{equation}
    es deducible (por el \autoref{lem:lema-4} la deducibilidad de $\forall x \pi \exists y (\varphi \wedge \text{Fx} \vee \lnot \text{Fy})$).
    $\beta$ pertenece a $K$ y, por tanto, es o satisfacible o refutable. Pero por \eqref{eq:III-1} y \eqref{eq:III-2}
    la satisfacibilidad de $\beta$ implica la de $\exists \mathfrak{x} \alpha$ y consiguientemente también la de $\alpha$,
    y lo mismo se puede aplicar para la refutabilidad. Por tanto, concluimos que también $\alpha$ es o satisfacible o refutable.
\end{proof}

Teniendo en cuenta el \autoref{thm:teoremaIII} anterior, basta para demostrar el \autoref{thm:teoremaII} con probar la siguiente:

\begin{proposicion} \label{prop:satisf-refut}
    Cada $K$-fórmula es satisfacible o refutable.
\end{proposicion}

Para ello definimos previamente el concepto de grado de una $K$-fórmula, y probaremos algunos resultados para poder probar la anterior
proposición.

\begin{definicion}
    Llamaremos grado de una $K$-fórmula
    \footnote{También, en el mismo sentido se le puede llamar "grado de un prefijo".}
    al número de series de cuantificaciones universales de su prefijo, separadas unas de otras
    por cuantificadores existenciales.
\end{definicion}

Probamos primeramente el siguiente resultado.

\begin{teorema}\label{thm:teoremaIV}
    Si cada $K$-fórmula de grado $n$ es o refutable o satisfacible, también lo es cada $K$-fórmula de grado $n+1$.
\end{teorema}




\chapter{El teorema de Incompletitud}

\endinput
%------------------------------------------------------------------------------------
% FIN DEL CAPÍTULO. 
%------------------------------------------------------------------------------------

