% !TeX root = ../libro.tex
% !TeX encoding = utf8
%
%*******************************************************
% Summary
%*******************************************************

\selectlanguage{english}
\chapter{Summary}

This paper presents the history of the theorem that shook the foundations of all known mathematics: Gödel's incompleteness theorems. 
To do so, we will give a brief historical review of what motivated this great mathematician of the twentieth century to deal with this 
peculiar subject, in which we will see that this merit is attributed to David Hilbert and his list of 23 problems. 

At the end of the 19th century, mathematics felt a slight tremor in its foundations, giving rise to the foundational crisis of mathematics, 
and it is here that Hilbert appears with his second problem of the list, in which he poses as an urgent problem the axiomatization of arithmetic. 
What Hilbert did not know was that this problem would never be solved. 

Gödel, after graduating in mathematics, comes across a recently published book by Hilbert and Ackermann in which an open problem of logic is posed: 
proving the completeness of the first-order logical calculus. Emerging from the unknown world, this great up-and-coming mathematician proposes 
an irrefutable solution in less than a year, and in a paper of barely 10 pages in which he states his strong result: 

\begin{center}
    \textit{In a first-order logic, every formula that is valid in a logical sense is provable.}
\end{center}
 
This simple statement promises us the existence of demonstration of any statement of elementary logical formulas, provided that these are in 
infinite numerable quantity. This great result gave much hope to Hilbert, who had hitherto held high hopes for the possibility of solving his 
second problem proposed at the 1900 congress. 

However, what Hilbert could not foresee was that, a year later, Gödel would publish the result that would shake the whole of mathematics, 
and not only that of his time, but forever. In one of the most important articles of twentieth century mathematics, Gödel presents the well-known 
Incompleteness Theorems, by means of a constructive proof based on a gödelization, which is nothing more than an assignment of numbers to 
elements of first-order logic. In them he states as follows:

\begin{center}
    \begin{enumerate}
        \item \textit{No formal mathematical theory is consistent and complete.}
        \item \textit{If the system of axioms of such a formal system is consistent, such consistency cannot be proved by means of these axioms.}
    \end{enumerate}
\end{center}

This result represents a great wall in the advancement of the knowledge of mathematics. In the first instance, we will never be able to create a 
mathematical system in which we can be sure of being able to prove every result, and also that we will not have contradictions in it. 

In this small bibliographic work we intend to collect the original proofs published between 1929 and 1931 by Kurt Gödel himself 
(although later other simpler and improved proposals have been published, I believe that in the original ones we can better understand the 
objectives of the same through a constructive process), in addition to the historical context necessary to understand the motivation of these 
proofs.

% Al finalizar el resumen en inglés, volvemos a seleccionar el idioma español para el documento
\selectlanguage{spanish} 
\endinput
